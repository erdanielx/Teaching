\section{Concepts}

\paragraph{Multi-Threading}
Multi-threading executes multiple parts of a program at the same on the same machine.
These parts are called \textbf{threads}.

Ideally, this uses a separate CPU for each thread to maximize the speed-up.
But there can also be more threads running in parallel than there are CPUs.
In that case, the threads must be scheduled.

\paragraph{Speed-Up}
The speed up is the gain in time complexity if parallel execution is taken into account.

Formally, let $C^k(n)$ is the time complexity for input of size $n$ if $k$ CPUs are available for parallel execution.
In particular, $C^1(n)=C(n)$.
Then the speedup for $k$ machines is defined by $C^k(n)/C(n)$.

\paragraph{Blocking}
If multiple threads are present, a thread may block, i.e., it waits until a certain condition is fulfilled.
This gives a thread the ability to yield CPU access to other threads.

\paragraph{Fairness}
Usually there is no guarantee when or in which order threads gain access to a CPU.
In extreme cases, a thread may be finished completely before the next thread is started.

Fairness refers to a scheduling that gives each thread some guarantee on getting CPU access eventually.

Parallel algorithms are usually designed independent of fairness: they have to yield correct results no matter how the threads are scheduled.

\paragraph{Distributed Algorithm}
If the parts of a program are executed on completely different machines, we speak of a distributed algorithm.

\section{Parallelization Operators}

\subsection{Threads}

Most mainstream programming languages offer multi-threading.
The details vary a lot.

But typically a thread is a class with a method $run():\Unit$.
When $run$ is called, it returns immediately, and the program continues normally.
But from now on, the body of the method $run$ is executed in parallel with the remainder of the program.

\subsection{Parallel List Operations}

List-like data structures are great for parallelization because we often apply the same operation to all elements in a list.

We specify two new operations:

\begin{ctabular}{|l|l|l|}
\hline
function & returns & effect \\
\hline
\multicolumn{3}{|c|}{below, let $l\in A^*$ be of the form $[a_0,\ldots,a_{l-1}]$} \\
$parMap[B](l\in A^*, f:A\to B)\in B^*$ & nothing & $[f(a_0),\ldots,f(a_{l-1})]$ computed in parallel\\
$parForeach(l\in A^*, f:A\to \Unit)\in \Unit$ & $f(a_0),\ldots,f(a_{l-1}$ run in parallel & nothing\\
\hline
\end{ctabular}

\subsection{Parallel Composition}

Let $C$ and $D$ be functions that take no arguments.
Then $C|D$ is the command that runs $C()$ and $D()$ in parallel.

This can be seen as a special case of a $parForeach$ using the list $[C,D]$.

\section{Examples}

\subsection{Parallel Depth-First Search}

Often we can turn a normal algorithm into a parallel algorithm by replacing $map$ with $parMap$ or $foreach$ with $parForeach$.

For example, we obtain a parallel DFS traversal of a tree as follows:
\begin{acode}
\afun{DFS[A]}{n: Tree[A], f:Tree[A]\to \Unit}{
  f(n)\\
  parForeach(n.children, x \mapsto DFS[A](x,f))
}
\end{acode}

This executes $f(n)$ in parallel for every node $n$.
Of course, the actual order in which nodes are visited is not entirely predicable anymore.
It is still guaranteed that $f(n)$ terminates before $f(c)$ begins for any child of $n$.
But it is unpredictable in which order the children are processed.


\subsection{Associative Folding}\label{sec:ad:monoidfold:parallel}

Consider the associative folding problem from Sect.~\ref{sec:ad:monoidfold:divide}.
The divide-and-conquer algorithm did not lower the complexity.

But if we obtain a speedup if we recurse in parallel:

\begin{acode}
\afun[A]{monoidFold[A]}{mon: Monoid[A], x: List[A]}{
 \aifelse{empty(x)}{mon.e}{
   n := length(x)\\
   i := n \divop 2\\
   lower := [x_0,\ldots,x_{i-1}]\\
   upper := [x_i,\ldots,x_{n-1}]\\
   lowerFold := monoidFold(mon,lower) \tb | \tb upperFold := monoidFold(mon,upper)\\
   mon.op(lowerFold, upperFold)
 }
}
\end{acode}

If we have more CPUs than $n=length(x)$, this runs in $\Theta(\log n)$.

%More generally, we have $C^k(n)\in \Theat(\log n)+\Theta(n/k)
