\documentclass{book}

\usepackage{graphicx}
\usepackage{xkeyval}
\usepackage{multirow}
%\usepackage{bm} %% bold face math symbols
\usepackage{listings}
\usepackage{../macros/mytikz}
\usepackage{multicol}
%\usepackage{stmaryrd}%\newcommand{\contra}{\lightning}
%\usepackage{rotating} \newcommand{\sw}[1]{\begin{sideways}#1\end{sideways}}

\usepackage{../macros/algorithm}
%\usepackage{ded}

\usepackage{../mylecturenotes}

\title{Lectures Notes on Data Structures and Algorithms}
\author{Florian Rabe}
\date{2017}

\begin{document}
\maketitle


\tableofcontents

\bigskip

These notes were originally prepared for my 2nd semester CS course at Jacobs University in Spring 2017.
In that course, the following chapters were skipped or treated only very superficially: \ref{sec:ad:finiteds}, \ref{sec:ad:numbers}, \ref{sec:ad:option}, \ref{sec:ad:functions}, \ref{sec:ad:unions}, \ref{sec:ad:parallel}, \ref{sec:ad:prot}, \ref{sec:ad:random}, \ref{sec:ad:quantum}.
In the other chapters, almost all material was covered; only a few subsections were skipped.
\newpage

% % % % % % % % % % % % % % % % % % % %
\part{Introduction and Foundations}

\chapter{Meta-Remarks}
 \begin{center}
\textbf{Important stuff that you should read carefully!}
\end{center}

\paragraph{State of these notes}
I constantly work on my lecture notes.
Therefore, keep in mind that:
\begin{compactitem}
\item I am developing these notes in parallel with the lecture---they can grow or change throughout the semester.
\item These notes are neither a subset nor a superset of the material discussed in the lecture.
\item Unless mentioned otherwise, all material in these notes is exam-relevant (in addition to all material discussed in the lectures).
\end{compactitem}
\medskip

\paragraph{Collaboration on these notes}
I am writing these notes using LaTeX and storing them in a git repository on GitHub at \url{https://github.com/florian-rabe/Teaching}.
Familiarity with LaTeX as well as Git and GitHub is not part of this lecture. But it is essential skill for you.
Ask in the lecture if you have difficulty figuring it out on your own.
\medskip

As an experiment in teaching, I am inviting all of you to collaborate on these lecture notes with me.
\medskip

By forking and by submitting pull requests for this repository, you can suggest changes to these notes.
For example, you are encouraged to:
\begin{compactitem}
\item Fix typos and other errors.
\item Add examples and diagrams that I develop on the board during lectures.
\item Add solutions for the homeworks if I did not provide any (of course, I will only integrate solutions after the deadline).
\item Add additional examples, exercises, or explanations that you came up or found in other sources.
 If you use material from other sources (e.g., by copying an diagram from some website), make sure that you have the license to use it and that you acknowledge sources appropriately!
\end{compactitem}
The TAs and I will review and approve or reject the changes.
If you make substantial contributions, I will list you as a contributor (i.e., something you can put in your CV).
\medskip

Any improvement you make will not only help your fellow students, it will also increase your own understanding of the material.
Therefore, I can give you up to $10\%$ bonus credit for such contributions.
(Make sure your git commits carry a user name that I can connect to you.)
Because this is an experiment, I will have to figure out the details along the way.

\paragraph{Other Advice}
I maintain a list of useful advice for students at \url{https://svn.kwarc.info/repos/frabe/Teaching/general/advice_for_students.pdf}.
It is mostly targeted at older students who work in individual projects with me (e.g., students who work on their BSc thesis).
But much of it is useful for you already now or will become useful soon.
So have a look.

\paragraph{Skipped Chapters}
These notes were originally prepared for my 2nd semester CS course at Jacobs University in Spring 2017.
In that course, the following chapters were skipped or treated only very superficially: \ref{sec:ad:finiteds}, \ref{sec:ad:numbers}, \ref{sec:ad:option}, \ref{sec:ad:functions}, \ref{sec:ad:unions}, \ref{sec:ad:parallel}, \ref{sec:ad:prot}, \ref{sec:ad:random}, \ref{sec:ad:quantum}.
In the other chapters, almost all material was covered; only a few subsections were skipped.

\chapter{Basic Concepts}\label{sec:ad:fund}
  These lecture notes do not follow a particular textbook.

Students interested in additional literature may safely use \cite{cormen_algorithms} (available online), one of the most widely used textbooks.
Knuth's book series on the Art of Computer Programming \cite{knuth_art}, although not usually used as a modern textbook, is also interesting as the most famous and historically significant book on the topic.

\section{What are Data Structures and Algorithms?}\label{sec:ad:whatare}

Data structures and algorithms are among the most fundamental concepts in computer science.

\subsection{Static vs. Dynamic}\label{sec:ad:static}

In all areas of life and science, we often find a pair of concept such that one concept captures static and the other one dynamic aspects.
This is best understood by example:

\begin{center}
\begin{tabular}{|l||l|l|}
\hline
area & static & dynamic \\
\hline
\hline
\multicolumn{3}{|c|}{in life}\\
\hline
existence & be & become \\
\hline
events & situation & development \\
\hline
food & ingredients & cooking \\
\hline
\hline
\multicolumn{3}{|c|}{in science}\\
\hline
mathematics & sets & functions \\
\hline
physics & space & time \\
\hline
chemistry & molecules & reactions \\
\hline
engineering & materials & construction \\
\hline
\hline
\multicolumn{3}{|c|}{\textbf{in computer science}}\\
\hline
hardware & memory & processing \\
\hline
abstract machines & states & transitions \\
\hline
programming  & types & functions \\
\hline
\textbf{software design} & \textbf{data structures} & \textbf{algorithms} \\
\hline
\end{tabular}
\end{center}

The static aspects describes things as they are at one point in time.
The dynamic aspects describes how they change over time.

Data structures and algorithms have this role in software design.
Data structures are sets of objects (the data) that describe the domain that our software is meant to be used for.
Algorithms are operations that describe how the objects in that domain change.

\subsection{Basic Definition and Examples}\label{sec:ad:basicdef}

\begin{definition}[Data Structure]\label{def:ad:ds}
Assume some set of effective objects.

A data structure defines a subset of these objects by providing effective methods for determining
\begin{compactitem}
 \item whether an object is in the data structure or not,
 \item whether two objects are equal.
\end{compactitem}
\medskip

In practice, a data structure is often bundled with several algorithms for it.
\end{definition}

\begin{definition}[Algorithm]\label{def:ad:algo}
An algorithm consists of
\begin{compactitem}
\item a data structure that defines the possible input objects
\item a data structure that defines the possible output objects
\item an effective method for transforming an input object into an output object
\end{compactitem}
\end{definition}

These definitions are not very helpful---they define the words ``data structure'' and ''algorithm'' by using other not-defined words, namely ``effective object'' and ``effective method''.
Let us look at some examples before discussing effective objects and methods in Sect.~\ref{sec:ad:effective}.

\begin{example}[Natural Numbers]
The most important data structure are the natural numbers.

It is defined as follows:
\begin{compactitem}
 \item The string $0$ is a natural number.
 \item If $n$ is a natural number, then the string $s(n)$ is a natural number.
 \item All natural numbers are obtained by applying the previous step finitely many times, and these are all different.
\end{compactitem}

We immediately define the usual abbreviations $1,2,\ldots,$.
It is also straightforward to define algorithms for the basic functions on natural numbers such as $m+n$, $m-n$, $m*n$, etc.
\end{example}

\begin{example}[Euclidean Algorithm]\label{ex:ad:euclid}
The Euclidean algorithm (see also Sect.~\ref{sec:ad:history}) computes the greatest common divisor $\gcd(m,n)$ of two natural numbers $m,n\in\N$.
It consists of the following components:
\begin{compactitem}
\item input: $\N\times\N$
\item output: $\N$
\item effective method:
\begin{acode}
\afun[\N]{gcd}{m:\N,n:\N}{
x := m \acomment{introduce variables, initialize with input data}\\
y := n \\
%% loop invariant gcd(m,n)=gcd(x,y)
\awhile[repeat as long as $\gcd(x,y)\neq x$]{x\neq y}{
  \aifelse[subtract the smaller number from the bigger one, which does not affect $\gcd(x,y)$]{x<y}{y := y-x}{x := x-y}
}\\
\areturn[now trivially $\gcd(x,y)=x$]{x}
}
\end{acode}
\end{compactitem}

The algorithm starts by introducing variables $x$ and $y$ and initializes them with the input data $m$ and $n$.
Then it repeatedly subtracts the smaller number from the greater one until both are equal.
This works because $\gcd(x,y)=\gcd(x-y,y)$.
If $x$ and $y$ are equal, we can return the output because $\gcd(x,x)=x$.
\medskip

This algorithm has a subtle bug (Can you see it?) that we will fix in Ex.~\ref{ex:ad:euclid:term}.
\end{example}

For a simpler example, consider the definition of the factorial $n!=1\cdot \ldots n$ for $n\in\N$.

\begin{example}[Factorial]\label{ex:ad:factorial}
The factorial can be defined as follows:
\begin{compactitem}
\item input: $\N$
\item output: $\N$
\item effective method:
\begin{acode}
\afun[\N]{fact}{n:\N}{
product := 1\\
factor  := 1 \\
%% loop invariant n! = product * factor * ... * n
\awhile{factor \leq n}{
  product := product \cdot factor\\
  factor := factor+1
}\\
\areturn{product}
}
\end{acode}

Here the variable $factor$ runs through all values from $1$ to $n$ and the variable $product$ collects the product of those values.
\end{compactitem}
\end{example}

\begin{notation}
It is convenient to give the effective method of an algorithm as a function definition using pseudo-code.
That way the input and output do not have to be spelled out separately because they are clear from the data structures used in the header of the function definition.
\end{notation}

\subsection{Effective Objects and Methods}\label{sec:ad:effective}

It is now a central task in computer science to define data structures and algorithms that correspond to given sets and functions.
This question that was first asked by David Hilbert in 1920, one of the most influential mathematicians at the same time.
In modern terminology, he wanted to define data structures for all sets and algorithms for all functions and then machines to mechanize all mathematics.

In the 1930s, several scientist worked on this problem and eventually realized that it cannot be done.
These scientists included Alonzo Church, Kurt G\"odel, John von Neumann, and Alan Turing.
Their work provided partial solutions and theoretical limits to the problem.
In retrospect, this was the birth of computer science.

Not every set and not every function can be represented by a data structure or an algorithm (see Sect.~\ref{sec:ad:computable} for the reason why not).
That limitations bring us back to the question of effective objects and methods:

\begin{definition}[Effective Object]
An effective object is any object that can be stored, manipulated, and communicated by a physical machine.

Here, \emph{physical} means any machine that we can build in the physical world.\footnotemark
\end{definition}
\footnotetext{Sometimes we use hypothetical machines. For example, quantum computers are physical machines that we think we can build but have not been able to build in practice at useful scales yet.}

Thus, every physical machine defines its own kind of effective objects.
All digital machines (which includes all modern computers) use the same effective objects: lists of bits.
These are stored in memory or on hard drives, which provide essentially one very, very long list of bits.

Data structures use fragments of these lists to represents sets.
For example, the set $\Z_{2^{32}}$ of $32$-bit-integers is represented by a list of $32$ bits.

\begin{definition}[Effective Method]
An effective method consists of a sequence of instructions such that
\begin{compactitem}
 \item any reasonably intelligent human can carry out the instructions
 \item and all such humans will carry out the instructions in exactly the same way (in particular reaching the same result).
\end{compactitem}
\end{definition}

The first condition makes sure that any prior knowledge needed to understand the instructions is be explicitly stated or referenced.
The second conditions makes sure that an effective method has a well-defined result: There may be no ambiguity, randomness, or unspecified choice.

\begin{example}
The third condition excludes for example the following instructions
\begin{compactitem}
 \item ``Let $x$ be the factorial of $5$.'': Different humans could compute the factorial differently because it is not clear which algorithm to use for the factorial.
 \item ``Let $x$ be a random integer.'': Randomness is not allowed.
 \item ``Let $x$ be an element of the list $l$.'': It is not specified which element should be chosen.
\end{compactitem} 
\end{example}

\subsection{History}\label{sec:ad:history}

One of the earliest and most famous (arguably \emph{the} earliest) algorithms is Euclid's algorithm for computing the greatest common divisor (see Ex.~\ref{ex:ad:euclid}).
It is given around 300 BC in Euclid's Elements \cite[Book VII, Proposition 2]{elements}, maybe the most influential textbook of all time.
\medskip

The word \emph{algorithm} is much younger.
It is derived from the name of the 9th century scientist al-Khwarizmi.
He was one of the most important scientists of his millennium but is relatively unknown in the Western world because he was and wrote in Arabic.
Translations of his work on arithmetic in the 12th century spread several new results to the Western world.

This included the use of numbers as abstract objects as opposed to geometric distances that had dominated Europe since the work of the Greek mathematicians (such as Euclid).
It also included the positional number system and the base-$10$ digits that are still in use today.
The corresponding arithmetical operations on numbers were named \emph{algorismus} after him in Latin, which developed into the modern word.
He also worked on algorithms for solving linear and quadratic equations, and one of his basic operations called \emph{al-jabr} gave rise to the word \emph{algebra}.
\medskip

The modern \emph{meaning} of the word \emph{algorithm} is even younger: Its formalization was effected by a major development in the 1920s and 1930s that eventually gave to modern computer science itself.
Hilbert was the most influential mathematician in the early 20th century.
One of his legacies was to call for solutions to certain fundamental problems \cite{hilbertsproblems}.
Another legacy was his program \cite{hilbertsprogram}, a call for the formalization of mathematics that (among other things) should yield an algorithm for determining whether any given mathematical formula is a theorem.

Hilbert's program inspired seminal work by (among others) Alonzo Church, Kurt G\"odel, and Alan Turing.
This led to several concrete definitions of \emph{algorithm}, including Turing-machines and the $\lambda$-calculus, from which all modern programming languages are derived.
It also led to an understanding of the limits of what algorithms can do (see Sect.~\ref{sec:ad:computable}), which has led to the modern theory of computation.

\subsection{The Limits of Data Structures and Algorithms}\label{sec:ad:computable}

\subsubsection{Countability of Data Structures and Algorithms}

We can now see immediately why not all mathematical objects are effective in digital machines: There are only countably many lists of bits.
Therefore, there can only be countably many effective objects.

Similarly, any data structure we define must be defined as a list of characters in some language.
But there are only countably many such lists.
Therefore, there can only be countably many data structures.
For the same reason, there can only be countably many algorithms.
\medskip

Inspecting the sizes of the constructed sets from Sect.~\ref{sec:math:sets}, we can observe that
\begin{compactitem}
\item If all arguments are finite, so is the constructed set---except for lists.
\item If all arguments are at most countable, so is the constructed set---except for function and power sets.
\end{compactitem}
Because of these exceptions, we cannot restrict attention to finite or countable sets only---working with them invariably leads to uncountable sets.

\subsubsection{Computability}

At best, we can hope to give data structures for all countable sets.
But not even that is possible.
Because countable sets have uncountably many subsets, we cannot give data structures for every subset of every countable set.

Therefore, we give the sets that have data structures a special name:

\begin{definition}[Decidable]
A set is called \textbf{decidable} if we can give a data structure for it.
\end{definition}

Similarly, at best we can hope to give algorithms for all functions between decidable sets.
Again that is not possible.
Because countable sets have uncountably many functions between them, we cannot give algorithms for all functions between decidable sets.

Therefore, we give the sets that have data structures a special name:

\begin{definition}[Computable]
A function between decidable sets is called \textbf{computable} if we can give an algorithm for it.
\end{definition}

At Jacobs University, decidability and computability are discussed in detail in a special course in the $2nd$ year.

\subsubsection{The Role of Programming Languages}

\paragraph{Vagueness of the Definitions}
It is not possible to precisely define effective objects and methods---every definition eventually uses not-defined concepts like ``machine'' or ``instruction''.
Thus, it impossible to precisely define data structures and algorithms are.
Instead, we must assume those concepts to exist a priori.

That may seem flawed---but it is actually very normal.
We can compare the situation to physics where there is also no precise definition of \emph{space} and \emph{time}.
In fact, the question what space and time are is among the difficult of all of physics.\footnotemark
\footnotetext{For example, even today physicists have no agreed-upon answer to the question why time moves forwards but not backwards.}

Similarly, the question of what data structures and algorithms are is among the most fundamental of computer science.
Every computer and evey programming language give their own answer to the question.

\paragraph{Data Description and Programming Languages}
To make the definitions of \emph{data structure} and \emph{algorithm} precise, we have to choose a concrete formal language.

\begin{definition}[Languages]
A \textbf{data description language} is a formal language for writing objects and data structures.

A \textbf{programming language} is a formal language for writing algorithms.
\end{definition}

Because algorithms require data structures, every programming language includes a data description language.
And because all data structures usually come with specific algorithms, we are usually mostly interested in programming languages.

But there are some languages that are pure data description languages.
These are useful when storing data on hard drives or when exchanging data between programs and computers (e.g., on the internet).
Examples of pure data description languages are JSON, XML, HTML, and UML.

\paragraph{Types of Programming Languages}
Programming languages can vary widely in how they represent data structures.

We can distinguish several groups:
\begin{compactitem}
\item Untyped languages avoid explicit definitions of data structures.
Instead, they use algorithms such as $\mathit{isNat}$ to check, e.g., if an object is a natural number.\\
Examples are Javascript and Python.
\item Functional languages focus on using inductive data types.\\
Examples are SML and Haskell.
\item Object-oriented languages focus on using classes.\\
Examples are Java and C++.
\item Multi-paradigm languages combine functional and object-oriented features.\\
Examples are Scala and F\#.
\end{compactitem}


\paragraph{Independence of the Choice of Language}
Above we have seen that the concrete meaning of \emph{data structure} and \emph{algorithm} seems to depend on the choice of programming language.
Thus, it seems that whether a set is decidable or a function computable also depends on the choice of programming language.

One of the most amazing and deepest results of theoretical computer science is that this is not the case:

\begin{theorem}[Church-Turing Thesis]
All known programming languages (including theoretical ones such as Turing machines)
\begin{compactitem}
\item can define data structures for exactly the same sets,
\item can define algorithms for exactly the same functions.
\end{compactitem}

Thus, it does not depend on the chosen programming language
\begin{compactitem}
\item whether a set is decidable,
\item whether a function is computable.
\end{compactitem}
\end{theorem}
\begin{proof}
The proof is very complex.
For every program of every language, we must provide an equivalent program in every other language.

However, this can be done (and has been done) for all languages.
\end{proof}
%This should only hold for Turing complete languages (and some especially nice programming languages like e.g. Agda are not Turing complete). Non Turing complete languages can not nessecarily construct all the set a Turing complete language can. For instance in Agda (and some other total functional programming languages) you can only implement algorithms (including constructors of data types) for programs proven to be terminating. According to the halting problem this implies, that there are some algorithms that you can define on a Turing machine, but not in Agda. 

A related (stronger) theorem is that every programming language $P$ allows defining for every programming language $Q$ a program that executes $Q$-programs.

It is generally believed but impossible to prove that there is no programming language that can define more data structures or algorithms than the known ones.

\section{Specification vs. Design vs. Implementation}

Above we have seen sets and functions as well as data structures and algorithms.
Moreover, we have already mentioned programs consisting of types and functions.

The following table gives a overview of the relation between these concepts:

\begin{ctabular}{|l|l|l|}
\hline
Specification & Design/Architecture & Implementation \\
\hline
\hline
sets          & data structures & types    \\
functions     & algorithms      & functions\\
\hline
\end{ctabular}

Software development consists of $3$ steps:
\begin{enumerate}
\item The \textbf{specification} describes the intended behavior in terms of mathematical sets and functions.\\
It does not prescribe in any way how these sets and functions are realized.
The same specification can have multiple different correct realizations differing among others in size, maintainability, or efficiency.\\
A good specification should be:
\begin{compactitem}
  \item adequate: actually describe the problem that needs solving
  \item simple: easy to understand
  \item unambiguous: impossible to misunderstand
  \item consistent: possible to realize
  \item (optionally) complete: no freedom in what it means (An incomplete specification is not necessarily a flaw. For example, one might specify a function on integers without saying what should happen for negative input.)
\end{compactitem}
\item The \textbf{architecture} makes concrete choices for the data structures and algorithms that realize the needed sets and functions.\\
It usually defines many auxiliary data structures and algorithms that are not part of the specification.\\
The architecture does not prescribe a programming language. It can be correctly realized in any programming language.
\item The \textbf{implementation} chooses a programming languages and then writes a \textbf{program} in it that realizes the architecture.
The program includes concrete choices for the type and function definitions that realize the needed data structures and algorithms.\\
It usually defines many auxiliary types and functions that are not part of the architecture.
\end{enumerate}

\begin{terminology}
\emph{Design} and \emph{architecture} can usually be used synonymously.

The words \emph{specification}, \emph{design}, and \emph{implementation} can refer to both the process and the result.
For example, we can say that the result of implementation is one implementation.
\end{terminology}

It is critical to distinguish the three steps in software development:
\begin{itemize}
 \item Specification changes are much more expensive than design changes.
 Changing the specification may completely change, which design is appropriate.
 Therefore, every single design decision must be revisited and checked for appropriateness.
 \item Design changes are much more expensive than implementation changes.
 Changing the design may completely change which components of the implementation are needed and how they interact.\\
 Therefore, every part of the program must potentially be revisited. \\
 In particular, whenever the design of component $X$ is changed, we have to revisit every place of the program that uses $X$. 
 This often introduces bugs.
\end{itemize}
Typically any specification change entails bigger design changes, and any design change entails bigger implementation changes.
Moreover, specification changes require
\begin{compactitem}
	\item re-verification (i.e., checking that the implementation still correctly implements the specification)
	\item re-certification by regulatory agencies (if applicable to the specific software)
	\item changes to documentation, manuals, and tutorials, re-training of users, etc.
	\item distribution of software updates, which confuses and disrupts their workflows
	\item need for other software projects to adapt to the updated software
\end{compactitem}

An ideal programmer proceeds in the order specification-design-implementation.
However, it is often necessary to loop back: The design phase may reveal problems in the specification, and the implementation phase may reveal problems in the design.
Therefore, we usually have to work on all $3$ parts in parallel---but with a strong preference against changing specification and design.

Many self-taught or not-well-taught programmer do not understand the difference between the $3$ steps or do not systematically apply it.
There are many such programmers, who never studied CS or got a degree without taking a rigorous foundations course.
Their programs are typically awful because:
\begin{compactitem}
 \item They begin programming without writing down the specification.
 Consequently, they do not realize that they have not actually understood the specification.
 This results in programs that do not meet the specification, which then leads to retroactive changes to the design.
 Over time the program becomes (sometimes called ``spaghetti code'') that is unmaintainable and cannot be understood by other programmers, often not even by the programmer herself.
 \item They begin programming without consciously choosing a design.
 Consequently, they end up with a random design that may or may not be appropriate for the task.
 Over time they change the design multiple times (without being aware that they are changing the design).
 Each change introduces new bugs and more mess.
\end{compactitem}

\begin{example}[Greatest Common Divider]\label{ex:ad:euclid2}
The specification of the greatest common divider function $\gcd$ is as follows:
Given natural numbers $m$ and $n$, return a natural number $g$ such that
\begin{compactitem}
\item $g|m$ and $g|n$
\item for every number $h$ such that $h|m$ and $h|n$ we have that $h|g$
\end{compactitem}
\medskip

Before we design an algorithm, we should check whether $\gcd$ is indeed a function:
\begin{compactitem}
  \item Consistency: Does such a $g=\gcd(m,n)$ always exists?
  \item Uniqueness: Could there be more than one such $g$?
\end{compactitem}
Using mathematics, we can prove that $g$ indeed exists uniquely.
\medskip

Now we design an algorithm.
Let us assume that we have already designed data structures for the natural numbers with the usual operations.
There are many reasonable algorithms, among them the one from Ex.~\ref{ex:ad:euclid}.
For the sake of example, we use a different one here:
\medskip

\begin{acode}
\afun[\N]{gcdRec}{m:\N, n:\N}{
  \aifelse{n==0}{m}{\gcd(n,m\modop n)}
}
\end{acode}

\medskip
This is a recursive algorithm: The instructions may recursive call the algorithm itself with new input.
\medskip

Finally, we implement the algorithm.
We choose SML as the programming language.
First we implement the data structure for natural numbers and the function $mod:nat*nat \to nat$ that were assumed by the specification.
Note that this requires some auxiliary functions that were not part of the algorithm:
\begin{lstlisting}
datatype nat = zero | succ of nat

fun leq(m: nat, n: nat): bool = case (m,n) of
  (zero,    zero)    => true
| (zero,    succ(y)) => true
| (succ(x), zero)    => false
| (succ(x), succ(y)) => leq(x,y)

fun minus(m: nat, n: nat): nat = case (m,n) of
  (zero,    zero)    => zero
| (zero,    succ(y)) => zero (* error case, should not happen *)
| (succ(x), zero)    => succ(x)
| (succ(x), succ(y)) => minus(x,y)
  
fun mod(m:nat, n:nat):nat =
  if m = n then zero
  else if leq(m,n) then m
  else mod(minus(m,n), n)
\end{lstlisting}
 
Then we define
\begin{lstlisting}
fun gcdRec(m:nat, n: nat): nat = if n = zero then m else gcdRec(n,mod(m,n))
\end{lstlisting}
\end{example}

\section{Stateful Aspects}

\subsection{Immutable vs. Mutable Data Structures}

Consider a data structure for the set $\N^*$ of lists of natural number and assume we have a variable $x:\N^*$.

\subsubsection{Immutable Data Structures and Call-by-Value}

We can always assign a new value to $x$ as a whole.
For example, after executing $x:=[1,3,5]$, we have the following data stored in memory:
\begin{amemory}
\avar{x}{\N^*}{[1,3,5]}
\alocations
\aloc{P}{[1,3,5]}
\end{amemory}
Here the left part shows the variables as seen by the programmer.
The right part shows the objects as they are maintained in memory by the programming language.
$P$ is some name for the memory location holding the value of $x$.
Importantly, the programmer is completely unaware of the organization of the data in memory and only sees the value of $x$.

In particular, $x$ is just an abbreviation for the value $[1,3,5]$.
If we pass $x$ to a function $f$, there is no difference between saying $f(x)$ and $f([1,3,5])$.
That is called \textbf{call-by-value}.

For example, if we execute the instruction $y = delete(x,2)$, we obtain:
\begin{amemory}
\avar{x}{\N^*}{[1,3,5]}
\avar{y}{\N^*}{[1,3]}
\alocations
\aloc{P}{[1,3,5]}
\aloc{Q}{[1,3]}
\end{amemory}
All old data is as before.
For the new variable $y$, a new memory location $Q$ has been allocated and filled with the result of the operation.
This has the drawback that the entire list was duplicated, and we now use twice as much memory as before.

Immutable data structures and call-by-value are the usual way how functions work in mathematics.
Such data structures are closely related to their specification and make writing, understanding, and analyzing algorithms very easy.

\subsubsection{Mutable Data Structures and Call-by-Name}

If our data structure is mutable, the value of a variable $x$ is just a reference to the memory location where the value is stored.

For example, after executing $x:=[1,3,5]$, we have the following data stored in memory:
\begin{amemory}
\avar{x}{\N^*}{P}
\alocations
\aloc{P}{[1,3,5]}
\end{amemory}
The value of $x$ is now the reference to the memory location.
The programmer still cannot see $P$ directly.%
\footnote{Some programming languages allow explicitly creating and manipulating these references.
The most notable example is $C$ (where the references are called \emph{pointers}).
With a few caveats (most importantly that it can allow for maximal optimization), that can be considered a design flaw in the programming language.}

But there are two carefully-designed ways how $P$ can be accessed indirectly.
Firstly, we can assign new values to each component of $x$.
For example, after $x.1:=4$, the memory looks like
\begin{amemory}
\avar{x}{\N^*}{P}
\alocations
\aloc{P}{[1,4,5]}
\end{amemory}
The old value at location $P$ is gone and has been replaced by the new value.

Secondly, when we pass $x$ to a function $f$, we pass the reference to the value, not the value itself.
This is called \textbf{call-by-name} or \textbf{call-by-reference}.

For example, after executing $delete(x,2)$, we have
\begin{amemory}
\avar{x}{\N^*}{P}
\alocations
\aloc{P}{[1,4]}
\end{amemory}
No additional memory location has been allocated for the result, and no copying took place.
That makes the operation much more time- and memory-efficient.
But from a mathematical perspective, this is very odd: The function call $delete(x,2)$ \emph{changed} the value of $x$ under the hood.
\medskip

In many programming languages (in particular object-oriented ones), mutable data structures are called \emph{classes}.
Some functions involving a mutable data structure will make use of mutability, some will not.
This must be part of the specification of each function.

\subsection{Environments and Side Effects}

So far we have said that algorithms realize mathematical functions.
That makes algorithms very close to the specification and makes writing, understanding, and analyzing them very easy.
But it is not the whole picture in computer science---computer science needs a generalization:

\begin{definition}[Stateful Functions]
Let $E$ be the set of environments.
An \textbf{effectful function} from $A$ to $B$ is a function $A\times E\to B\times E$.
\end{definition}

Again this is a vague definition because the word ``environment'' is not defined.
That is normal---there is no universally recognized definition for it.
Intuitively, an object $e\in E$ represents the state of the environment.
$e$ contains all information that is visible from the outside of our algorithms and that can be acted on by the algorithm.
These usually include the global variables, all kinds of input/output, threads, and exceptions.

An effectful function $f$ from $A$ to $B$ can do two things besides returning a result of type $B$:
\begin{compactitem}
 \item It can use the environment (because $E$ occurs in its input type).
   Thus, calling $f$ twice on the same $a\in A$ may return different results if the environment has changed in between.\\
   Formally, if $f(a,e_1)=(b_1,e_1')$ and $f(a,e_2)=(b_2,e_2')$ always implies $b_1=b_2$, we say that $f$ is \textbf{environment-independent}.
 \item It can change the environment (because $E$ occurs in its output type).
   Thus, programmers must be careful when to call $f$ and how often to call $f$ because every call may have an effect that can be observed by the user.\\
   Formally, if $f(a,e)=(b,e')$ always implies $e=e'$, we say that $f$ is \textbf{side-effect-free}.
\end{compactitem}
If $f$ is both environment-independent and side-effect-free, $f$ is called \textbf{pure}.
In that case, we always have $f(a,e)=(g(a),e)$ for some function $g:A\to B$, i.e., we can ignore environments entirely.
Thus, pure functions are essentially the same as the usual mathematical functions.

An environment $e\in E$ is usually a big tuple containing among others
\begin{compactitem}
 \item the current values of all accessible mutable variables
 \item console input/output:
   \begin{compactitem}
      \item the list of characters to be printed out to the user
      \item the list of characters typed by the user that are available for reading
   \end{compactitem}
 \item file and peripheral network input/output: for every open file, network connection or similar
   \begin{compactitem}
      \item the list of data to be written to the connection
      \item the list of data that is are available for reading
   \end{compactitem}
 \item information about exceptions
   \begin{compactitem}
      \item by depending on this aspect of the environment, effectful functions can handle exceptions
      \item by effecting this aspect of the environment, effectful functions can raise exceptions
   \end{compactitem}
 \item the set of currently active threads
 \item additional components depending on the features of the respective programming language
\end{compactitem}

Environment-dependency and side effects are important.
Without input/output side effect, the user could never provide input for algorithms and could never find out what the output is.
Moreover, computers could not be used to read sensor data or control peripheral devices.

But they also present major challenges to algorithm design.
Because the precise definition of $E$ depends on the details of the programming language, it is very difficult to precisely specify effectful functions.
And without a precise specification, the programmer never knows whether an algorithm is designed and implemented correctly.
Therefore, some programming languages such as Haskell try to systematically restrict environment-dependency and side-effects as much as possible.

\section{Parametric Polymorphism}

Many important data structures and algorithms are polymorphic in the following sense:

\begin{definition}[Polymorphism]
A \textbf{polymorphic data structure} $D$ is an operator that maps data structures $D_1,\ldots,D_n$ to a data structure $D[D_1,\ldots,D_n]$.

A \textbf{polymorphic algorithm} $F$ is an operator that maps data structures $D_1,\ldots,D_n$ to an algorithm $F[D_1,\ldots,D_n]$.

The $D_i$ are called the \textbf{type parameters} or \textbf{type arguments} of the data structure/algorithm.
\end{definition}

This is best understood by example:

\begin{example}[Lists]
Lists are a polymorphic data structure.
$A^*$ is the set of lists whose elements have type $A$.
Any data structure for $A^*$ should take $A$ as a type parameter.

For example, $List[A]$ may be a data structure such that $List[\Int]$ is the type of lists of integers.

Most algorithms about lists are polymorphic as well.
For example, reversing a list can be realized using an algorithm
\begin{acode}
\afun[{List[A]}]{reverse[A]}{x: List[A]}{
  \ldots
}
\end{acode}
\end{example}

\begin{terminology}
There are many different concepts of polymorphism that are (correctly, confusingly, or even wrongly) called \emph{polymorphism}.
The special kind described here is usually called \emph{parametric polymorphism}.

Both terminology and notations vary across programming languages, communities, and textbooks.
\end{terminology}

A more difficult example arises if we want to sort a list: To sort a list over $A$, we need a comparison function $\leq(x:A,y:A):\Bool$.
Moreover, $\leq$ has to be a total order.
We can handle that using abstract classes:

\begin{example}
Consider the following polymorphic abstract class for total orders:
\begin{acode}
\aclassA{TotOrd[A]}{}{}{}{
  \afun[\Bool]{lessOrEqual}{x:A,y:A}{}
}
\end{acode}
It requires a function $lessOrEqual$ that provides the comparison $\leq$.
The axioms for being a total order can usually not be programmed---they can only be added as part of the documentation.

Then a polymorphic sorting algorithm could look like
\begin{acode}
\afun[{List[A]}]{sort[A]}{ord: TotOrd[A], x:List[A]}{
\ldots
}
\end{acode}
\end{example}

\begin{notation}[Omitting Type Parameters]
Most of the time it is possible to omit the type parameters when calling a polymorphic function without ambiguity.
For example, if $l: List[\Int]$, we can simply say $revert(l)$ instead of $revert[\Int](l)$---both human readers and compiler can infer the type argument.

Most programming languages that allow polymorphism also allow omitting parameters if they can be inferred uniquely.
It is also allowed to do so in examples and pseudo-code.
\end{notation}

\subsubsection{In Programming Languages}

Even though polymorphism is relatively simple mathematically, not all programming languages do a good job of implementing it.
Therefore, we will often gloss over issues of polymorphism when giving algorithms.

But we give a few examples of polymorphism in a few typed programming languages.

\paragraph{Scala}
Scala's syntax is very similar to te pseudo-code used in these notes:

\begin{lstlisting}
abstract class TotOrd[A] {
  def lessOrEqual(x:A, y:A): Boolean
}

object IntSmaller extends TotOrd[Int] {
  def lessOrEqual(x:Int, y:Int): Boolean = x <= y
}

object Sort {
  def sort[A](ord: TotOrd[A], x: List[A]): List[A] = {
    ...
  }
}

object Test {
  def main(args: Array[String]) {
    sort[Int](IntSmaller, List(4,3,5))
  }
}
\end{lstlisting}

\paragraph{Java}
In Java, polymorphic data structures are called \emph{generics}.
It uses angular instead of square brackets and puts the parameter types of a polymorphic algorithm before the return type instead of after the name:

\begin{lstlisting}
interface TotOrd<A> {
  public Boolean lessOrEqual(A x, A y);
}

class Sort {
  static <A> List<A> sort(TotOrd<A> ord, List<A> x) {
    ...
  }
}

class IntSmaller implements TotOrd<Integer> {
  public Boolean lessOrEqual(Integer x, Integer y) {
    return x <= y;
  }
  public static IntSmaller it = new IntSmaller();
}

class Test {
  public static void main (String[] args) {
    Sort.sort(IntSmaller.it, Arrays.asList(3,5,4));
  }
}
\end{lstlisting}

\paragraph{C++}
In C++, we can use templates to implement polymorphism.
C++ also uses angular brackets, and the parameter types of classes and functions must be declared using the \lstinline|template| keyword.

\begin{lstlisting}
using namespace std;
#include <list>

template <class A>
class TotOrd {
  bool lessOrEqual(A x, A y);
};

class IntSmaller: public TotOrd<int> {
  bool lessOrEqual(int x, int y) {return x <= y;}
};
IntSmaller* is = new IntSmaller();
  
template <class A>
list<A> sort(TotOrd<A> ord, list<A> x) {
  ...
};

int test() {
    sort<int>(*is, {3,5,4});  
}
\end{lstlisting}

\paragraph{SML}
In SML, we do not have abstract classes, but we can use a datatype instead.
The type parameters of polymorphic types and functions are not declared explicitly.
Instead, they are implicit given as variables like \lstinline|'a|.

\begin{lstlisting}
datatype 'a TotOrder = TotOrder of 'a * 'a -> bool
fun lessOrEqual(ord: 'a TotOrder): 'a * 'a -> bool = case ord of TotOrder(f) => f

val IntSmaller: int TotOrder = TotOrder(fn (x,y) => x <= y)

fun sort(ord: 'a TotOrder, x: 'a list) = ...

fun test() = sort(IntSmaller, [3,5,4])
\end{lstlisting}

\chapter{Design Goals}\label{sec:ad:goals}
  \input{goals}

\chapter{Arithmetic Examples}\label{sec:ad:arithex}
  \section{Exponentiation}\label{sec:ad:exp}

\subsection{Specification}\label{sec:ad:exp:spec}

The function $power(x\in\Z,n\in\N)\in\N$ (also written as $x^n$) returns the $n$-the power of $x$ defined by
\[x^0=1\]
\[x^{n}=x\cdot x^{n-1} \tb\mif n > 0\]

By induction on $n$, we show this indeed specifies a unique function.

\subsection{Naive Algorithm}\label{sec:ad:exp:naive}

It is straightforward to give an algorithm for exponentiation.
For example,
\begin{acode}
\afun[\N]{power}{x:\Z,n:\N}{
  \aifelse{n==0}{1}{x\cdot power(x,n-1)}
}
\end{acode}

\paragraph{Correctness}
The correctness of this algorithm is immediate because it follows the specification literally.
For example, $T(x,n)=n$ is already a termination ordering.

\paragraph{Complexity}
Assuming that all multiplications take $O(1)$ no matter how big $x$ is, the complexity of this algorithm is $\Theta(n)$ because we need $n$ multiplications and recursive calls.

\subsection{Square-and-Multiply Algorithm}\label{sec:ad:exp:sqmult}

It is easy to think that $\Theta(n)$ is also the complexity of the specification, i.e., that there is no sub-linear algorithm for it.
But that is not true.

Consider the square-and-multiply algorithm:
\begin{acode}
\afun[\N]{sqmult}{x:\Z,n:\N}{
  \aifelse{n==0}{1}{
	  r := sqmult(x, n\divop 2)\\
	  sq := r\cdot r\\
	  \aifelseI{n \modop 2==0}{sq}{x\cdot sq}
  }
}
\end{acode}

\paragraph{Correctness}
To prove the correctness of this algorithm, we note that
\[x^{2i+0}=(x^i)^2\]
\[x^{2i+1}=x\cdot(x^i)^2\]
Moreover, we know that $n=2(n\divop 2)+(n\modop 2)$.
Partial correctness of $sqmult$ follows immediately.

To prove termination, we observe that $T(x,n)=n$ is a termination ordering: $n\divop 2$ always decreases (because $n\neq 0$) and remains positive.

\paragraph{Complexity}
Computing the run time of a recursive function often leads to a recurrence relation: The function occurs on both sides with different arguments.
In this case, we get:
 \[C(n)=C(n\divop 2) + c\]
where $c\in O(1)$ is the constant-time effort needed in each iteration.
We systematically expand this further
 \[C(n)=C(n\divop 2) + c = C(n\divop 2\divop 2) + 2\cdot c=\ldots=C(n\overbrace{\divop 2\ldots\divop 2}^{k+1\,\text{times}})+(k+1)\cdot c\]

Now let $n=(b_k\,\ldots\,b_0)_2$ be the binary representation of the exponent.
We know that $k=\lfloor\log_2 n\rfloor$ and $n\overbrace{\divop 2\ldots\divop 2}^{k+1\,\text{times}}=0$.
Moreover, we know from the base case that $C(0)=1$.

Substituting these above yield
\[C(n)\in O(1)+\Theta(\log_2 n)\cdot O(1)=\Theta(\log_2 n)\]

Thus, we can compute $power$ in logarithmic time.

\section{Fibonacci Numbers}\label{sec:ad:fib}

\subsection{Specification}\label{sec:ad:fib:spec}

The Fibonacci numbers $Fib(n\in\N)\in\N$ are defined by
\[fib(0)=0\]
\[fib(1)=1\]
\[fib(n)=fib(n-1)+fib(n-2) \tb\mif n>1\]

By induction on $n$, we prove that this indeed specifies a unique function.

Moreover, we can prove the non-obvious result that
 \[fib(n)=\frac{\phi^n-(1-\phi)^n}{\sqrt{5}} \tb\mfor \phi=\frac{1+\sqrt{5}}{2}\]
($\phi$ is also called the golden ratio.)
That can be further simplified to
 \[fib(n)=round\left(\frac{\phi^n}{\sqrt{5}}\right)\]
where we round to the nearest integer.

\subsection{Naive Algorithm}\label{sec:ad:fib:naive}

It is straightforward to give an algorithm for computing Fibonacci numbers.
For example:
\begin{acode}
\afun[\N]{fib}{n:\N}{
  \aifelse{n\leq 1}{n}{fib(n-1)+fib(n-2)}
}
\end{acode}

\paragraph{Correctness}
The correctness of this algorithm is immediate because it follows the specification literally.
For example, $T(n)=n$ is a termination ordering.

\paragraph{Complexity}
We obtain the recurrence relation $C(n)=C(n-1)+C(n-2)+c$ where $c\in O(1)$ is the constant-time effort of the recursion.
That is the same recurrence as for the definition of the Fibonacci numbers themselves, thus $C(n)\in O(fib(n))=\Exp$.

This naive approach is exponential because every functions spawns $2$ further calls.
Each time $n$ is reduced only by $1$ or $2$, so we have to double the number of calls about $n$ times to $\Theta(2^n)$ calls.

\subsection{Linear Algorithm}\label{sec:ad:fib:linear}

It is straightforward to improve on the naive algorithm turning an exponential into a liner solution.
For example:
\begin{acode}
\afun[\N]{fib}{n:\N}{
  \aifelse{n\leq 1}{n}{
    prev := 0\\
    current := 1 \\
    i = 1 \\
    \awhile{i<n}{
      next := current + prev \\
      prev := current \\
      current := next\\
      i := i+1
    }\\
    \areturn{current}
   }
}
\end{acode}

\paragraph{Correctness}
As a loop invariant, we can use
\[F(n,prev,current,i) = prev==fib(i-1)\wedge current==fib(i)\]
which is straightforward to verify.
After the loop, we have $i==n$ and thus $current=fib(n)$, which yields partial correctness.

As a termination ordering, we can use $T(n,prev,current,i)=n-i$.
Again this is straightforward to verify.

\paragraph{Complexity}
Both the code before and inside the loop take $O(1)$, and the loop is repeated $n-1$ times.
Thus, the complexity is $O(n)$.

\subsection{Inexact Algorithm}\label{sec:ad:fib:inexact}

It is tempting to compute $fib(n)$ directly using $fib(n)=round(\phi^n/\sqrt{5})$.
Because we can precompute $1/\sqrt{5}$, that requires $n+1$ floating point multiplications, i.e., also $O(n)$.

However, it is next to impossible verify the correctness of the algorithm.
While termination is trivial, partial correctness does not hold.
We know that the formula $fib(n)=round(\phi^n/\sqrt{5})$ is true, but that has no immediate use for floating point arithmetic.
Rounding errors will accumulate over time and may eventually lead to a false result.

\subsection{Sublinear Algorithm}\label{sec:ad:fib:sublinear}

Maybe surprisingly, we can still do better.
Inspecting the body of the while loop in the linear algorithm, we see that we can rewrite the assignments as
\[(current,prev):=(current+prev, current)\]
which we can write in matrix form as
\[(current,prev):=(current,prev)\cdot\begin{pmatrix}1&1\\1&0\end{pmatrix}\]

Thus, we obtain
\[(fib(n),fib(n-1))= (1,0)\cdot\begin{pmatrix}1&1\\1&0\end{pmatrix}^n \tb\mfor n>0\]

We can now pick any algorithm for computing the $n$-power of a matrix, e.g., by using square-and-multiply from Sect.~\ref{sec:ad:exp:sqmult} for matrices.

\paragraph{Correctness}
Correctness follows from the correctness of square-and-multiply.

\paragraph{Complexity}
Square-and-multiply has complexity $O(\log n)$.
Thus, we can compute $fib(n)$ with logarithmic complexity.

%\section{Multiplication}
% Karatsuba


\section{Matrices}\label{sec:ad:matrix}

\subsection{Specification}

We write $\Z^{mn}$ for the set $(\Z^n)^m$ of vectors over vectors (i.e., matrices) over integers.

We define two operations on matrices:
\begin{compactitem}
\item Addition: For of $x,y\in \Z^{mn}$, we define $x+y\in\Z^{mn}$ by
\[(x+y)_{ij}=x_{ij}+y_{ij}\]

\item Multiplication: For $x\in \Z^{lm}$ and $y\in \Z^{mn}$, we define $x\cdot y\in\Z^{ln}$ by
\[(x\cdot y)_{ij}=x_{i1}\cdot y_{1j} +\ldots +x_{im}\cdot y_{mj}\]
\end{compactitem}


\subsection{Naive Algorithms}

Vectors and matrices are best stored using arrays.
We assume that
\begin{compactitem}
 \item $Mat$ is the data structure of arrays of arrays of the same length of integers,
 \item if $x$ is an object of $Mat$, then $x.rows$ is the length of the array and $x.columns$ is the length of the inner arrays,
 \item $\anew{Mat}{m,n}$ produces a new array of length $m$ of arrays of length $n$ in which all fields are initialized as $0$.
\end{compactitem}

Then we have the straightforward algorithms
\begin{acode}
\afun[Mat]{add}{x:Mat,y:Mat}{
  r = \anew{Mat}{x.rows,x.columns}\\
  \afor{i}{1}{x.rows}{
    \afor{j}{1}{x.columns}{
      r.i.j := x.i.j+y.i.j
     }
  }\\
  \areturn{r}
} \\
\\
\afun[Mat]{mult}{x:Mat,y:Mat}{
  r = \anew{Mat}{x.rows,y.columns}\\
  \afor{i}{1}{x.rows}{
    \afor{j}{1}{y.columns}{
      \afor{k}{1}{x.columns}{
        r.i.j := r.i.j + x.i.k\cdot y.k.j
      }
    }
  }\\
  \areturn{r}
}
\end{acode}

\paragraph{Correctness}
The algorithms directly implement the definitions.
Thus, correctness---seemingly---obvious.

But there is one subtlety: The functions take two arbitrary matrices---there is no way to force the user to pass matrices of the correct dimensions.
Therefore, we have to state correctness a bit more carefully:
\begin{compactitem}
 \item for $z:=add(x,y)$
   \begin{compactitem}
     \item[precondition:] $x.rows==y.rows$ and $x.columns==y.columns$,
     \item[postcondition:] $z==x+y$ and $z.rows==x.rows$ and $z.columns==x.columns$.
   \end{compactitem}
 \item for $z:=mult(x,y)$
   \begin{compactitem}
     \item[precondition:]  $x.columns==y.rows$
     \item[postcondition:] $z:=mult(x,y)$ is $x\cdot y$ and $z.rows==x.rows$ and $z.columns==y.columns$
   \end{compactitem}
\end{compactitem}
Then we can easily show that $add$ and $mult$ are correct in the sense that the precondition implies the postcondition.

\paragraph{Complexity}
Assuming that all additions and multiplications take constant time, the complexity is easy to analyze.
For addition it is $\Theta(mn)$ and for multiplication $\Theta(lmn)$ where $l$, $m$, and $n$ are the dimensions of the respective matrices.

For addition, we can immediately see that we cannot improve on $\Theta(mn)$: Just creating the new array and returning it already takes $\Theta(mn)$ steps.
Thus, $\Theta(mn)$ is the complexity of the specification, and the naive algorithm is optimal.

This is not obvious for multiplication.
Using the same argument, we can say that the complexity of multiplication is $\Omega(ln)$.
But there cannot be an $\Theta(ln)$-algorithm because $m$ must matter---if $m$ increases, it must take longer.

\subsection{Strassen's Multiplication Algorithm}\label{sec:ad:matrix:strassen}

Inspecting the definition of matrix multiplication, we see that we can split up matrices into rectangular areas of submatrices, for example, like so:
\[\begin{pmatrix}x_{11} & x_{12} & x_{13} & x_{14} \\ x_{21} & x_{22} & x_{23} & x_{24} \\ x_{31} & x_{32} & x_{33} & x_{34} \\ x_{41} & x_{42} & x_{43} & x_{44}\end{pmatrix}
= \begin{pmatrix}
    \begin{pmatrix}x_{11} & x_{12}\\ x_{21} & x_{22}\end{pmatrix} & \begin{pmatrix} x_{13} & x_{14} \\ x_{23} & x_{24} \end{pmatrix} \\
    \begin{pmatrix}x_{31} & x_{32}\\ x_{41} & x_{42}\end{pmatrix} & \begin{pmatrix} x_{33} & x_{34} \\ x_{43} & x_{44} \end{pmatrix}
  \end{pmatrix}
\]
Moreover, if matrices are split up like that, we can still obtain their product in the same way using recursive matrix multiplication:
\[\begin{pmatrix} a & b \\ c & d\end{pmatrix}\cdot \begin{pmatrix} e & f \\ g & h\end{pmatrix}=
  \begin{pmatrix} ae+bg & af+bh \\ ce+dg & cf+dh\end{pmatrix}=\begin{pmatrix} p & q \\ r & s\end{pmatrix}\]

Strassen's algorithm works in the general.
But for simplicity, we only consider the case $l=m=n$, i.e., we are multiplying square matrices.
Then the naive algorithm has complexity $\Theta(n^3)$, and we know the specification has complexity $\Omega(n^2)$.
The question is to find a solution in between.

We further simplify to $n=2^k$, i.e., we can recursively subdivide our $2^k$-matrices to $4$ $2^{k-1}$-matrices.
Then we can design a recursive algorithm that only needs $k$ nested recursions.

The complexity depends on the details of the implementation.
Naively, computing $p,q,r,s$ requires $8$ recursive calls to multiplications and $4$ additions of $2^{k-1}$-matrices.
That yields
 \[C(n)=8\cdot C(n/2) + \Theta(n^2) = \ldots = 8^k\cdot C(1)+\Theta(n^2)\]
Because $k=\log_2 n$ and $C(1)\in O(1)$, that yields $C(n)\in\Theta(n^{\log_2 8})=\Theta(n^3)$.

However, Strassen observed that we can do better.
With some fiddling around, we can replace the $8$ multiplications and $4$ additions with $7$ multiplications and $18$ additions:
\[M_1 = a(f-h)\]
\[M_2 = (a+b)h\]
\[M_3 = (c+d)e\]
\[M_4 = d(g-e)\]
\[M_5 = (a+d)(e+h)\]
\[M_6 = (b-d)(g+h)\]
\[M_7 = (a-c)(e+f)\]
\[\begin{pmatrix} a & b \\ c & d\end{pmatrix}\cdot \begin{pmatrix} e & f \\ g & h\end{pmatrix}=
  \begin{pmatrix} M_5 + M_4 + M_2 + M_6 & M_1 + M_2 \\ M_3 + M_4 & M_1 + M_5 - M_3 - M_7 \end{pmatrix}\]

The extra additions do not harm because they are $\Theta(n^2)$.
But turning the $8$ into a $7$ yields $C(n)=\Theta(n^{\log_2 7})$.
Thus, Strassen's algorithm reduces $n^3$ to $n^{2.81\ldots}$, which can yield practically relevant improvements for relatively small $n$, e.g., $n\approx 30$.
\medskip

Even more efficient algorithms are found regularly.
The current record is $\Theta(n^{2.37\ldots})$.
However, the sufficiently large $n$ for which these are actually faster than Strassen's algorithm is so large that they have no practical relevance at the moment.

% multiplication: polynomials, n-digit numbers (optimal solution unknown)

  
\chapter{Example: Lists and Sorting}\label{sec:ad:sort}
  \section{Specification}\label{sec:ad:listsort:spec}

Lists are the most important non-primitive data structure in computer science.

\subsection{Lists}\label{sec:ad:list:spec}

For a set $A$, the set $A^*$ contains all lists $[a_0,\ldots,a_{l-1}]$ with elements $a_i\in A$ for some $l\in\N$.
$l$ is called the length of the list.

Because $A^*$ is a set for an arbitrary set $A$, data structures for lists must be polymorphic with a type parameter $A$.

\paragraph{Immutable Lists}
The following table specifies the most important functions involving lists:

\begin{ctabular}{|l|l|l|}
\hline
function & returns & abbreviation\\
\hline
$nil[A]\in A^*$ & $[]$ & \\
$range(m\in\N,n\in\N)\in\N^*$ & $[m,\ldots,n-1]$ or $[]$ if $m\geq n$ & \\
\hline
\multicolumn{3}{|c|}{below, let $l\in A^*$ be of the form $[a_0,\ldots,a_{l-1}]$ and assume $n<l$} \\
$length[A](x\in A^*)\in \N$ & $l$ & \\
$get[A](x\in A^*, n\in\N)\in A^*$ & $a_n$ & $x_n$ or $x[n]$\\
$prepend[A](a\in A, x\in A^*)\in A^*$ & $[a,a_0,\ldots,a_{l-1}]$ &\\
$append[A](x\in A^*, a\in A)\in A^*$ & $[a_0,\ldots,a_{l-1},a]$ &\\
$revert[A](x\in A^*)\in A^*$ & $[a_{l-1},\ldots,a_0]$ & \\
$concat[A](x\in A^*, y\in A^*)\in A^*$ & $[a_0,\ldots,a_{l-1},b_0,\ldots,b_{k-1}]$ if $y=[b_0,\ldots,b_{k-1}]$ &  $x+y$\\
$map[A,B](x\in A^*, f\in A\to B)\in B^*$ & $[f(a_0),\ldots,f(a_{l-1})]$ & $l\;map\;f$\\
$fold[A,B](x\in A^*, b\in B, f\in A\times B\to B)\in B$ & $f(a_1,f(a_2,\ldots,f(a_n,b))\ldots)$ & \\ 
\hline
$delete[A](x\in A^*, n\in\N)\in A^*$ & $[a_0,\ldots,a_{n-1},a_{n+1},\ldots,a_{l-1}]$ & \\
$insert[A](x\in A^*, a\in A, n\in\N)\in A^*$ & $[a_0,\ldots,a_{n-1},a,a_n,a_{n+1},\ldots,a_{l-1}]$ & \\
$update[A](x\in A^*, a\in A, n\in\N)\in A^*$ & $[a_0,\ldots,a_{n-1},a,a_{n+1},\ldots,a_{l-1}]$ & \\ % $insert(delete(l,n),a,n)
\hline
\end{ctabular}

Most of them are polymorphic.
$map$ and $fold$ even take a second type parameter for the return type of the function.

These are split into three groups:
\begin{compactitem}
\item The first group contains functions to create new lists. These are important to have any lists.
\item The second group contains functions that take a list $l\in A^*$ as their first argument and return data about $l$ or use $l$ to build new data.
\item The third group also takes a list $l\in A^*$ but also returns an element of $A^*$.
 This distinction is irrelevant in mathematics but critical in computer science: These functions may be implemented using in-place-updates.
 With in-place update, the list $l$ is changed to become the intended result. The original value of $l$ is lost in the process.
 If this is the case, we speak of \textbf{mutable} lists.
\end{compactitem}

\paragraph{Mutable Lists}
The following table specifies the most important functions on mutable lists that differ from immutable lists.
Instead of returning a new list, they have the effect of assigning a new value to the first argument.

\begin{ctabular}{|l|l|l|l|}
\hline
function & returns & effect & abbreviation\\
\hline
\multicolumn{4}{|c|}{below, let $l\in A^*$ be of the form $[a_0,\ldots,a_{l-1}]$ and assume $n<l$} \\
$delete[A](x\in A^*, n\in\N)$ & nothing & $x:=[a_0,\ldots,a_{n-1},a_{n+1},\ldots,a_{l-1}]$ & \\
$insert[A](x\in A^*, a\in A, n\in\N)$ & nothing & $x:=[a_0,\ldots,a_{n-1},a,a_n,a_{n+1},\ldots,a_{l-1}]$ & \\
$update[A](x\in A^*, a\in A, n\in\N)$ & nothing & $x:=[a_0,\ldots,a_{n-1},a,a_{n+1},\ldots,a_{l-1}]$ & $x_n := a$ or $x[n]:= a$\\ % $insert(delete(l,n),a,n)
\hline
\end{ctabular}

The other functions such as $length$ and $get$ are not affected.


\subsection{Sorting}\label{sec:ad:sort:spec}

Sorting a list is intuitively straightforward.
We need a function that takes a list and returns a list with the same elements in a different order, namely such that all elements occur according to their size.

\begin{example}
Consider $x=[4,6,5,3,5,0]\in\N^*$.
Then $sort(x)$ must yield $[0,3,4,5,5,6]$.

Here we made the implicit assumption that we want to sort with respect to the $\leq$-order on $\N$.
We could also use the $\geq$-order.
Then $sort(x)$ should return $[6,5,5,4,3,0]$.

Thus, sorting always depends on the chosen order.
\end{example}

\begin{definition}[Sorting]\label{def:ad:sort:spec}
Fix a set $A$ and a total order $\leq$ on $A$.

A list $x=[a_0,\ldots,a_l]\in A^*$ is called $\leq$-\textbf{sorted} if $a_0\leq a_1 \leq \ldots \leq a_{l-1}\leq a_l$.

Let $count(x\in A^*,a\in A)\in\N$ be the number of times that $a$ occurs in $x$.
Two list $x,y\in A^*$ are a \textbf{permutation} of each other if $count(x,a)=count(y,a)$ for all $a\in A$.

$sort:A^*\to A^*$ is called a $\leq$-\textbf{sorting} function if for all $x\in A^*$, the list $sort(x)$ is a $\leq$-sorted permutation of $x$.
\end{definition}

As usual we check that the specification indeed defines a function:

\begin{theorem}[Uniqueness]
The function $sort$ from Def.~\ref{def:ad:sort:spec} exists uniquely.
\end{theorem}
\begin{proof}
Because $\leq$ is assumed to be total, every list $x$ has a unique least element, which must occur first in $sort(x)$.
By induction on the length of $x$, we show that all elements of $sort(x)$ are determined.
\end{proof}

For immutable lists, the above definition is all the specification we need.
For mutable lists, we specify an alternative sorting function that does not create a new list:

\begin{definition}[In-place Sorting]
An effectful function $sort$ that takes an argument $x\in A^*$ and has the side-effect of modifying the value $v$ of $x$ to $v'$ is called an \textbf{in-place} $\leq$-\textbf{sorting} function if $v'=s(v)$ for a $\leq$-sorting function $s$.
\end{definition}

\subsection{Sorting by a Property}\label{sec:ad:sort:stable}

Often we do not have a total order on $A$, and we want to sort according to a certain property.
The property must be given by a function $p:A\to P$ such that we have a total order $\leq$ on $P$.

For example, we may want to sort a list of students by age.
Then $A=Student$, $P=\N$, and $p:(s\in Student)\mapsto age(s)$.

However, there may be ties: A list may contain multiple different elements that agree in the value of $p$.
To break, we require that the order in the original list should be preserved.
Formally:

\begin{definition}[Sorting by Property]\label{def:ad:sort:stable}
Fix sets $A$ and $P$, a function $p:A\to P$, and a total order $\leq$ on $P$.

Given a list $x\in A^*$, we define a total order $\leq^p$ on the elements of $x$ as follows:
 \[x_i \leq^p x_j \tb\miff\tb p(x_i) < p(x_j) \tb \mor \tb p(x_i)=p(x_j) \mand i\leq j\]

$sort:A^*\to A^*$ is called a \textbf{stable sorting} function for $p$ and $\leq$ if it is a sort function for $\leq^p$.
\end{definition}

Note that normal sorting becomes a special case of sorting by property using $P=A$ and $p(a)=a$.

\subsection{Why Do We Care About Sorting?}

Thus, a good, modern programmer might respond as follows:
\begin{compactenum}
\item How do you implement sorting a list? --- I call the $\mathit{sort}$ function of my programming language's basic library.
\item OK, but what if there is no $\mathit{sort}$ function? --- I import a library that provides it.
\item OK, but what if there is no such library? --- I use a different programming language.
\item OK, but what if circumstances beyond your control prevent you from using third-party libraries? --- I copy-paste a definition from the internet.\footnote{Nowadays an internet search for elementary problems almost always finds a solution for every programming language, usually on \url{http://www.stackexchange.org}.}
\end{compactenum}

Thus, for most people the only realistic situations in which to implement sorting algorithms is in exams, job interviews, or similar situations.
Then the question is never actually about sorting---it just uses sorting as an example to see whether the programmer understands how to design algorithms, analyze their complexity, and verify their correctness.

In any case, sorting is an extremely good subject for an introductory computer science class because it
\begin{compactitem}
 \item is an elementary problem that is easy to understand for students,
 \item is complex enough to exhibit many important general principles in interesting ways,
 \item is simple enough for all analysis to be doable manually,
 \item has multiple solutions, none of which is better than all the others,
 \item is extremely well-studied,
 \item is widely taught so that the internet is full of good visualizations that help learners.
\end{compactitem}


%%%%%%%%%%%%%%%%%%%%%%%%%%%%%%%%%%%%%%%%%%%%%%%%%%%%%%%%%%%%%%%%%%%%%%%%
\section{Design: Data Structures for Lists}\label{sec:ad:list:ds}


Besides natural numbers, the most important examples of a data structure are lists.
There are many different data structures for lists that differ subtly in how simply and/or efficiently the various functions can be implemented.
We will write $List[A]$ whenever we mean an arbitrary data structure for lists.

\subsection{Immutable Lists}

For immutable lists, functions like $delete$, $insert$, and $update$ (see Sect.~\ref{sec:math:sets:derivfun}) always return new lists.
That requires copying (parts of) the old list, which takes more time and memory.

Without further qualification, this is usually what $List[A]$ refers to.

\subsubsection{Functional Style: Lists as an Inductive Type}

Functional languages usually implement lists an in inductive data type:
\begin{acode}
\adata{{IndList[A]}}{nil,{cons(head: A,\, tail:IndList[A])}}
\end{acode}
Now the list $[1,2,3]$ is built as $cons(1,cons(2,cons(3,nil)))$.

Then functions on lists are implemented using recursion and pattern-matching.
For example:
\begin{acode}
\afun[{IndList[B]}]{map}{x:IndList[A],f:A\to B}{\amatch{x}{\acase{nil}{nil},\acase{cons(h,t)}{cons(f(h),map(t,f))}}}
\end{acode}

\subsubsection{Object-Oriented Style: Linked Lists}

Every inductive data type can also be systematically realized in an object-oriented language.
The correspondence is as follows:

\begin{ctabular}{|l|l|l|}
\hline
inductive type & class & example: lists\\
\hline
name of the type & abstract class & $IndList$ \\
parameters of the type & parameters of the class & $A$ \\
constructor & concrete subclass & e.g., $cons$\\
constructor arguments & constructor arguments & $head:A,tail:IndList[A]$ \\
\hline
\end{ctabular}

A basic realization looks as follows:
\begin{acode}
\aclassA{{IndList[A]}}{}{}{}\\
\aclass{{nil[A]}}{}{IndList[A]()}{}\\
\aclass{{cons[A]}}{head:A,tail:List[A]}{IndList[A]()}{}
\end{acode}
Now the list $[1,2,3]$ is built as $\anew{cons}{1, \anew{cons}{2, \anew{cons}{3, \anew{nil}{}}}}$.

Instead of pattern-matching, we have to use instance-checking to split cases.
For example:
\begin{acode}
\afun[{IndList[B]}]{map}{x:IndList[A],f:A\to B}{
  \aifelse{\aisinst{x}{nil}}
    {\anew{nil}{}}
    {xc := \aasinst{x}{cons} \\
     \anew{cons}{f(xc.head), map(x.tail,f)}
    }
}
\end{acode}

Moreover, we have to override equality so that, e.g., two instances of $cons$ are equal iff they used equal constructor arguments.

\subsubsection{Complexity}

Complexity of lists is measured in the lenght $n$ of the list.

Most operations on lists are linear because the algorithm must traverse the whole list.
For example, the straightforward implementation of $length$ takes $\Theta(n)$.

Similarly, $get(x,i)$ takes $i$ steps to find the element. This is $n$ in the worst case and $n/2$ on average.
So it also takes $\Theta(n)$.

In general, immutable lists require copying the list, whenever we insert, delete, or update elements.
These algorithms must traverse the list.
Therefore, they usually take $\Theta(n)$ time for the traversal \emph{and} $\Theta(n)$ memory for the result list.

In the case of $map(x,f)$ and $fold(x,a,f)$, the complexity depends on the passed function $f$.
However, in the typical case where the run time of $f$ does not depend on the length of the list, we can assume it takes constant time $c$.
Thus, the overall run time is $\Theta(cn)=\Theta(n)$.

However, there is one important exception: $prepend$ takes $\Theta(1)$.
This is because we can implement $prepend(a,x)$ simply by calling $cons(a,x)$.
Correspondingly, removing the first element takes $\Theta(1)$.

\subsection{Mutable Lists}

Mutable lists allow assignments to the individual elements of the list.
This allows updating an element without copying the list, thus allowing for many operations with $\Theta(1)$ time or memory complexity.

Because we can update the list in place, it becomes critical for efficiency how exactly the list is stored in memory.
Three cases are of great importance, all with advantages and disadvantages:

\begin{ctabular}{|l|l|p{6cm}|}
\hline
data structure & memory layout & remark \\
\hline
array & all in a row & easy to find elements but difficult to insert/delete \\
(singly-)linked list & every element points to next one & easy to insert/delete but traversal needed \\
doubly-linked list &  every element points to next and previous one & traversal in both directions possible, more overhead\\
growable array & linked list of arrays & compromise between the above \\
\hline
\end{ctabular}

\subsubsection{Arrays}

The data structure $Array[A]$ stores all elements in a row in memory.
Arrays must be a primitive feature of the programming language and are so in most languages.

For example, the list $x=[1,2,5]$ is stored in $3$ consecutive memory locations:
\begin{amemory}
\avar{x}{\N^*}{P}
\alocations
\aloc{P}{1}
\aloc{P+1}{3}
\aloc{P+2}{5}
\end{amemory}

That allows implementing $get$ and $update$ in $\Theta(1)$.
$get(x,n)$ is evaluated by retrieving the element in memory location $P+n$.
That takes one step to retrieve $x$, one step for the addition, and one step to retrieve the element at $P+n$.
$update(x,a,n)$ works accordingly.

Inserting and deleting elements still takes $\Theta(n)$.
For example, we can implement deleting by:
\begin{acode}
\afun{delete}{x:Array[A],n:\N}{
  \afor{i}{n}{length(x)-1}{
    x[i] := x[i+1]
  }
}
\end{acode}

Inserting an element into an array is difficult though: The memory location behind the array may not be available because it was already used for something else.
Therefore, arrays are often realized in such a way that the programmer chooses in advance the maximal length of the array.
Thus, technically this data structure does not realize the set $A^*$ but the set $A^n$ for some length $n$.
This may waste memory if $n$ is chosen too large.
But arrays are unbeatable in the common situation where we know that we will never call $insert$ anyway.

\subsubsection{Linked Lists}

Mutable linked list consist of a reference to the first element.
Each element consists of a value and a reference to its successor.
We can implement that using classes (or similar primitives like structs in C):
\begin{acode}
\aclass{{LinkedList[A]}}{head:Elem[A]}{}{}\\
\aclass{{Elem[A]}}{value:A, next:Elem[A]}{}{}
\end{acode}

Technically, $head$ and $next$ should have the type $Elem(A)^?$ to allow for empty lists and the end of the list, respectively.
However, object-oriented programmers usually use a trick where the built-in value $null$ is used:
\begin{compactitem}
 \item If $head$ is null, we have the empty list.
 \item If $next$ is null, we have the last element of the list.
\end{compactitem} those cases.

Now the list $[1,2,5]$ is built as $x:=\anew{LinkedList}{\anew{Elem}{1, \anew{Elem}{2, \anew{Elem}{5, null}}}}$.
It is stored in memory as
\begin{amemory}
\avar{x}{\N^*}{P}
\alocations
\aloc{P.head}{Q}
\hline
\aloc{Q.value}{1}
\aloc{Q.next}{R}
\hline
\aloc{R.value}{2}
\aloc{R.next}{S}
\hline
\aloc{S.value}{5}
\aloc{S.next}{null}
\end{amemory}

Deletion can now be realized in-place as follows
\begin{acode}
\afun{delete}{x:LinkedList[A],n:\N}{
 \aifelse{n==0}{x.head := x.head.next}{
  previous := x.head \\
  current := x.head.next\\
  \afor{i}{1}{n-1}{
    previous := current\\
    current := current.next\\
  }
  previous.next := current.next
 }
}
\end{acode}

Like immutable lists, linked lists take $\Theta(n)$ time for most operations.
However, they still perform better because changes can be done in-place.
Moreover, many operations can be done in $\Theta(1)$ memory whereas immutable lists often require $\Theta(n)$ memory.

An interesting exception is the following variant of $insert$:
Instead of taking the position $n$ at which to insert (which takes linear time to find), it take the element after which to insert:
\begin{acode}
\afun{insert}{x:LinkedList[A],after:Elem[A],a:A}{
 after.next := \anew{Elem}{a, after.next}
}
\end{acode}

A similar trick for deleting does not work so well: We can implement $delete(x:LinkedList[A], after:Elem[A])$ in $\Theta(1)$ if we know after which element to delete.
But a function $delete(x:LinkedList[A], e:Elem[A])$ where $e$ is to be deleted would be useful to delete---that still requires $\Theta(n)$ to find $e$ in the linked list.

\subsubsection{Doubly-Linked Lists}

Doubly-linked linked list are the same as linked lists except that each element also knows its predecessor ($null$ for the first element).
Moreover, the list knows its first and last element.

\begin{acode}
\aclass{{DoubleLinkedList[A]}}{head:Elem[A], last:Elem[A]}{}{}\\
\aclass{{Elem[A]}}{value:A, previous: Elem[A], next:Elem[A]}{}{}
\end{acode}

Now the list $x=[1,2,5]$ is stored in memory as
\begin{amemory}
\avar{x}{\N^*}{P}
\alocations
\aloc{P.head}{Q}
\aloc{P.last}{S}
\hline
\aloc{Q.value}{1}
\aloc{Q.previous}{null}
\aloc{Q.next}{R}
\hline
\aloc{R.value}{2}
\aloc{R.previous}{Q}
\aloc{R.next}{S}
\hline
\aloc{S.value}{5}
\aloc{S.previous}{R}
\aloc{S.next}{null}
\end{amemory}

Operations on doubly-linked lists are usually in the same complexity class as the corresponding ones for singly-linked lists.

A doubly-linked list has more memory overhead and thus copying and update operations have more time overhead.
But doubly-linked lists can be traversed efficiently in \emph{both} directions.
For example, processing the elements of a singly-linked list in reverse order requires two traversals: one to find the find element, one to process.
The same operation on a doubly-linked list requires only one traversal.
Both are $\Theta(n)$ though, but the latter may be twice as fast.

In a double-linked list, we can also define nice constant-time variants for both $insert$ and $delete$.
For example:
\begin{acode}
\afun{delete}{x:DoubleLinkedList[A],e:Elem[A]}{
 \aifelse{e.previous == null}{x.head := e.next}{e.previous.next := e.next} \\
 \aifelse{e.next == null}{x.last := e.previous}{e.next.previous := e.previous}
}
\end{acode}

The following table summarizes the complexity of some operations on arrays, linked lists and doubly-linked lists in terms of the length $l$: 
\begin{ctabular}{|l|l|l|l|l|l|l|l|l|} 
\hline
\multirow{2}{*}{}  & \multirow{2}{*}{$length[A]$} & $get[A]$ & $update[A]$ & $insert[A]$ & $delete[A]$ & \multirow{2}{*}{$prepend[A]$} & \multirow{2}{*}{$append[A]$} & \multirow{2}{*}{$reverse[A]$} \\ \cline{3-6}
&       & \multicolumn{4}{c|}{at position $n$}  &         &        &         \\ \hline
Array & $\Theta(1)$ & \multicolumn{2}{c|}{$\Theta(1)$} & \multicolumn{2}{c|}{$\Theta(l-n)$} & $\Theta(l)$ & $\Theta(1)$ & $\Theta(l)$ \\
Linked list & $\Theta(l)$ & \multicolumn{2}{c|}{$\Theta(n)$} & \multicolumn{2}{c|}{$\Theta(n)$} & $\Theta(1)$ & $\Theta(l)$ & $\Theta(l)$ \\
Doubly-linked List & $\Theta(l)$ & \multicolumn{2}{c|}{$\Theta(n)$} & \multicolumn{2}{c|}{$\Theta(n)$} & $\Theta(1)$ & $\Theta(1)$ & $\Theta(l)$  \\ \hline
\end{ctabular}


\subsubsection{Growable Arrays}

Growable arrays are a compromise between arrays and linked lists.
Initially, they behave like an array with a fixed length $l$.
However, when inserting an element that increases the length beyond $l$, we create a second array of length $l$ (elsewhere in memory) and connect the two.
Thus, a growable array is a linked list of fixed-length arrays.
The choice of $l$ is up to the data structure designer, who may allow the programmer to tweak it.

Retrieval and update technically are linear now.
To access the element in position $n$, we have to make $n/l$ retrievals to jump to the needed array.
Because $l$ is constant, that yields $\Theta(n)$ retrievals.
However, $l$ is usually large so that element access is only a little slower than for an array and much faster than for a linked list.

%%%%%%%%%%%%%%%%%%%%%%%%%%%%%%%%%%%%%%%%%%%%%%%%%%%%%%%%%%%%%%%%%%%%%%%%
\section{Design: Algorithms for Sorting}\label{sec:ad:sort:algo}

We assume a fixed set $A$ and a fixed comparison function $\leq:A\times A \to \B$.
For $x\in A^*$, we write $Sorted(x)$ if $x$ is $\leq$-sorted.

\paragraph{Auxiliary Functions}
Many in-place sorting algorithms have to swap two elements in a mutable list at some point.
Therefore, we define an auxiliary function

\begin{acode}
\afun{swap}{x:MutableList[A], i:\N, j:\N}{
  h := x[i]\\
  x[i] := x[j]\\
  x[j] := h
}
\end{acode}
Here $MutableList$ is any of the mutable data structures from above.

It is easy to see that this function indeed has the effect of swapping two elements in $x$.
For arrays, the time complexity of $swap$ is $\Theta(1)$.
For linked lists, it is $\Theta(n)$.


\subsection{Bubblesort}\label{sec:ad:sort:bubble}

Bubblesort is a stable in-place sorting algorithm that closely follows the natural way how a human would sort.
The idea is to find two elements that are not in order and swap them.
If no such elements exist, the list is sorted.

\begin{acode}
\afun{bubblesort}{x:Array[A]}{
 sorted := \false \\
 \awhile{!sorted}{
   sorted:=\true \\
   \afor{i}{0}{length(x)-2}{
     \aif{! x[i]\leq x[i+1]}{
       sorted := \false \\
       swap(x,i,i+1)
     }
   }
 }
}
\end{acode}

\paragraph{Correctness}
The for-loop compares all $length(x)-1$ pairs of neighboring elements.
It sets $sorted$ to $\false$ if the list is not sorted.
Thus, we obtain the loop invariant $F(x,sorted)=sorted==Sorted(x)$, which immediately yields partial correctness.

Total correctness follows from the termination ordering
 \[T(x,sorted)=\text{number of pairs $i,j$ such that $! x_i\leq x_j$} + \cas{1\mifc sorted==\false\\ 0\mifc sorted==\true}\]
Indeed, this number decreases in every iteration of the loop in which $x$ is not sorted.
The second summand is necessary to make $T(x,sorted)$ also decreases if $x$ is already sorted (which happens exactly one in the last iteration).

\paragraph{Complexity}
If $n$ is the length of $x$, each iteration of the while-loop has complexity $\Theta(n)$.
Moreover, the while-loop iterates at most $n$ times.
That happens in the worst-case: when $x$ is reversely sorted initially.
Thus, the complexity is $\Theta(n^2)$.

In the best-case, when $x$ is already sorted initially, the complexity is $\Theta(n)$.
That is already optimal because it requires $n-1$ comparisons to determine that a list is sorted.

\subsection{Insertionsort}\label{sec:ad:sort:insertion}

Insertion is also a stable in-place algorithm.

The idea is to sort increasingly large prefixes of a list $x$.
If $[x_0,\ldots,x_{i-1}]$ is sorted already, the element $x_i$ is inserted among them.

\begin{acode}
\afun{insertionsort}{x:Array[A]}{
  \afor{i}{1}{length(x)-1}{
    current := x[i] \\
    pos := i \\
    \awhile[shift elements to the right to make space for $current$]{pos > 0 \aand current < x[pos-1]}{
       x[pos] := x[pos-1] \\
       pos := pos - 1
    }\\
    x[pos] := current
  }
}
\end{acode}

\paragraph{Correctness}
We use a loop-invariant for the for-loop: $F(x,i)=Sorted([x_0,\ldots,x_{i-1}])$.
The preservation of the loop-invariant is non-obvious but straightforward to verify.
It holds initially because the empty list is trivially sorted.
That yields partial correctness.

Termination is easy to show using the termination ordering $T(x,i,current,pos)=pos$ for the while-loop.

\paragraph{Complexity}
If $n$ is the length of $x$, the for-loop runs $n$ times with $i=0,\ldots,n-1$
Inside, the while-loop runs $i$ times in the worst-case: if $x$ is reversely sorted, all $i$ elements before $current$ must be shifted to the right.
That sums up to $0+1+\ldots+n-1\in \Theta(n^2)$.

Everything else is $O(n)$.
Thus, the worst-case complexity is $\Theta(n^2)$.

In the best-case, if $x$ is already sorted, the while-loop never runs, and the complexity is $\Theta(n)$.

\subsection{Mergesort}\label{sec:ad:sort:merge}

Mergesort is based on the observation that
\begin{compactitem}
  \item sorting smaller lists is much easier than sorting larger lists (because the number of pairs that have to be compared in $\Theta(n^2)$,
  \item merging two sorted lists is easy (linear time).
\end{compactitem}
Thus, we can divide a list into two halves, sort them recursively, then merge the results.
This is similar to the idea of square-and-multiply (Sect.~\ref{sec:ad:exp:sqmult}) and an example of the family of divide-and-conquer algorithms.

Because it needs auxiliary memory to do the merging of two half lists into one, it is easiest to implement as non-in-place algorithm.
Then the input data structure does not matter and can be assumed to be immutable.
The following is a straightforward realization:

\begin{acode}
\afun[{List[A]}]{mergesort}{x:List[A]}{
  n := length(x) \\
  \aifelse{n<2}{x}{
    k := n\divop 2\\
    l := mergesort([x_0,\ldots,x_{k-1}]) \\
    r := mergesort([x_k,\ldots,x_{n-1}]) \\
    \areturn{merge(l,r)}
  }
}\\
\\
\afun[{List[A]}]{merge}{x:List[A], y:List[A]}{
  xRest := x\\
  yRest := y\\
  res = [] \\
  \awhile{nonempty(xRest) \aor nonempty(yRest)}{
    takefromX := empty(yRest) \aor (nonempty(xRest) \aand xRest.head \leq yRest.head)\\
    \aifelse{takefromX}{
      res := cons(xRest.head, res) \\
      xRest := xRest.tail
    }{
      res := cons(yRest.head, res) \\
      yRest := yRest.tail
    }
  }\\
  \areturn{reverse(res)}
}
\end{acode}

\paragraph{Correctness}
Because the function $merge$ is not part of the specification, we have to first specify which property we want to prove about it.
The needed property for $z:=merge(x,y)$ is:
 \begin{compactitem}
   \item precondition: $Sorted(x)$ and $Sorted(y)$
   \item postcondition: $Sorted(z)$ and $z$ is a permutation of $x+y$
 \end{compactitem}

Now we can prove each function correct.
\medskip

First we consider $mergesort$.
Partial correctness means to prove $Sorted(mergesort(x))$.
That is very easy:
\begin{compactitem}
  \item If $n<2$, $x$ is trivially sorted.
  \item Otherwise:
   \begin{compactitem}
     \item $Sorted(a)$ and $Sorted(b)$ follow from the postcondition of the recursive call.
     \item Then the postcondition of $merge$ yields $Sorted(merge(a,b))$.
   \end{compactitem}
\end{compactitem}

Relative termination is immediate (assuming that $merge$ always terminates, which we prove below).
A termination ordering is given by $T(x)=length(x)$.
Indeed, $mergesort$ recurses only into strictly shorted lists.
\medskip

Second we consider $merge$.
We use a loop invariant $F(x,y,xRest,yRest,res)$ that states that
 \begin{compactitem}
  \item $Sorted(reverse(res))$ and $Sorted(xRest)$ and $Sorted(yRest)$
  \item All elements in $res$ are in $\leq$-relation to all elements in $xRest+yRest$.
  \item $res+xRest+yRest$ is a permutation of $x+y$
 \end{compactitem}
It is non-obvious but it is straightforward to see that this indeed a loop invariant:
 \begin{compactitem}
   \item $reverse(res)$ remains sorted because we always take the smallest element in $yRest+xRight$ and prepend it to $res$.
    In particular, because $xRest$ and $yRest$ are sorted, the smallest element must be $xRest.head$ or $yRest.head$.
   \item For the same reason, all elements of $res$ remain smaller than the ones of $xRest$ and $yRest$.
   \item Because we only remove elements from $xRest$ and $yRest$, they remain sorted.
   \item Because every element that is removed from $xRest$ or $yRest$ is immediately added to $res$, they remain a permutation.
 \end{compactitem}

To show partial correctness, we see that
\begin{compactitem}
  \item The loop invariant holds initially, which is obvious.
  \item After completing the loop, $xRest$ and $yRest$ are empty.
  \item Then, using the loop invariant, it is easy to show that $reverse(res)$ is sorted and a permutation of $x+y$.
\end{compactitem}

To show termination, we use $T(x,y,xRest,yRest,res)=length(xRest)+length(yRest)$.
It is easy to see that $T$ is a the termination ordering for the while-loop.

\paragraph{Complexity}
We have to analyze the complexity of both functions.

First we consider $merge$.
Let $n=length(x)+length(y)$.
\begin{compactitem}
 \item The three assignments in the beginning are $O(1)$.
 \item The while-loop is repeated once for every element of $x$ and $y$, which requires $\Theta(n)$ steps.
 The body of the loop takes $O(1)$. So $\Theta(n)$ in total.
 \item The last step requires reverting $res$, which has $n$ elements at this point.
 Reverting a list requires building a new list by traversing the old one. That is $\Theta(n)$ as well.
\end{compactitem}
Thus, the total complexity of $merge$ is $\Theta(n)=\Theta(length(x)+length(y))$.
\medskip

Second we consider $mergesort$.
Let $n=length(x)$.
We compute the time complexity $C(n)$:
\begin{compactitem}
 \item The assignments and the if-statement are in $O(1)$.
 \item The recursive calls to $mergesort$ take $C(n/2)$ each.
 \item The call to $merge$ takes $\Theta(length(a)+length(b))=\Theta(n)$.
\end{compactitem}
That yields
 \[C(n)=2\cdot C(n/2)+\Theta(n) = \ldots = 2^k\cdot C(n/2^k) + k\cdot \Theta(n)\]
 By choosing $k=\log_2 n$ and $C(1)=C(0)\in O(1)$, we obtain
 \[C(n)=n\cdot O(1)+\log_2 n\cdot \Theta(n)=\Theta(n\log_2 n)\]
\medskip

Thus, mergesort is quasilinear and thus strictly more efficient than bubblesort and insertionsort.

Contrary to bubblesort and insertionsort, mergesort takes the same amount of time no matter how sorted the input already is.
The recursion and the merging happen in essentially the same way independent of the input list.
Thus, its best-case complexity is also $\Theta(n\log_2 n)$.

\begin{remark}[Building the list reversely in $merge$]
$merge$ could be simplified by always adding the element $xLeft.head$ or $yLeft.head$ to the \emph{end} of $res$ instead of the beginning.
However, as discussed in Sect.~\ref{sec:ad:list:ds}, adding an element to the beginning of an immutable list takes constant time whereas adding to the end takes linear time.

Therefore, if we added elements to the end of $res$ would become quadratic instead of linear.
Then mergesort as a whole would also be quadratic.
\end{remark}

\subsection{Quicksort}\label{sec:ad:sort:quick}

Quicksort is similar to mergesort in that two sublists are sorted recursively.
The main differences are:
\begin{compactitem}
 \item It does not divide the list $x$ in half.
  Instead if picks some element $a$ from the list (called the \emph{pivot}).
  Then it divides $x$ into sublists $a$ and $b$ containing the elements smaller and greater than $x$ respectively.\\
  No merging is necessary because all elements in $a$ are smaller than all elements in $b$.
  Thus the sorted list is $quicksort(a)+x+quicksort(b)$.
 \item To divide the list, quicksort has to traverse and reorder the list anyway.
 Therefore, it can easily be implemented in-place avoiding the use of auxiliary memory.
\end{compactitem}

When implemented as an in-place sorting algorithm, the recursive call takes two additional arguments: two numbers $first$ and $last$ that describe the sublist that should be sorted.

\begin{remark}[Additional Arguments in a Recursion]
Carrying along auxiliary information is very typical for recursive algorithms.
Therefore, we often find pairs of function:
 \begin{compactitem}
  \item A recursive function that takes additional arguments.\\
   That is $quicksortSublist$ below, which takes the entire list and the information about which sublist to sort.
  \item A non-recursive function that does nothing but call the other function with the initial arguments.\\
   That is $quicksort$ below, which calls $quicksortSublist$ on the entire list (e.g., on the sublist from $0$ to the end of $x$).
 \end{compactitem}
\end{remark}


\begin{acode}
\afun{quicksort}{x:Array[A]}{
  quicksortSublist(x,0,length(x)-1)
}\\
\\
\afun{quicksortSublist}{x:Array[A], first:\N, last: \N}{
  \aifelse{first \geq last}{
    \areturn{}
  }{
    pivot := x[last]\\
    pivotPos := first\\ %%    // place for swapping
    \aloopinv{x[k]\leq pivot \mfor k=first,\ldots,pivotPos-1 \mand pivot \leq x[k] \mfor k=pivotPos,\ldots,j-1}
    \afor{j}{first}{last - 1}{
      \aif{x[j] \leq pivot}{
         swap(x,pivotPos,j) \\
         pivotPos := pivotPos + 1
      }
    }\\
    swap(x,pivotPos,last)\\
    \\
    quicksortSubList(x, first, pivotPos - 1)\\
    quicksortSubList(x, pivotPos + 1, last)
  }
}
\end{acode}

\paragraph{Correctness}
Before proving correctness we have to specify the behavior of the auxiliary function $quicksortSublist$:
\begin{compactitem}
 \item precondition: none
 \item postcondition: $Sorted([x_{first},\ldots,x_{last}])$
\end{compactitem}
Then the correctness of $quicksort$ follows immediately from that of $quicksortSublist$.
\medskip

Now we prove the partial correctness of $quicksortSublist$.
First, the base case is trivially correct: It does nothing for lists of length $0$ or $1$.
For the recursive case, we prove that the following two properties holds just before the two recursive calls:
\begin{compactitem}
 \item The sublist $[x_{first},\ldots,x_{last}]$ is a permutation of its original value, and no other elements of $x$ have changed.
  That is easy to because we only change $x$ by calling $swap$ on positions between $first$ and $last$.
 \item All values $x_k$ are
  \begin{compactitem}
    \item smaller than $pivot$ for $k=first,\ldots,pivotPos-1$,
    \item equal to $pivot$ for $k=pivotPos$,
    \item greater than $pivot$ for $k=pivotPos+1,\ldots,last$.
  \end{compactitem}
  We prove that by using the indicated loop invariant for the for-loop.
  It is trivially true before the for-loop because $first=pivotPos$ and $pivotPos=j$.
  It is straightforward to check that it is preserved by the for-loop.
  Therefore, it holds after the for-loop for the value $j=last-1$.
  The last call to $swap$ moves the pivot element into $x_{pivotPos}$ so that the loop invariant is now also true for $j=last$.
  Then the needed properties can be seen easily.
\end{compactitem}
\medskip

To prove the termination of $quicksortSublist$, we use the termination ordering $T(x,first,last)=last-first+1$ (which is the length of the sublist).
That value always decreases because the pivot element is never part of the recursive call.

\paragraph{Complexity}
Let $n=last-first-1$ be the length of the sublist.
It is easy to see that, apart from the recursion, $quicksortSublist$ takes $\Theta(n)$ steps because the for-loop traverses the sublist.
Thus, the complexity of quicksort depends entirely on the lengths of the sublists in the recursive calls.
However, the pivot position and therefore those lengths are hard to predict.

The best-case complexity arises if the pivot always happens to be in the middle.
Then the same reasoning as for mergesort, yields best-case complexity $\Theta(n\log_2 n)$.
The worst-case arises if the list is already sorted: then the pivot position will always be the last one, and the two sublists have sizes $n-1$ and $0$.
That results in $n$ recursive calls on sublists of length $n$, $n-1$, \ldots, $1$ as well as $n$ calls on empty sublists.
Consequently, the worst-case complexity is $\Theta(n^2)$.

However, the worst-case complexity does not do quicksort justice because it is much higher than its average-case complexity.
Because there are only finitely many permutations for a list of fixed length, the average-case complexity can be worked out systematically.
The result is $\Theta(n\log_2 n)$.
\medskip

It may seem that quicksort is less attractive than mergesort because of its higher worst-case complexity.
However, that is a minor effect because the algorithms have the same best-case and average-case complexity.
Instead, the constant factors, which are rounded away by using $\Theta$-classes, become important to compare two algorithms with such similar complexity.

Here quicksort is superior to mergesort.
Moreover, quicksort can be optimized in many ways.
In particular, the choice of the pivot can be tuned in order to increase the likelihood that the two sublists end up having the same size.
For example, we can randomly pick $3$ elements of the sublist and use the middle-size one as the pivot.
With such optimizations, quicksort can become substantially faster than mergesort.

\subsection{Other Algorithms}

There is a number of other sorting algorithms that we will not go into here.
Examples include counting sort, radix sort, and bucket sort.

One particularly important sorting algorithm is heap sort, which we discuss in Sect.~\ref{sec:ad:heapsort} (after introducing heaps).

\subsection{In Programming Languages}

Most programming languages come with a standard library that includes efficient sorting algorithms.
Moreover, other libraries for other algorithms may be around.
In some cases, languages only specify the interface and leave the implementation (and thus the choice of algorithm )to individual implementations of compilers/interpreters.

The following gives some examples.

Python uses Timsort (named after the programmer), which is a hybrid of mergesort and insertionsort with various optimizations.
It is written directly in C.

Java used to use just quicksort.
Java 7 uses either Timsort (ported to Java) or a variant of quicksort that uses two pivot elements.

Scala defers to Java's implementation.

C++'s std library specification does not prescribe a sorting algorithm but requires $O(n\log_2 n)$ worst-case complexity (average-case in earlier versions).
Implementations vary in their choice of algorithm, e.g., using hybrid algorithms that perform some iterations of quicksort before switching to insertionsort for the resulting small lists.

For Javascript, the choice is up to the browser (because every browser is a separate implementation of Javascript).


% % % % % % % % % % % % % % % % % % % %
\part{Important Data Structures}\label{sec:ad:ds}

\chapter{Finite Data Structures}\label{sec:ad:finiteds}
 \input{finite}

%char, string

\chapter{Number-Based Data-Structures}\label{sec:ad:numbers}
 \input{numbers}

\chapter{Option-Like Data Structures}\label{sec:ad:option}
 \input{option}

\chapter{List-Like Data Structures}\label{sec:ad:lists}
 The specification and several data structures for mutable and immutable lists are already discussed in Sect.~\ref{sec:ad:listsort:spec}.

Here we only discuss some additional data structures for the set $A^*$.

\section{Stacks}\label{sec:ad:stack}

$Stack[A]$ is a data structure for the set $A^*$.

$Stack[A]$ is very similar to $List[A]$.
The difference is that $Stack[A]$ provides \emph{less} functionality.
While $List[A]$ is a general-purpose list, $Stack[A]$ is custom-fitted to one specific, very common use case.
By requiring fewer operations, they allow more optimized implementations.

Stacks can be mutable or (less commonly) immutable.
Here we will use the mutable variant.
The functions for mutable stacks are:

\begin{ctabular}{|l|l|l|}
\hline
function & returns & effect \\
\hline
$push(x\in A^*, a\in A)\in\Unit$ & nothing & prepend $a$ to $x$\\
$pop(x\in A^*)\in A^?$ & the first element of $x$ (if any) & remove the first element of $x$ \\
$top(x\in A^*)\in A^?$ & the first element of $x$ (if any) & none \\
\hline
\end{ctabular}

The intuition behind stacks is that they provide a LIFO store of data.
LIFO means last-in-first-out because every $pop$ returns the most recently pushed value.
This is exactly the behavior of a literal stack of items: We can put an item on top of a stack ($push$), remove an item from the stack ($pop$), or check what item is on top ($top$).
We cannot easily see or remove the other items.

Very often, the LIFO behavior is exactly what is needed.
For example, when we solve a maze, we can push every decision we make.
When we hit a dead end, we trace back our steps---for that, we have to pop the most recent decision, and so on.

\section{Queues}\label{sec:ad:queue}

Queues are very similar to stacks.
Everything about stacks also applies to queues except for the following.

The functions for mutable queues are:

\begin{ctabular}{|l|l|l|}
\hline
function & returns & effect \\
\hline
$enqueue(x\in A^*, a\in A)\in\Unit$ & nothing & append $x$ to $A$\\
$dequeue(x\in A^*)\in A^?$ & the first element of $x$ & remove the first element of $x$ \\
$empty(x\in A^*)\in\Bool$ & true if $x$ is empty & none \\
\hline
\end{ctabular}

The intuition behind queues is that they provide a FIFO store of data.
FIFO means first-in-first-out because every $dequeue$ returns the least recently enqueued value.
This is exactly the behavior of a literal queue of people: Every newcomer has to queue up at the end of the queue ($enqueue$), and every time a server is ready the first in line gets served ($dequeue$).
Newcomers cannot cut in line, and the server cannot easily see who else is waiting.

Very often, the FIFO behavior is exactly what is needed.
For example, when we have a list of tasks that need to be done.
Every time we create a new task, we enqueue it, and whenever we have time we dequeue the next task.

Queues are often used when components exchange messages or commands.
In that case, some components---called the producers---only call enqueue, and other components---called the consumers---only call dequeue.
For example, the producers can be different programs, $A$ is the type of print jobs, and the consumers are different printers.

More complex queue data structures may also for dequeueing based on priority (see also Sect.~\ref{sec:ad:heapqueue}).

\section{Buffers}\label{sec:ad:buffer}

Buffers are conceptually very similar to queues.
But $\mathit{Buffer}[A]$ is usually optimized for enqueueing and dequeueing many elements of $A$ at once.
Therefore, while stacks and queues can be implemented well using linked lists, buffers usually use arrays to be faster.

A typical $\mathit{Buffer}[A]$ consists of three components:
\begin{compactitem}
 \item an $Array[A]$ $b$
 \item two integers $begin$ and $end$ indicating the first and last valid entry in the array.
\end{compactitem}
Enqueueing writes to $b[end+1]$ and increments $end$.
Dequeueing reads from $b[begin]$ and increments $begin$.


A buffer overflow occurs when incrementing $begin$ 

For example, when a browser receives a web page, its network component loads the page into a $\mathit{Buffer}[\Char]$.
In parallel, its HTML parser component starts processing the partially received page.
That way the HTML page can be displayed partially already before it is fully loaded.

Buffers are almost always used automatically when a program is writing to a file.
In that case, a $\mathit{Buffer}[\Int]$ or $\mathit{Buffer}[\Char]$ is used that holds the data written to the file.
The write command does not actually write data to the file directly---it only enqueues it in the buffer.
That is advantageous because enqueueing to a buffer in memory is much faster than writing to the hard drive.
While the program is already moving on, the programming language libraries or the operating system work in the background to periodically dequeue and write all characters to the file.

When working with files, an important operation is \emph{flushing} the buffer.
This forces the immediate processing of all data in the buffer.
Flushing happens automatically at the latest when the program terminates.
However, occasionally manual flushing is necessary:
\begin{compactitem}
 \item When a program terminates with an error, buffers have to be flushed to avoid losing data.
 \item When a program writes log data to a file that the programmer wants to read immediately, it is important to flush regularly to make sure the programmer reads updated information.
\end{compactitem}

\section{Iterators}\label{sec:ad:iter}

\subsection{Specification}

$Iterator[A]$ is a data structure for the set $A^*$.

Iterators are usually mutable.
Their functionality is even more restricted than the one of stacks and queues:

\begin{ctabular}{|l|l|l|}
\hline
function & returns & effect \\
\hline
$getNext(x\in A^*)\in A$ & the first element of $x$ & remove the first element of $x$ \\
$hasNext(x\in A^*)\in\Bool$ & $\true$ if $x$ is not empty & none \\
\hline
\end{ctabular}

The typical way to use an iterator $i\in Iterator[A]$ is the following:
\begin{acode}
\awhile{hasNext(i)}{
  a := getNext(i)\\
  \text{do something with $a$ here}
}
\end{acode}
This is called \textbf{traversing} the iterator.
Afterwards the iterator is traversed and cannot be used again.

$Iterator[A]$ may look somewhat boring.
In order to understand the value of iterators, we have to make one definition:
A data structure $D[A]$ is called \textbf{iterable} if there is a function
 \[iterator(x\in D[A])\in Iterator[A]\]

Now the imoprtance of iterators follows from two facts:
\begin{compactitem}
 \item Many data structures $D$ are iterable (see Sect.~\ref{sec:ad:iter:create}).
 \item Many important operations for $D$ can be realized using only the functionality of iterators (see Sect.~\ref{sec:ad:iter:use}).
\end{compactitem}
Thus, iterators provide a sweet-spot in the trade-off between simplicity and expressivity---they are very simple,  but we can do a lot with them.

\begin{remark}[Simplicity vs. Expressivity]
The trade-offs between simplicity and expressivity comes up again and again in computer science.
The best data structures combine both properties, but usually they are mutually exclusive.

All the important data structures presented in Part~\ref{sec:ad:ds} have become important because they do well in this way.
\end{remark}

\subsection{Working with Iterable Data Structures}\label{sec:ad:iter:use}

Let us assume an iterable data structure $D[A]$.
Our goal is to define functions on $x\in D[A]$ that use only $iterator(x)$.
There are indeed many of those.
Some important ones are:
\begin{ctabular}{|l@{}l@{}l|l|}
\hline
function &&& returns \\
\hline
\multicolumn{4}{|c|}{below, let $X=iterator(x)$}\\
$length$&$(x\in D[A])$&$\in \N$ & numbers of elements in $X$ \\
$contains$&$(x\in D[A],\; a\in A)$&$\in \Bool$ & $\true$ if $a$ occurs in $X$ \\
$index$&$(x\in D[A],\; a\in A)$&$\in \N^?$ & the position of the first occurrence of $a$ in $X$ (if any)\\
$find$&$(x\in D[A],\; p\in A\to\Bool)$&$\in A^?$ & the first element $a$ in $X$ (if any) such that $p(a)$ is $\true$ \\
$count$&$(x\in D[A],\; p\in A\to\Bool)$&$\in\N$ & the number of elements $a$ in $X$ for which $p(a)$ is $\true$ \\
$forall$&$(x\in D[A],\; p\in A\to\Bool)$&$\in \Bool$ & $\true$ if $p(a)$ is $\true$ for every element $a$ in $X$ \\
$exists$&$(x\in D[A],\; p\in A\to\Bool)$&$\in \Bool$ & $\true$ if $p(a)$ is $\true$ for some element $a$ in $X$ \\
$map$&$(x\in D[A],f\in A\to B)$&$\in Iterator[B]$ & an iterator for $[f(a_1),\ldots,f(a_n)]$ where $x=[a_1,\ldots,a_n]$ \\
$filter$&$(x\in D[A],p\in A\to \Bool)$&$\in Iterator[B]$ & like $x$ but skips elements that do not satisfy $p$ \\
$results$&$(x\in D[A],\;f\in A\to B)$&$\in List[B]$ & the list of results from applying $f$ to all $a$ in $X$ \\
$fold$&$(x\in D[A],\; b\in B, f\in A\times B\to B)$&$\in B$ & $f(a_1,f(a_2,\ldots,f(a_n,b))\ldots)$ with $X=[a_1,\ldots,a_n]$\\
\hline
\end{ctabular}
All of the above functions should not have a side-effect.
However, some of them take other functions as arguments.
It is usually a bad to do so, but it is technically possible that these functions have side-effects.
There is only one exception where we explicitly allow $f$ to have a side-effect:
\begin{ctabular}{|l|l|l|}
\hline
function & returns & effect \\
\hline
$foreach(x\in D[A],f\in A\to \Unit)\in \Unit$ & nothing & apply $f$ to all $a$ in $X$ \\
\hline
\end{ctabular}

The trick behind $map$ (and the difference to $results$) is that $x$ is not traversed right away.
Instead, we create a new iterator that, when traversed, applies $f$.
That way we ensure that $f$ is applied only as often as necessary.


\subsection{Making Data Structures Iterable}\label{sec:ad:iter:create}

We can give a data structure for iterators as an abstract class:
\begin{acode}
\aclassA{Iterator[A]}{}{}{
 \afun[\Bool]{hasNext}{}{}
 \acomment{precondition for $getNext$ is $hasNext==\true$}\\
 \afun[A]{getNext}{}{}
}
\end{acode}

Then we can define, e.g.,  $map$ as follows:
\begin{acode}
\aclass{Map[A,B]}{x:Iterator[A], f:A\to B}{Iterator[B]}{
  \afunI[\Bool]{hasNext}{}{x.hasNext}\\
  \afunI[B]{getNext}{}{f(x.getNext)}
}\\
\\
\afunI{map}{x:D[A], f:A\to B}{\anew{Map[A,B]}{iterator(x),f}}
\end{acode}

Many important data structures are naturally iterable, and that can be realized by implementing the abstract class.
That includes in particular all data structures for lists:
\begin{acode}
\aclass{ListIterator[A]}{l: List[A]}{Iterator[A]}{
  index := 0\\
  \afunI[\Bool]{hasNext}{}{index < length(l)} \\
  \afun[A]{getNext}{}{
    a := get(l, index) \\
    index := index + 1 \\
    a
  }
}\\
\\
\afunI[{Iterator[A]}]{iterator}{l:List[A]}{\anew{ListIterator}{l}}
\end{acode}

\section{Streams}

$Stream[A]$ is not a data structure for the set $A^*$.
Instead, it is a data structure for the set $A^\N$.

The set $A^\N$ contains functions $f:\N\to A$, which we can think of as infinite lists $[f(0),f(1),\ldots]$.
Because they are so similar to lists, they are usually treated together with lists, even though they do not realize the same set.

The set $A^\N$ is uncountable.
Therefore, not all possible streams are effective objects that can be represented in a physical machine.
However, for many practical purposes, it is fine to treat $Stream[A]$ as if it were the type of all possible streams.

$Stream[A]$ is usually implemented in the same way as $Iterator[A]$ with the understanding that $hasNext$ is always $\true$, i.e., the stream is never over.

Consequently, the functions on $Iterator[A]$ behave slightly differently when used for $Stream[A]$.
For exapmle:
\begin{compactitem}
 \item We cannot call $length$, $count$, $results$, $fold$, and $foreach$ on streams.
 \item We can call $contains$ on a stream. However, the function may run forever if the searched-for element is not in the stream.
 The same caveat applies to $index$, $find$, $forall$, and $exists$.
 \item We can call $map$ (because $map$ returns a new iterator without actually applying the map-function to all elements right away).
\end{compactitem}

\section{Heaps}\label{sec:ad:heaplists}

Heaps are formally defined in Sect.~\ref{sec:ad:heaps}.

$Heap[A,O]$ is not a data structure for the set $A^*$.
Instead, it is a data structure for the subset of $A^*$ containing only lists sorted according to $O$.
Therefore, heaps are very useful for sorting and prioritizing.
We discuss applications of heaps to lists in Sect.~\ref{sec:ad:heapqueue} and~\ref{sec:ad:heapsort}.

First we introduce some basic operations on heaps in Sect.~\ref{sec:ad:heapops}.

\subsection{Operations on Heaps}\label{sec:ad:heapops}

Because heaps are mostly used for efficiency, they are usually mutable.
The main operations on a heap are similar to those on a stack:

\begin{ctabular}{|l|l|l|}
\hline
function & returns & effect \\
\hline
$insert(x\in Heap[A,O], a\in A)\in\Unit$ & nothing & add $a$ to $x$ in any position\\
$extract(x\in Heap[A,O])\in A^?$ & the $O$-smallest element of $x$ (if any) & remove that element from $x$ \\
$find(x\in Heap[A,O])\in A^?$ & the $O$-smallest element of $x$ (if any) & none \\
\hline

\end{ctabular}

$insert$, $extract$, and $find$ for heaps correspond exactly to $push$, $pop$, and $top$ for stacks.
The crucial different is that $insert(x,a)$ does not prepend $a$ to $x$---instead, it is unspecified where and how $x$ is added.
$extract$ and $find$ do not return the most recently added element---instead, they return the smallest element with respect to $O$.

It is unspecified what exactly a heap looks like and where and how $insert$ actually performs the insertion.
That way heaps have a lot of freedom to organize the data in an efficient way.
That freedom is exploited to make the operations $extract$ and $find$ fast.
\medskip

Because $Heap[A,O]$ is underspecified, there are many different options how to implement heaps.
In practice, there are dozens of competing variants using different efficiency trade-offs.
A critical property is that all operations take only $O(\log n)$ where $n$ is the number of elements in the heap.

\subsection{Implementations}

The most important case of $Heap[A,O]$ are binary heaps $H$, i.e., binary trees over $A$ that are also heaps.
There is a wide variety of optimized implementations of heaps.


\subsubsection{Using Trees}

For a straightforward implementation, we use a tree.

Let $n$ be the number of nodes in $H$ and $h$ be the height of $H$.
All operations are such that $H$ remains almost-perfect: for every depth $d<h$ there are maximally many nodes, i.e., $2^d$ nodes.
At depth $h$, we have to allow for fewer than $2^h$ nodes because not every $n$ there is a perfect heap.
We use the convention that the nodes at level $h$ are as far to the left as possible.
That way, we always have $h\leq\log_2 n$, and all branches have length $h$ or $h-1$, i.e., $O(\log_2 n)$.

$find$ is trivial: We return the root of $H$. That takes $O(1)$.

$insert(H,x)$ inserts $x$ into one of the branches with minimal length.
If the heap is perfect, we extend it to a new level $h+1$ and insert $x$ all the way to the left.
Otherwise, we add it in the left-most free slot at level $h$.
The insertion occurs at the position that keeps the branch sorted.
Because it was sorted already, that requires $O(l)$ operations, where $l$ is the length of the branch, i.e., $O(\log_2 n)$.
It is easy to check that the resulting tree is again a heap.

$extract$ removes the root of $H$ and returns it.
That takes $O(1)$.
Additionally, we have to repair the heap property.
To do that, we take some leaf $l$ of $H$ and put it at the root.
Now we have an almost-perfect binary tree again, but it is not a heap yet: $l$ may be too big to be the root.
Therefore, we push $l$ down by iteratively swapping it with its smallest child until we have a heap.
Finding a leaf and pushing along some branch takes $O(\log_2 n)$.

\subsubsection{Using Arrays}

For efficiency, it is often preferable to store the nodes of the heap as an array.

This requires a bijection that translates between positions in the heap to positions in the array.
Let $h$ be the height of the heap.
Then a position in the heap is a list $p\in\{left,right\}^*$ of length up to $h$ that describes the path from the root to a node.
A position in the array is an integer $i\in\{0,\ldots,2^{h+1}-1\}$.

It is straightforward to give such a bijection.
For example, we can number the nodes of the heap in BFS order.
Then the heap-position $p$ corresponds to the array position $2^{length(p)}+j$ where $j$ is the number obtained by treating $p$ as a binary number (with $left$ and $right$ corresponding to $0$ and $1$).

\subsection{Priority Queues}\label{sec:ad:heapqueue}

A $PriorityQueue[A]$ behaves like a $Queue[A]$ except that dequeueing returns the element with the highest priority.

This is achieved by using a data structure for $Heap[A,O]$ where $O$ orders elements by decreasing priority.
Then $insert$ and $extract$ correspond to $enqueue$ and $dequeue$.

\subsection{Heapsort Algorithm}\label{sec:ad:heapsort}

Heapsort is a sorting algorithm that runs in $\Theta(n\log n)$.

If $\leq$ is the total order for sorting, a simple heapsort is given by
\begin{acode}
\afun[A^*]{heapsort}{x: A^*}{
  h := \anew{Heap[A,\geq]}{}\\
  \\
  left := x\\
  \afor{i}{0}{length(x)-1}{
    next := left.head \\
    insert(h,next)\\
    left := left.tail\\
  }\\
  res := Nil\\
  \afor{i}{0}{length(x)-1}{
    next := extract(h)\\
    res := prepend(next, res)
  }
}
\end{acode}

This uses two loops using $length(n)$ iterations each.
The first loop throws all elements of $x$ into the heap; the second loop pulls them out again and builds the list $res$ to be returned.
Because $extract$ always returns the greatest element, the result is automatically sorted.
Any other implementation of a priority queue yields a corresponding sorting algorithm.

If $n$ is the length of the list, each $insert$ and $extract$ operation takes at most $\Theta(\log n)$.
Thus, heapsort runs in $\Theta(n\log n)$.
\medskip

There are much more optimized implementations of heapsort than the above example, possibly using optimized implementations of heaps.
In particular, there are encodings of the heap structure in an array, which allow using heapsort as an in-place sorting algorithm.
With those optimizations, heapsort is among the fastest sorting algorithms (but still takes $\Theta(n\log n)$).

\chapter{Tree-Like Data Structures}\label{sec:ad:trees}
 \input{tree}

\chapter{Set-Like Data Structures}\label{sec:ad:sets}
 \section{Specification}

The set $Set[A]$ contains the finite subsets of $A$.
It is countable if $A$.

Sets can be mutable or immutable.
The main operations for immutable sets are:

\begin{ctabular}{|l|l|l|}
\hline
function & returns & effect \\
\hline
$contains(x\in Set[A], a\in A)\in \Bool$ & $\true$ iff $a\in x$ & none \\
$insert(x\in Set[A], a\in A)\in Set[A]$ & $x\cup\{a\}$ & none \\
$delete(x\in Set[A], a\in A)\in Set[A]$ & $x\sm\{a\}$ & none \\
\hline
\end{ctabular}

% lattice operations

The main operations for mutable sets are:

\begin{ctabular}{|l|l|l|}
\hline
function & returns & effect \\
\hline
$contains(x\in Set[A], a\in A)\in \Bool$ & $\true$ iff $a\in x$ & none \\
$insert(x\in Set[A], a\in A)\in\Unit$ & nothing & $x:=x\cup\{a\}$ \\
$delete(x\in Set[A], a\in A)\in\Unit$ & nothing & $x:=x\sm\{a\}$ \\
\hline
\end{ctabular}

In both cases, we often need operations for combining and comparing sets.
We only consider the immutable case:
\begin{ctabular}{|l|l|l|}
\hline
function & returns & effect \\
\hline
$equal(x\in Set[A], y\in Set[A])\in \Bool$ & $\true$ iff $x=y$ & none \\
$union(x\in Set[A], y\in Set[A])\in Set[A]$ & $x\cup y$ & none \\
$inter(x\in Set[A], y\in Set[A])\in Set[A]$ & $x\cap y$ & none \\
$\mathit{diff}(x\in Set[A], y\in Set[A])\in Set[A]$ & $x\sm y$ & none \\
\hline
\end{ctabular}

Equality is listed explicitly here because it can be very complex.
For most data structures such as the ones for lists and trees, equality is straightforward.
This may or may not be the case for data structures for sets.

\section{Data Structures}

The complexity of data structures for sets is usually measured in terms of the size $n$ of the set.

\subsection{Bit Vectors}\label{sec:ad:vectorset}

\subsubsection{Design}
If $A$ is finite with $|A|=m$, an easy data structure for $Set[A]$ are bit vectors of length $m$ such as $Array[\Bool](m)$.
Given such a vector $a$, we put $a[i]=\true$ to represent that $i$ is in the set.

\subsubsection{Complexity}
We can implement $insert$ and $delete$ easily in $\Theta(1)$.
We can also implement $equal$, $union$, $inter$, and $\mathit{diff}$ easily in $\Theta(m)$.

A major drawback is the memory requirement: We need $\Theta(m)$ for each $x:Set[A]$, which is only feasible for small $m$.

\subsection{List Sets}\label{sec:ad:listset}

\subsubsection{Design}
If $A$ is much larger than the sets to be represented, a better data structure for $Set[A]$ is $ListSet[A]$.
It represents the set $\{a_1,\ldots,a_n\}$ as the list $[a_1,\ldots,a_n]$.
Thus, it represent sets as lists without repetition.

The operations on $ListSet[A]$ are defined in the same way as for $List[A]$ with one exception: the $insert(x,a)$ operation does nothing if $x$ already contains $a$.

\subsubsection{Complexity}
If $n$ is the size of the $ListSet$, the operations $contains$, $insert$, and $delete$ takes $\Theta(n)$.
However, higher-level operations like building a set with $n$ elements step-by-step by calling $insert$ $n$ times requires $n$ insertions and thus costs $\Theta(n^2)$.

Moreover, these operations require calls to the equality on $A$.
For example, to implement $contains(x,a)$, we have to compare $a$ to every element of $x$.
That may be easy, e.g., if $A=\Int$.
But it can be arbitrarily costly if $A$ is more complex data structure itself.

For equality, union, intersection, and difference of $x$ and $y$, we may have to compare every element of $x$ with every element of $y$.
So it may take $O(|x|\cdot|y|)$.

These operations quickly become too costly for large subsets of $A$.

\subsection{Hash Sets}\label{sec:ad:hashset}

\subsubsection{Design}
Hash sets try to combine the advantages of bit vector and list sets.
The key parameter is a function $hash:A\to \Z_m$.
This is called the hash function.

$hash$ has two purposes:
\begin{compactitem}
 \item The set $A$ is supported by a finite, managably small set $\Z_m$.
   That makes if feasible to use arrays of length $m$.
 \item The equality operation on $A$ is supported by the $O(1)$ equality on $\Z_m$.
   To check $a=a'$, we first check $hash(a)=hash(a')$.
   If false, we know $a\neq a'$; otherwise, we call the usual equality on $A$.
   That minimizes the number of equality on $A$ is called.
\end{compactitem}

Of course, the function $hash$ will usually not be injective.
A \textbf{collision} is a pair $x,y\in A$ such that $hash(x)=hash(y)$.

A good hash function should be fast and rarely have collisions.
An (unrealistically) ideal hash function runs in $O(1)$ and the probability of $hash(x)=hash(y)$ is $1/m$.
Those two properties work against each other: For example, it is easy to be fast by always returning $0$, but that has maximally many collisions.
Vice versa, it is easy to minimize collisions by choosing $hash$ carefully, but then $hash$ may be very expensive to compute.
Thus, hash functions must make a trade-off.

For a fixed hash function $hash:A\to \Z_m$, the data structure $HashSet[A]$ is given by
\begin{acode}
\atypedef{HashSet[A]}{Array[ListSet[A]](m)}\\
\afunI{insert}{h: HashSet[A], a:A}{insert(h[hash(a)],a)}\\
\afunI{delete}{h: HashSet[A], a:A}{delete(h[hash(a)],a)}
\end{acode}
The elements of the array are called hash \textbf{buckets}.
Thus, the bucket for $i$ contains all elements of the set whose hash value is $i$.

\subsubsection{Complexity}
If $n$ is the size of the subset of $A$, the sets $h[0]$, \ldots, $h[m-1]$ have average size $n/m$.
Thus, $contains$, $insert$ and $delete$ take $n/m$ on average.
$equal$, $union$, $inter$, and $diff$ are similarly sped up.
\medskip

Asymptotically, hash sets do not beat list sets because they only spped up by a constant factor.
But that constant factor is a critical improvement.

The speed up is bigger if $m$ is bigger.
However, the memory requirement increases linearly with $m$: Even the empty subset requires $\Theta(m)$ space and $\Theta(m)$ time to initialize that space.

Optimized data structures for hash sets can dynamically choose $m$ in order to find a good trade-off.
Often users of the $HashSet$ data structure can choose the value of $m$.
That can help if they know in advance how big the subset is going to get and what kind of operations will be called.

\subsection{Binary Search Trees}\label{sec:ad:bst}

\subsubsection{Design}
If we have a total order $\leq$ on $A$, we can use binary trees to realize $Set[A]$.
The idea is that the nodes of the tree hold the elements of the set, and every node $n$ splits a range of values into two subranges: all left descendants of $n$ have smaller and all right descendants have greater values than $n$.

\begin{example}
We represent the set $\{20,29,30,33,34,37\}\sq \N$ as the following binary tree over $\N$:

\begin{center}
\begin{tikzpicture}[scale=.7]
\node[] (0) at (0,0) {$30$};
\node[] (00) at (-2,-1) {$20$};
\node[] (01) at (2,-1) {$35$};
\node[] (001) at (-1,-2) {$29$};
\node[] (010) at (1,-2) {$33$};
\node[] (011) at (3,-2) {$37$};
\node[] (0101) at (1.5,-3) {$34$};
\draw[arrow] (0) -- (00);
\draw[arrow] (0) -- (01);
\draw[arrow] (00) -- (001);
\draw[arrow] (01) -- (010);
\draw[arrow] (01) -- (011);
\draw[arrow] (010) -- (0101);
\end{tikzpicture}
\end{center}
Now we can locate an element efficiently by traversing just one branch of the tree.

Note how a binary search tree has to distinguish between the left and the right subtree even if there is only one subtree (e.g., for the nodes for $20$ and $33$).
In those cases, we have to use a dummy node as the omitted left child.
These are usually labeled with $null$ or $\bot$ and omitted when drawing the tree.
\end{example}

The formal definition is as follows:

\begin{definition}[Binary Search Trees]
$BST[A,O]$ is the subset of $Tree[A^?]$ containing only full binary trees satisfying the following properties:
\begin{compactitem}
 \item All leafs are labeled with the value $\bot$; all other nodes are labeled with elements of $A$.
 \item For every non-leaf node $n$:
  \begin{compactitem}
    \item All values in the left subtree of $n$ are strictly smaller than the one at $n$.
    \item All values in the right subtree of $n$ are strictly greater than the one at $n$.
  \end{compactitem}
\end{compactitem}
\end{definition}

The leafs hold dummy values as placeholders for elements that we may want to insert later.
When drawing binary search trees, we always omit them.
The reason for labeling all leafs with $\bot$ is a technicality: we already need dummy leafs for omitted left children; then it can be simpler to just define that all leafs are dummy nodes.

\subsubsection{Complexity}

Ideally, the binary tree is perfect or nearly perfect.
That happens when every node is labeled with the median of all the values among its descendants.

In that case, the height of the tree is logarithmic in the size $n$ of the set.
Then $contains$, $insert$, and $delete$ can be implemented in $O(\log n)$.
\medskip

But a series of random insertions and deletions may make the tree arbitrarily imperfect.
In the worst case, the tree degenerates to a list.
Therefore, a binary search tree must be rearranged from time to time.

This can be done from scratch in one big operation.
For example, we can do a depth-first traversal of the tree to obtain a sorted list of all elements.
Then we can create a perfect binary tree of height $\log n$ and put the elements into it.
Overall that takes $\Theta(n)$.

Alternatively, we can rearrange the tree incrementally.
Here we try to make minor changes to the tree after every insertion or deletion.
To keep the tree near perfect.
One way to do that is to use red-black trees (see Sect.~\ref{sec:ad:redblacktree}).

\subsection{Red-Black Trees}\label{sec:ad:redblacktree}

Red-black trees are a sophisticated variant of binary search trees.
They guarantee $\Theta(\log n)$ cost for for insertion and deletion.

\subsection{Characteristic Functions}\label{sec:ad:charfun}

\footnote{This section was not part of the course.}

For every set $x:Set[A]$, we can define its characteristic function $\mathtt{Char}_x:A\to\Bool$ by
\[\mathtt{Char}_x(a) = contains(a,x)\]

In mathematics, the inverse is true as well: for every function $f:A\to \Bool$, we can define the set $\{x: A|f(x)\}$.
However, we can usually not implement that set in programming languages.
That is easy to see: bit vectors, list sets, hash sets, or binary search trees can only ever represent \emph{finite} sets.
But functions $f:A\to\Bool$ can easily represent \emph{infinite} sets as well.
For example, \[\alam[\String]{x}{\true}\tb:\tb\String\to \Bool\] is the characteristic function of the infinite set of all strings.

Therefore, characteristic functions are a representation of sets that is more expressive than all of the above.
If we want to represent infinite sets, characteristic functions are the only option.

All set operations on characteristic functions can be implemented in $O(1)$ with the following exceptions:
\begin{compactitem}
 \item $contains$ must call the characteristic function. There is no way to predict how long that will take.
 \item It is impossible to implement $equal$ correctly if $A$ is infinite.
 \item Operations that require finiteness such as computing the size of a set cannot be implemented.
\end{compactitem}

\chapter{Graph-Like Data Structures}\label{sec:ad:graphs}
 After lists, and trees, graphs are the most important data structure in computer science.
In fact, just like lists are a special of trees, trees are a special of graphs.

Data structures for lists and trees are of course used to represent lists and trees.
But they are also used to represent other data.
For example, we can represent a set as a list (Sect.~\ref{sec:ad:listset}) or as a tree (Sect.~\ref{sec:ad:redblacktree}) or a list as a tree (Sect.~\ref{sec:ad:heaplists}).
That is because choosing the more complex data structure (i.e., a tree instead of a list) can allow for more efficient algorithms.

Data structures for graphs on the other hand are almost exclusively used to represent graphs.
That is because they are rather difficult to work with.
But they are needed because graph-structured data occurs very frequently.

Practical examples of graph-structured data include
\begin{compactitem}
 \item Social networks: nodes are given by people and edges by the social connection relation.
  These relations may be asymmetric (e.g., the follows-relation of twitter), which leads to directed graphs, or symmetric (e.g., the are-friends-with-each-other relation of facebook), which leads to undirected graphs.
 \item Roads: nodes are given by cities and intersections and edges by roads between them. Flight routes, subway systems, etc. work accordingly.
 \item Utilities supply networks for water, electricity, internet, etc.: nodes are given by power plants/switchboards/hubs/etc. and households and edges by pipes/cables/etc.
 \item Mazes and other explorable territories: nodes are given by intersections and edges by paths.
 \item Games: generalizing the tree intuition from Sect.~\ref{sec:ad:minmax}, nodes are given by states and edges by steps/moves/developments.
 \item Neighborhood relations: nodes are given by countries or other territories and edges by the share-a-border-with-each-other relation.
\end{compactitem}

In fact, binary function and binary relations are the most ubiquitous complex mathematical objects.
Algorithms are our primary representation method for the former and graphs for the latter.

\section{Specification}

\begin{definition}[Graph]
A \textbf{graph} is a pair $G=(N,E)$ such that $E$ is a binary relation on $N$.

If $E$ is symmetric, $G$ is called \textbf{undirected}, otherwise \textbf{directed}.
\end{definition}

The set $N$ is usually but not necessarily finite.

Like for trees, there are many definitions to talk about the parts of a graph:

\begin{definition}[Parts of a Graph]
Consider a graph $G=(N,E)$.

An element $n\in N$ is called a \textbf{node} or a \textbf{vertex}.
An element $(m,n)\in E$ is called an \textbf{edge} from $m$ to $n$.
It is also called an \textbf{incoming edge} of $n$ and an \textbf{outgoing edge} of $m$.

For every node $n$, the number of incoming edges is called the \textbf{in-degree} of $n$, and the number of outgoing edges is called the \textbf{out-degree} of $n$.
If $G$ is undirected graph, incoming and outgoing edges are the same, and we simply speak of the \textbf{degree} of $n$.

A \textbf{path} from $a_0$ to $a_n$ is a list $[a_0,\ldots,a_n]\in N^*$ such that there is an edge from $a_{i-1}$ to $a_i$ for $i=1,\ldots,n$.

$n$ is called the \textbf{length} of the path.
If $n=0$ (and thus $a_0=a_n$), the path is called \textbf{empty}.
If there is a path from $a_0$ to $a_n$, then $a_n$ is called \textbf{reachable} from $a_0$.

A \textbf{cycle} is a non-empty path from $a$ to itself.
If $G$ has (no) cycles, it is called \textbf{(a)cyclic}.

Let us write $\ov{G}$ for the undirected graph $(N, E\cup E^{-1})$ in which all edges go both ways.
Then $G$ is called \textbf{connected} if all nodes in $\ov{G}$ are reachable from each other.

A \textbf{clique} is a subset $C$ of $N$ such that there is an edge from every $a\in C$ to every other $b\in C$.
$G$ is called \textbf{complete} if $N$ is a clique.
\end{definition}


\paragraph{Visualization}
A good intuition to think of graphs is to imagine the nodes as places and the edges as streets between them.
In a directed graph, all edges are one-way streets.

All concepts about graphs also have very intuitive visual aspects:

\begin{ctabular}{|l|l|l|}
\hline
& \multicolumn{2}{c|}{Visual Intuition}\\
Concept & undirected &  directed \\
\hline
node & \multicolumn{2}{c|}{point}  \\
edge from $a$ to $b$ & line from $a$ to $b$ & arrow from $a$ to $b$\\
incoming edge of $a$ & &  arrow pointing at $a$ \\
outgoing edge of $a$ & &  arrow pointing away from $a$ \\
$b$ reachable from $a$ & we can walk from $a$ to $b$ along edges & \ldots in arrow direction \\
path from $a$ to $b$ of length $n$ & a walk from $a$ to $b$ in $n$ steps & \ldots in arrow direction \\
{weight\footnotemark} of an edge & \multicolumn{2}{c|}{cost intuition: length of the line}\\ 
                               & \multicolumn{2}{c|}{capacity intuition: width of the line}\\ 
complete & we can walk everywhere in $1$ step &  \ldots in arrow direction \\
cycle & we can walk in a circle & \ldots in arrow direction \\
connected & \multicolumn{2}{c|}{graph can be drawn in one stroke} \\
\hline
\end{ctabular}
\footnotetext{See below for weighted graphs.}

\paragraph{Reachability Relation}
Many graph properties are just rephrasings of or closely related to relation properties.
Most importantly:

\begin{theorem}[Reachability]
For every graph $G$, the relation ``$b$ is reachable from $a$'' is
\begin{compactitem}
 \item reflexive and transitive
 \item symmetric iff $G$ is undirected
 \item anti-symmetric iff $G$ is acyclic
\end{compactitem}
\end{theorem}
\begin{proof}
Exercise.
\end{proof}

\paragraph{Labeled Graphs}
Like for trees, graphs are only useful for computation, if we can store data in them.
Contrary to trees, we often need to store data in the nodes \emph{and} the edges.

\begin{definition}[Labeled Graph]
A $A$-$B$-\textbf{labeled} graph is a triple of
\begin{compactitem}
 \item a graph $G=(N,E)$
 \item a function $nodeLabel:N\to A$
 \item a function $edgeLabel:E\to B$
\end{compactitem}
$Graph[A,B]$ is the set of $A$-$B$-labeled graphs.
\end{definition}

The most important special case arises when the nodes are not labeled (i.e., we put $A=\Unit$) and the edges are labeled with numbers, i.e., $B=\Z$:

\begin{definition}
A \textbf{weighted} graph is a $\Unit$-$B$-labeled graph where $B$ is any set of numbers.
The label of an edge is called its \textbf{weight}.
\end{definition}
Most often $B$ is $\N$, but $\Z$, $\R$, or $\Z^\infty$, etc. are also common.

There are two important applications of weighted graphs that use different interpretations of the weights:
\begin{itemize}
 \item Cost intuition: The weight of an edge is the cost of moving along the edge.
 For example, if the nodes represent cities and the edges flight routes, the weight can be the distance.
 \item Capacity intuition: The weight of an edge is the capacity for moving objects along the edge.
 For example, if the nodes represent cities and the edges flight routes, the weight can be the number of flights per day.
\end{itemize}

Correspondingly, we define:
\begin{definition}
Consider a weighted graph.
We write $weight(i,j)$ for the weight of the edge from $i$ to $j$.

We make $weight:N\times N\to \N^\infty$ a total function by using a default value whenever there is no edge from $i$ to $j$:
\begin{compactitem}
 \item A \textbf{cost-weighted} graph uses the default $weight(i,j)=\infty$.
 \item A \textbf{capacity-weighted} graph uses the default $weight(i,j)=0$.
\end{compactitem}

In a cost-weighted graph, the \textbf{cost of a path} is the sum of the weights of all edges.

In a capacity-weighted graph, the \textbf{capacity of a path} is the minimal weight of any edge in it.
\end{definition}

The intuition behind the default values is that if there is no edge from $i$ to $j$,
\begin{compactitem}
 \item cost intuition: it is impossible to go from $i$ to $j$, i.e., which corresponds to infinite cost,
 \item capacity intuition: it is impossible to move objects from $i$ to $j$, which corresponds to empty capacity.
\end{compactitem}

\section{Data Structures}

We consider only graphs $G=(N,E)$ where $|N|=m$, i.e., is finite.
We number the elements of $N$ (in any order) so that we can assume $N=\Z_m$.

Moreover, we consider only graphs whose edges are labeled with weights from set $B$.
Note that an unlabeled graph can be seen as the special case where $B=\Unit$.

Graphs are among the trickiest data structures to design because it is difficult to represent the set $E$ efficiently.
There are a number of frequently-needed operations, whose complexity depends on the data structure.
We may want to obtain
\begin{compactitem}
 \item $edge(G,i,j)$: for two nodes $i$, $j$, the edge from $i$ to $j$ (if any),
 \item $outgoing(G,i)$: for a node $i$, the list of outgoing edges,
 \item $incoming(G,i)$: for a node $i$, the list of incoming edges,
 \item $edges(G)$: an iterator over all edges.
\end{compactitem}

If we have one of those operations, we can compute the others---but not necessarily efficiently.

\subsection{Adjacency Matrix}

\paragraph{Design}
An often useful representation is via a matrix, called the \textbf{adjacency matrix} of $G$.
It stores the entire function that maps two nodes to their edge in a single object.

\begin{definition}[Adjacency Matrix]
The adjacency matrix of an unlabeled graph $G$ is the matrix $Adj\in \Bool^{mm}$ where $Adj_{ij}==\true$ iff there is an edge from $i$ to $j$ in $G$.
\end{definition}

Adjacency matrices have the nice property that we can multiply them.
Here matrix multiplication is computed using conjunction and disjunction instead of multiplication and addition:
\begin{definition}
For $X,Y\in \Bool^{mm}$, we define $(X\cdot Y)_{ik}:= \bigvee_{j=0,\ldots,m-1}X_{ij}\wedge Y_{jk}$.
\end{definition}
This is useful because:
\begin{theorem}
If $Adj$ is the adjacency matrix of $G$, then $(Adj^n)_{ij}$ iff there is a path of length $n$ from $i$ to $j$ in $G$.
\end{theorem}

This is advantageous because it lets us compute all paths of length $n$ efficiently by computing $Adj^n$, e.g., using square-and-multiply (Sect.~\ref{sec:ad:exp:sqmult}).
Moreover, in an acyclic graph, there are only finitely many paths.
Thus, we eventually have $Adj^n=Adj^{n+1}=\ldots$, at which point we have computed all paths.

If the edges are labeled with weights from $B$, we can use an adjacency matrix $W\in B^{mm}$ such that $W_{ij}$ is the weight of the edge from $i$ to $j$.
If there is no edge, we use a default value, e.g., $\infty$, $0$, or $null$.

\paragraph{Complexity}
A drawback of the adjacency matrix is that its size $|N|^2$.
Moreover, for an undirected graph, half the space is wasted because we always have $Adj_{ij}=Adj_{ji}$.

$edge(G,i,j)$ takes $\Theta(1)$ (assuming we store the matrix efficiently using arrays).

$outgoing(G,i)$ and $incoming(G,i)$ must pull out a row or column from the adjacency matrix.
That takes $\Theta(1)$ or $\Theta(m)$, depending on how we store the arrays.

$edges(G)$ takes $\Theta(1)$ because we can use the iterator of the array.

\subsection{Adjacency Lists}

\paragraph{Design}
For graphs with many nodes and few edges, it is better to store adjacency lists.
Those are the lists of outgoing edges for each node:

\begin{definition}[Adjacency List]
For an unlabeled graph, the adjacency list of a node $i$ is the sorted list $L_i\in List[\Z_m]$ of all $j$ such that there is an edge from $i$ to $j$ in $G$.

The adjacency list--representation of $G$ consists of an list $[(0,L_0),\ldots,(m-1,L_{m-1})]$ pairing every node with its adjacency list.
\end{definition}

If the edges are labeled with weights from $A$, the adjacency list $L_i\in List[\Z_m\times A]$ contains pairs $(j,w)$ where $W$ is the weight of the edge from $i$ to $j$.
We do not need default values because we can simply omit those $j$ for which there is no edge.

\paragraph{Complexity}
The size of the adjacency list--representation is $|N|+|E|$, which is usually much smaller than $|N|^2$.

$edge(G,i,j)$ takes $\Theta(d)$ where $d$ is the maximal number of outgoing edges of any node.
$d$ is usually much smaller than $|E|$.

$outgoing(G,i)$ takes $\Theta(1)$.

$incoming(G,i)$ is inefficient. We have to check each node for an edge into $i$ and collect the results.

$edges(G)$ takes $\Theta(m)$ because we have to concatenate $m$ iterators.

\subsection{Opposite Adjacency List}

\paragraph{Design}
The adjacency list stores for every node the list of \emph{outgoing} edges.

For directed graphs, we can alternatively store the list of incoming edges.
That is the same as storing the adjacency lists for the graph in which all edges are flipped.

\paragraph{Complexity}
All results are opposite to the previous case.

\subsection{Redundant Adjacency Lists}

\paragraph{Design}

If we want to be able to do both, it is best to store both adjacency lists for each node.

\paragraph{Complexity}
The size of the representation is now twice as big as when using a single adjacency list.

$edge(G,i,j)$ takes $\Theta(d)$ as before.

$outgoing(G,i)$ and $incoming(G,i)$ take $\Theta(1)$.

$edges(G)$ takes $\Theta(m)$ because we have to concatenate $m$ iterators.

The main drawback of this data structure is that it is more difficult to implement because we have to keep the adjacency lists in sync.
Every time we add or remove an edge from $i$ to $j$, we have to change two lists, namely $incoming(G,j)$ and $outgoing(G,i)$.

\section{Important Algorithms}

Unless mentioned otherwise, we use a connected finite directed graph $G=(N,E)$, whose edges are labeled with natural numbers.
We assume $N=\Z_m$.

\subsection{Reachability}\label{sec:ad:reachability}

We work with unlabeled graphs in this section.

\paragraph{Problem}
Given a start node $n\in N$, our goal is to explore all nodes that are reachable from $n$.

This problem of exploring all reachable nodes comes up all the time in practice.
In a maze, it is a way to find the exit.
In a game, it means to analyze a particular situation in the game.

\paragraph{Algorithm}
We use a data structure for which $outgoing$ is efficient.
Then we start at $n$ are recursively call $outgoing$ until we reach no further nodes.

Like for trees, there are two typical options: DFS and BFS.
Contrary to trees, we have to watch for cycles: if $G$ has a cycle and we do not keep track of which nodes we have found already, we would never terminate.
Therefore, we use two auxiliary data structures:
\begin{compactitem}
 \item a set $reachable:List[N]$ where we store all nodes in the order in which we found them and which we return eventually
 \item a queue $horizon:List[N]$ where we store all nodes that we have found already but whose outgoing edges we have not yet looked at
\end{compactitem}

For BFS, the algorithm looks as follows:

\begin{acode}
\afun[{List[N]}]{BFS}{G:Graph,start:N}{
	reachable: List[N] = Nil\\
	horizon: Queue[N] = \anew{Queue[N]}{}\\
	enqueue(horizon, start)\\
	\awhile{!empty(horizon)}{
	  i = dequeue(horizon)\\
	  reachable = prepend(i, reachable)\\
	  foreach(outgoing(G,i), \alam{j}{\ablock{
	    \aif[the critical check to avoid cycles]{!contains(reachable,j) \wedge !contains(horizon,j)}{
	      enqueue(horizon,j)
	    }
	  }}\\)
	}\\
	reverse(reachable)
}
\end{acode}

Like for trees, we obtain a DFS exploration algorithm if we use a stack instead of a queue.

\paragraph{Correctness}
The postcondition is that $BFS$ return the nodes reachable from $start$.
As a loop invariant we use that
\begin{compactitem}
 \item every node reachable from $start$ or from a node in $reachable$ is
		\begin{compactitem}
		 \item an element of $reachable$ or
		 \item reachable from some element of $horizon$,
		\end{compactitem}
	and
 \item all nodes in $reachable$ or $horizon$ are reachable from $start$.
\end{compactitem}

As a termination ordering we use $|N|-length(reachable)$.
Indeed, reachable grows in every iteration of the loop and (because it never contains duplicates) becomes at most $|N|$.

\paragraph{Complexity}
Let $r<|N|$ be the number of reachable nodes and $e$ be the number of edges between reachable nodes.
Then $BFS$ takes $\Theta(r+e)$ because every reachable node is visited once and every edge is traversed once.

\subsection{Minimal Spanning Tree}\label{sec:ad:spanningtree}

We work with undirected graphs in this section.

\paragraph{Problem}
A \textbf{spanning tree} for a graph $G$ is a subgraph of $G$ that is a tree and includes all nodes of $G$.
In other words, it is a subset of edges of $G$ such that $G$ becomes a tree.

Spanning trees are closely related to the exploration of reachable nodes: they provide a minimal set of edges needed to reach all reachable nodes.
For example, when planning to roads or laying cables a spanning tree can provide the minimal amount of connections needed to reach all households.

We can capture the minimality condition with the following observations.
A spanning tree has $|N|-1$ edges because every node but the root has exactly one incoming edge.
It is not possible to use fewer edges because any connected subgraph of $G$ that includes all nodes must have at least $|N|-1$ edges: $1$ edge can connect two nodes; any additional edge can connect only one additional node.

The problem becomes more interesting if the edges of $G$ are cost-weighted.
Then the \textbf{weight of a subgraph} $T$ of $G$ is the sum of the weights of the edges in $T$.
Then our goal is not only to find \emph{some} spanning tree but a minimal one, i.e., a spanning tree with the least possible weight.

For example, when planning roads, the cost of an edge could be the distance (more generally: the financial cost of building the road) between two points.
A minimal spanning tree minimizes the overall cost of building the roads.

\paragraph{Algorithm}
The BFS and DFS exploration algorithms immediately yield algorithms to find \emph{some} spanning tree.
All we have to do is to store the nodes we find in a $reachable: Tree[N]$ instead of $reachable:List[N]$.

It is less obvious how to find a minimal spanning tree of a cost-weighted graph.
One of the most well-known solutions is Kruskal's algorithm:

\begin{acode}
\acomment{$G=(N,E)$}\\
\afun[{Set[E]}]{Kruskal}{G:Graph}{
  edges := \text{list of edges $e\in E$, sorted by increasing weight}\\
  sot := \anew{Set[E]}{}\\
  foreach(edges, \alam{e}{\aifI{isSetOfTrees(sot\cup\{e\})}{insert(sot,e)}})\\
  sot
}
\end{acode}
The algorithm returns the set of edges that make up a minimal spanning tree.
Here the function $isSetOfTrees(S: Set[E]):\Bool$ has the following postcondition: if $S\sq E$ is a set of edges of $G$, then $isSetOfTrees(S)$ is true if the subgraph of $G$ containing only the edges in $S$ is a set of trees.\footnote{Such a graph is called a \textbf{forest}.}

\paragraph{Correctness and Complexity}
This is an example of a greedy algorithm.
We discuss its correctness and complexity in Ch.~\ref{sec:ad:greedy}.

\subsection{Cheapest Path}\label{sec:ad:shortestpath}

In this section, we interpret all edge weights as costs.
We think of it as the cost of moving along the edge, where the cost represents the physical distance or any kind of resource (money, gas, effort, amount of material, etc.) that must be expended.
Recall that the cost of a path is the sum of the costs of its edges.

When comparing paths, the literature often uses the words length/short/long, whereas we will use the words cost/cheap/expensive.
We do that to avoid confusing the cost of a path with its number of edges, both of which are often called the \emph{length}.
The cheapest path does not necessarily contain the least number of edges: a detour along cheap edges can lead to lower cost than the direct path along expensive edges.
For example, direct flights usually cost more money than multi-leg flights, and the cheapest flight route is usually not a direct flight.

\paragraph{Problem}
We are interested in finding the cheapest paths.
This has obvious applications in logistics, navigation, and similar situations.
Whenever we move in any kind of network, we prefer taking the cheapest path.

Note there may be multiple different paths with the same cost. So technically, we are looking for \emph{a} cheapest path, not \emph{the} cheapest path.

We write $Cost(p)$ for the cost of a path and $Cost(i,j)$ for the cost of the lowest cost of any path from $i$ to $j$.

There are multiple variants of the cheapest path problem:
\begin{compactitem}
 \item for fixed nodes $i$ and $j$, find a cheapest path from $i$ to $j$,
 \item for a fixed start node $i$, find cheapest paths to any node $j$,
 \item for a fixed end node $j$, find cheapest paths from any node $i$,
 \item find cheapest paths for all pairs of node $i$ and $j$.
\end{compactitem}

Clearly, all variants can be reduced to each other.
But that might not be efficient: finding the cheapest paths for all pairs at once can be much faster than the total time of doing so individually for each pair.
That is because cheapest paths are related, especially:
\begin{compactitem}
  \item if $p$ is a cheapest path, then so is any sub-path of $p$
  \item if $p$ and $q$ are cheapest paths from $i$ to $j$ and $j$ to $k$, then $Cost(i,k)\leq Cost(p)+Cost(q)$.
\end{compactitem}

\paragraph{Algorithm}
We look at the special case where we have a fixed start node $i$ and want to find the cheapest path to every node.
It is easy to see that these paths from a spanning tree of $G$ with root $i$.

One well-known solution is Dijkstra's algorithm.
It is also a greedy algorithm: we start with the root only, and each iteration adds to the tree whichever node is cheapest to reach.

\begin{acode}
\acomment{$G=(N,E)$, $m=|N|$}\\
\afun[{Array[List[N]]}]{Dijkstra}{G:Graph,start:N}{
  cheapest := \anew{Array[List[N]]}{m} \acomment{$cheapest[j]$ is the cheapest known path from $start$ to $j$}\\
  foreach(N, \alam{n}{cheapest[n] :=  null}) \acomment{no path known yet}\\
  cheapest[start] := [start] \acomment{empty path}\\
  \\
  cost := \anew{Array[\N^\infty]}{m} \acomment{$cost[j]$ is the cost of $cheapest[j]$}\\
  foreach(N, \alam{n}{cost[n] := \infty}) \acomment{no path known yet}\\
  cost[start] := 0 \acomment{cost of empty path}\\
  \\
  rest := \anew{Set[N]}{}\\
  foreach(N, \alam{n}{insert(rest,n)}) \acomment{remaining nodes to add to spanning tree}\\
  \awhile{!empty(rest)}{
    i := \text{a node $i\in rest$ with minimal value of $cost[i]$}\\
    delete(rest,i)\\
    foreach(outgoing(G,i), \alam{j}{\ablock{
      \acomment{check where we can go next after going to $i$}\\
      c := cost[i] + weight(i,j) \acomment{cost if we go to $j$ via $i$}\\
      \aif[cheaper than best known path]{c < cost[j]}{
        cheapest[j] := append(cheapest[i], j) \acomment{use the path via $i$}\\
        cost[j] := c
      }
    }}\\)
  }\\
  cheapest
}
\end{acode}

\paragraph{Correctness}
Termination is straightforward using the termination order $length(rest)$.
\medskip

Partial correctness is harder.
We can easily establish the loop invariant that $cheapest[j]$ (if not $null$) is a path from $start$ to $j$ whose cost is $cost[j]$.
That is easy to see.

Moreover, with a little thinking we see that for every node that is reachable from $start$, $cheapest[j]$ will not be $null$ at the end.

It remains to show that these paths are in fact the cheapest paths.
The following additional loop invariant is the key:
\begin{compactitem}
 \item for nodes $j\nin rest$: $cheapest[j]$ is a cheapest path from $start$ to $j$
 \item for nodes $j\in rest$: $cheapest[j]$ is cheapest among those paths from $start$ to $j$ whose intermediate nodes are not in $rest$.
\end{compactitem} 
Because $rest$ is empty at the end, this yields the desired result.

Because $rest$ contains all nodes initially, the invariant trivially holds before the loop.

It remains to show that it is indeed a loop invariant.

\paragraph{Complexity}
It is easy to see that everything outside the while-loop takes $\Theta(|N|)$.

The while-loop is run $|N|$ times.
But it is not so obvious how long the body of the while-loop takes:
\begin{compactitem}
 \item The algorithm does not clarify how to find $i$. To find it efficiently, we can use a priority queue, which takes at most $O(\log|rest|)\sq O(\log|N|)$.\\
 So the first part of the while-loop takes $|N|O(\log|N|)$.
 \item We need a special priority queue where we change the priorities (the values $cost[i]$) from time to time.
  We know that takes at most $O(\log |rest|)\sq O(\log|N|)$ each time.
 \item The body of the $foreach$ command is run once for every outgoing edge of $i$.
 It is unclear how many such edges there are.
 But we can merge the analysis with the while-loop: Overall the body of the $foreach$ is run once for every outgoing edge of every node, i.e., once for every edge.\\
 Every time we have to allow for $O(\log|N|)$ because we may change a priority.\\
 Thus, the second part of the while-loop takes $|E|O(\log|N|)$ overall.
\end{compactitem}

Adding everything up, we obtain $O((|N|+|E|)\log|N|)$ as an upper bound for the worst case time complexity.
The average time complexity is lower because of the if-statement: we do not have to change a priority every time.

The worst-case time complexity becomes lower if we use better-optimized data structures for the priority queue.


\paragraph{Other Algorithms}
There are also important algorithms for finding all cheapest paths between any nodes at once.
For example, the Floyd-Warshall algorithm.

\subsection{Greatest Flow}\label{sec:ad:maximalflow}

In this section, we interpret all edge weights as capacities.
We assume that $G=(N,E)$ has two distinguished nodes $source$ and $sink$.
Such a graph is called a \textbf{flow network}.

Intuitively, we think of the edges as pipes, and our goal is to make a liquid flow from the source to the sink.

\paragraph{Flows}
Intuitively, a flow says how much is flowing along an edge.

A flow from $source$ to $sink$ maps every $e\in E$ to a number $f(e)\in \{0,\ldots,weight(e)\}$ such that
\begin{compactitem}
 \item Nothing flows into the start node: $f(e)=0$ for incoming edges of $source$.
 \item Nothing flows out of the end node: $f(e)=0$ for outgoing edges of $sink$.
 \item What flows into a node must flow out of it and vice versa: for all nodes $n$ except for $source$ and $sink$
  \[\Sigma_{e\in incoming(n)} f(e) = \Sigma_{e\in outgoing(n)} f(e)\]
\end{compactitem}

From that we can prove that
 \[\Sigma_{e\in outgoing(source)} f(e) = \Sigma_{e\in incoming(sink)} f(e)\]
and that number is called the capacity $Cap(f)$ of $f$.
It describes how much is flowing through the network from the source to the sink.

\paragraph{Problem}
Now our goal is to find $f$ such $Cap(f)$ is as big as possible.
This is called the greatest flow, maximal flow, or max-flow problem.

For example, the flow network describes goods that have to be shipped from one place to another via a variety of paths, the maximal flow maximizes the amount of goods that are shipped.
Similar examples abound in logistics.

\paragraph{Correspondence between Paths and Flows}
Recall that paths from $start$ to $end$ are lists of nodes $[p_0,\ldots,p_n]$ such that
\begin{compactitem}
 \item The path begins at $start$: $p_0=e$.
 \item The path ends at $end$: $p_n=end$.
 \item There are edges that connect each node to is successor: for each $i=0,\ldots,n-1$, there is an edge from $p_i$ to $p_i+1$.
\end{compactitem}

Then the definitions of paths from $start$ to $end$ and flows from $start$ to $end$ are elegantly similar.
Moreover, the cost of a path corresponds to the capacity of path, and the cheapest path problem corresponds to the greatest flow problem.

\paragraph{Algorithm}
A well-known algorithm for the greatest flow is the Ford-Fulkerson algorithm.
It is also a greedy algorithm.

The basic idea is to
\begin{compactitem}
\item start with the empty flow where $f(e)=0$ for all edges $e$,
\item repeatedly look for some path from $source$ to $sink$ along which every edge still has free capacity and increase the flow accordingly.
\end{compactitem}
However, the algorithm also uses a subtle trick to allow for reducing the flow along some edge.

That has the effect of adjusting sub-optimal choices from previous iterations.

We omit the details.

\chapter{Function-Like Data Structures}\label{sec:ad:functions}
 \input{function}

\chapter{Product-Like Data Structures}\label{sec:ad:products}
 \input{product}

\chapter{Union-Like Data Structures}\label{sec:ad:unions}
 \input{union}

\chapter{Algebraic Data Structures}\label{sec:ad:theories}
 \section{Specification}

Algebraic data types are a not clearly delineated family of record types.
Typically one field of the record is a type, and the other fields are operations on that type.

Many of them represent mathematical theories, in which case their elements represent mathematical structures.
Therefore, they tend to come up a lot.
But their high level of abstraction leads to them often being neglected or not understood.

Three classes are of major importance: these are the data types based on one type $U$ and
\begin{itemize}
\item one binary relation $U\times U \to \Bool$ such as in graphs, preorders, orders, and equivalence relations,
\item one binary function $U\times U\to U$ such as in monoids, groups, and semi-lattices,
\item two binary functions $U\times U\to U$ such as in rings, fields, and lattices
\end{itemize}
and several axioms such as transitivity, associativity, or distributivity.

The most important special cases are specified in Sect.~\ref{sec:math:binrel} for a binary relation and in Sect.~\ref{sec:math:binop} for one binary function.
We omit the case for two binary functions.

\section{Data Structures}

Apart from the axioms, we can implement algebraic data types very well as polymorphic abstract classes that take $U$ as the type parameter.
However, the axioms can usually not be implemented elegantly (except in very advanced programming languages) and must be realized as comments.

\subsection{One Binary Relation}

We already implemented some of them when sorting lists.

For example, the type of orders on $U$ can be realized as
\begin{acode}
\aclassA{Relation[A]}{}{}{
  \afun[\Bool]{rel}{x:A,y:A}{\ldots}
}\\
\aclassA{Preorder[A]}{}{Relation[A]}{
  \acomment{$rel$ is reflexive and transitive}
}\\
\aclassA{Order[A]}{}{Preorder[A]}{
  \acomment{$rel$ is anti-symmetric}
}\\
\aclassA{TotalOrder[A]}{}{Order[A]}{
  \acomment{$rel$ is total}
}
\end{acode}

Individual structures can now be implemented as concrete classes that implement the abstract ones.
We have already implemented some concrete orders such as $\leq$ on $\Int$, $|$ on $\Int$, or the lexicographic ordering on $\String$.

\subsection{One Binary Function}

For example, the type of monoids on $U$ can be realized as
\begin{acode}
\aclassA{BinOp[A]}{}{}{
  \afun[A]{op}{x:A,y:A}{\ldots}
}\\
\aclassA{Monoid[A]}{}{BinOp[A]}{
  \afun[A]{e}{}{\ldots}\\
  \acomment{$op$ is associative and has neutral element $e$}
}
\end{acode}

Individual structures can now be implemented as concrete classes that implement the abstract ones.

The following implements some example monoids:
\begin{acode}
\aclass{Addition}{}{Monoid[\Int]}{
  \afunI{op}{x:\Int,y:\Int}{x+y}\\
  \afunI[A]{e}{}{0}
}
\end{acode}

\begin{acode}
\aclass{Multiplication}{}{Monoid[\Int]}{
  \afunI{op}{x:\Int,y:\Int}{x*y}\\
  \afunI[A]{e}{}{1}
}
\end{acode}

\begin{acode}
\aclass{Concatenation}{}{Monoid[\String]}{
  \afunI{op}{x:\String,y:\String}{x+y}\\
  \afunI[A]{e}{}{""}
}
\end{acode}

\begin{acode}
\aclass{Maximum}{}{Monoid[\N]}{
  \afunI{op}{x:\N,y:\N}{\aifelseI{x\leq y}{y}{x}}\\
  \afunI[A]{e}{}{0}
}
\end{acode}
In each case, we have to check that the axioms (associativity and neutrality) actually hold to show the correctness of the implementations.

For a more complex example, consider the monoid of $2\times 2$ matrices under multiplication:
\begin{acode}
\aclass{Matrix22}{a:\Int,b:\Int,c:\Int,d:\Int}{}{}
\\
\aclass{Matrix22Multiplication}{}{Monoid[Matrix22]}{
  \afunI{op}{x:\N,y:\N}{\ldots}\\
  \afunI[A]{e}{}{\anew{Matrix22}{1,0,0,1}}
}
\end{acode}
where $\anew{Mat}{a,b,c,d}$ represents the matrix $\begin{pmatrix}a & b \\ c & d\end{pmatrix}$.

\subsection{Two Binary Functions}

The types of, e.g., rings and fields are implemented accordingly.

\section{Important Algorithms}

The implementation of algebraic data types and algorithms that
\begin{compactitem}
 \item are used in concrete algebraic structures
 \item compute new algebraic structures from existing ones
\end{compactitem}
are studied in the field of \emph{computer algebra}.
This is essential to for mathematical computation.
\medskip

Additionally, many algorithms can be easily generalized to take an arbitrary structure as their input.
For example, sorting can be implemented as a function \[sort(x:List[\Int]):List[\Int].\]
But it is much more general, equally easy, and essentially as fast to implement a function \[sort[A](ord: TotOrd[A], x:List[A]):List[A]\] that sorts with respect to an arbitrary total order.

We give some examples.

\subsection{Folding Lists over a Monoid}\label{sec:ad:monoidfold}

The function \[fold[A,B](x:List[A], f: A\times B\to B): B\] for lists is important but often confusing.

It becomes much simpler if we consider the special case $A=B$:
\[fold[A](x:List[A], e: A, f: A\times A\to A): A\]
If we additionally write $f$ as infix, i.e., write $x\,f\,y$ instead of $f(x,y)$, we have
\[fold[A]([a_1,\ldots,a_n],e,f)=a_1\,f\,(a_2\,\ldots\,(a_{n-1}\,f\,(a_n\,f\, e))\ldots)\]

In general, the bracketing matters.
But if $f$ is associative, the bracketing becomes irrelevant.
In that case, we usually want $e$ to be a neutral element for $f$.

That yields
\[monoidFold[A](mon:Monoid[A], x:List[A])\]
\[monoidFold[A](mon, [a_1,\ldots,a_n])=a_1\,mon.op\,a_2\,\ldots\,a_{n-1}\,mon.op\,a_n\,mon.op\, mon.e\]
In particular, we have
\[monoidFold[A](mon, [])=mon.e \tb\mand\tb monoidFold[A](mon, [a])=a.\]

For example,
\begin{compactitem}
\item $monoidFold[\Int](\anew{Addition}{}, x)$ is the sum of all elements in $x$,
\item $monoidFold[\Int](\anew{Multiplication}{}, x)$ is the product of all elements in $x$,
\item $monoidFold[\N](\anew{Maximum}{}, x)$ is the greatest element in $x$,
\item $monoidFold[\String](\anew{Concatenation}{}, x)$ is the concatenation of all strings in $x$.
\end{compactitem}

\subsection{Square-and-Multiply}

We can finally give the square-and-multiply algorithm from Sect.~\ref{sec:ad:exp:sqmult} in full generality.

This should be a function
\begin{acode}
\afun[A]{sqmult[A]}{mon:Monoid[A], x:A, n: \N}{\ldots}
\end{acode}
that satisfies the specification
\[sqmult(mon, x, 0)=mon.e\]
\[sqmult(mon, x, n+1)=mon.op(power(mon, x, n), x)\]
and whose time complexity is $\Theta(\log_2 n)$.

% matrix multiplication for any semi-ring? use for flow?


% % % % % % % % % % % % % % % % % % % %
\part{Important Families of Algorithms}\label{sec:ad:algo}

\chapter{Recursion}\label{sec:ad:recurse}
 We speak of recursion if a function calls itself.
More generally, a set of functions is recursive if they call each other in a cyclic way.

A recursive algorithm is a good design choice if we can reduce the problem $P(x)$ to the problem $P(x')$ such that $x'$ is (in some sense) smaller than $x$.
Usually, $x'$ is just a little smaller than $x$, e.g., we might have $x\in\N$ and $x'=x-1$.

Recursive algorithms have been used in multiple places throughout these notes.
In the following, we look at some frequent patterns of recursively algorithms.

Here we only consider the case where $P(x)$ is reduced to slightly smaller instances, e.g., $P(x-1)$.
There are other patterns, which we discuss in separate chapters, that may or may not be implemented using recursion.
The following gives an overview:
\begin{compactitem}
  \item backtracking: $P(x)$ is reduced to some smaller instance, but we have to try multiple instances before finding the right one.
  \item divide-and-conquer: $P(x)$ is reduced to multiple much smaller instances, e.g., $P(x/2)$.
  \item dynamic programming: $P(x)$ is computed by first computing all smaller instances, e.g., $P(0),\ldots,P(x-1)$.
  \item parallelization: $P(x)$ is reduced to multiple smaller instances, whose solutions are computed in parallel.
\end{compactitem}

\section{Induction}

Induction is the special case of recursion that comes up in conjunction with inductive data types.

For every inductive, a whole range of relevant It is used to define most operations  such as natural numbers or lists.
We consider some examples.

\subsection{Natural Numbers}

Consider the data type of natural numbers:
\begin{acode}
\adata{nat}{zero,succ(pred: nat)}
\end{acode}

An inductive algorithm for $P(n:nat)$ employs a case split on $n$, usually using pattern-matching:
\begin{compactitem}
 \item if $n=zero$: return $P(zero)$ directly without recursion (the base case),
 \item if $n=succ(n')$: recursively compute $y=P(n')$, then use $y$ to compute $P(succ(n'))$ without further recursion (step case).
\end{compactitem}

As an example, consider the inductive algorithm for the factorial on the left:
\newpage
\begin{multicols}{2}
\begin{acode}
\afun{fact}{n:nat}{
  \amatch{n}{\acase{zero}{1}, \acase{succ(n')}{\ablock{y:=fact(n')\\ y\cdot succ(n')}}}
}
\end{acode}
\columnbreak
\begin{acode}
\afun{fact}{n:\Int}{
  \aifelse{n==0}{1}{y:=fact(n-1)\\ y\cdot n}
}
\end{acode}
\end{multicols}

Essentially the same algorithms can also be used if $nat$ is not defined as an inductive type.
Instead of pattern-matching, this requires an if-statement.
For example, if we use $\Int$ to represent natural numbers, we obtain the algorithm on the right above.
\medskip

Virtually all arithmetic operations can be elegantly defined as recursive algorithms along these lines.

\subsection{Lists}

Consider the inductive type of lists:
\begin{acode}
\adata{IndList[A]}{nil,{cons(head:A,tail:IndList[A])}}
\end{acode}

It is slightly more complicated than $nat$.
But an inductive algorithm for $P(x:IndList[A])$ proceeds in essentially the same way:
\begin{compactitem}
 \item if $x=nil$: return $P(nil)$ directly without recursion (the base case),
 \item if $x=cons(hd,tl)$: recursively compute $y:=P(tl)$, then use $hd$ and $y$ to compute $P(cons(hd,tl))$ without further recursion (step case).
\end{compactitem}

Some examples were already given elsewhere in these notes.
The simplest example is the $length$ function:
\begin{acode}
\afun[\Int]{length}{x:IndList[A]}{\amatch{x}{\acase{nil}{0},\acase{cons(hd,tl)}{\ablock{y:=length(tl)\\y+1}}}}
\end{acode}
Here the $cons$-case ignores the value $hd$.

Like for natural numbers, the corresponding recursive algorithms can also be used if lists are not defined as an inductive type.

\subsection{Binary Trees}

Full binary trees are often defined as the following inductive type.
If all nodes are labeled with values from $A$, we obtain
\begin{acode}
\adata{BinTree[A]}{Leaf(label: A),{Node(label: A, left:BinTree[A], right:BinTree[A])}}
\end{acode}

It is more complicated than $IndList[A]$ because it uses two inductive arguments.
But an inductive algorithm for $P(x:BinTree[A])$ proceeds in essentially the same way:
\begin{compactitem}
 \item if $x=Leaf(a)$: compute the result from $a$ directly without recursion,
 \item if $x=Node(a,l,r)$: recursively compute $y:=P(l)$ and $y'=P(r)$, then use $a$, $y$, and $y'$  without further recursion.
\end{compactitem}

As an example, consider a function that returns the list of labels in DFS-order:
\begin{acode}
\afun[{List[A]}]{labels}{x:BinTree[A]}{
  \amatch{x}{\acase{Leaf(a)}{{cons(a,nil)}},\acase{Node(a,l,r)}{\ablock{y:=labels(l) \\ y':=labels(r)\\ cons(a, concat(y,y'))}}}
}
\end{acode}
Here the $cons$-case ignores the value $hd$.

Like for other types, the corresponding recursive algorithms can also be used if trees are not defined as an inductive type.

% min-max example?

\section{Mutual Recursion}

We speak of \emph{mutual recursion} if multiple functions call each other.

A very simple non-trivial example is the $even$-$odd$ recursion:
\begin{acode}
\afun[\Bool]{even}{n:\N}{
  \aifelse{n==0}{\true}{odd(n-1)}
}\\
\\
\afun[\Bool]{odd}{n:\N}{
  \aifelse{n==0}{\false}{even(n-1)}
}
\end{acode}

Of course, that is very inefficient for large $n$ and only makes sense if the $n\modop 2$ function is not available.

Termination arguments for sets of mutually recursive functions are more difficult than for single recursive functions.
But the basic idea is the same: every recursive call should make the argument smaller in some sense.

\section{Recursion vs. While-Loops}\label{sec:ad:recurse:while}

Technically, recursion is redundant if the programming language allows for while-loops.
But recursion is such a versatile concept that it is part of every practical programming language.

The following example shows how a while-loop can be systematically replaced with a recursive function:

\begin{multicols}{2}
\begin{acode}
x:=x_0\\
y:=y_0\\
z:=z_0\\
\awhile{C(x,y,z)}{
  (x,y,z):=Body(x,y,z)\\
}
\end{acode}
\columnbreak
\begin{acode}
\afun{f}{x,y,z}{
  \aifelse{C(x,y,z)}{
    (x',y',z'):=Body(x,y,z)\\
    f(x',y',z')
  }{(x,y,z)}
}\\
f(x_0,y_0,z_0)
\end{acode}
\end{multicols}

On the left, $x$, $y$, and $z$ are the variables, whose values may change during an iteration of the loop, and $C(x,y,z)$ is the loop condition.
$Body$ is some piece of code that may use the current values of $x$, $y$, and $z$, and assign new values to them.
This is indicated in the line $(x',y',z'):=Body(x,y,z)$, which treat $Body$ as a function that takes the old values, performs arbitrary additional operations, and eventually returns the new values of the variables.

The right side shows the equivalent recursive function.

Above we use three variables $x$, $y$, and $z$.
Any other number works accordingly---the only critical aspect is that all variables whose value may be changed by $Body$ must become argument of the recursive function.

The example below uses two mutable variables $result$ and $i$.
It computes the factorial of $n$ in the variable $result$.
$C(result,i)$ is the formula $i>0$.

\newpage
\begin{multicols}{2}
\begin{acode}
result:=1\\
i:=n\\
\awhile{i > 0}{
  (result,i):=(result\cdot i, i-1)
}
\end{acode}
\columnbreak
\begin{acode}
\afun{f}{result,i}{
  \aifelse{i>0}{
    (result',i'):=(result\cdot i, i-1)\\
    f(result',i')
  }{(result,i)}
}\\
f(1,n)
\end{acode}
\end{multicols}

The example may look a bit odd because $Body$ is written in a way that emphasizes the correspondence between while-loops and recursions.
If we rewrite them separately into more familiar styles, we obtain

\begin{multicols}{2}
\begin{acode}
result:=1\\
i:=n\\
\awhile{i > 0}{
  result := result\cdot i\\
  i:=i-1
}
\end{acode}
\columnbreak
\begin{acode}
\afun{f}{result,i}{
  \aifelse{i>0}{
    f(result\cdot i,i-1)
  }{result}
}\\
f(1,n)
\end{acode}
\end{multicols}

Note that the recursive functions that correspond to while-loops have some special properties:
\begin{compactitem}
 \item There is exactly one recursive call, and no other code is executed after it.
 \item The return value (if any) is not used for further computations.
 \item The base case returns all function arguments.
  In order to return anything useful, at least one of the arguments must have the role of collecting the result.
  In the example, the argument $result$ has that role: it is not used in $f$ except for building the result; therefore, the base case can simply return $result$.
\end{compactitem}

Thus, while-loops correspond to a very small class of recursive functions.
It is also possible to translate \emph{any} recursive function (even any set of mutually recursive functions) into a while-loops.
But the translation is a bit more complicated, and the two translations are not inverse to each other.

\section{Tail-Recursion}\label{sec:ad:recurse:tail}

\subsection{Tail-Recursive Functions}

In Sect.~\ref{sec:ad:recurse:while}, we encountered a class of recursive functions with special properties.

Concretely, a recursive function $f$ is called \textbf{tail-recursive}, if all its recursive calls occur in positions where the result of the recursive call directly becomes the result of the $f$.

For example, the following is a tail-recursive implementation of the factorial of $n$:
\begin{acode}
\afun{f}{result,i}{
  \aifelse{i>0}{
    f(result\cdot i,i-1)
  }{result}
}\\
f(1,n)
\end{acode}

\subsection{Optimization}

If $f$ is tail-recursive, a interpreter/compiler may turn $f$ into the corresponding while-loop, which usually yields much more efficient code.
Note that this is one of the rare situations, where the complexity of an algorithm depends on the interpreter/compiler.
For example, Java does not do it; many C compilers do.

The reason for the efficiency gain is the following.
Both the recursive function and the while-loop must pass information from one iteration to the next.
The while-loop uses assignments to mutable variables; because the mutable variables reside in the same memory location for the current and the next iteration, no physical passing of data is needed.
But in a recursive function, the recursive function call (like any other function call) allocates new memory locations and then copies the function arguments into them.
This overhead does not change the $\Theta$-class but can still be substantial.

Moreover, many interpreters/compilers allocate only a fixed amount of memory for the stack (see Sect.~\ref{sec:ad:callstack}).
For large function arguments, a recursive function may create so many nested function calls that it exhausts the available stack space (causing a \emph{stack overflow error}).
Tail-recursive functions are a special case where this danger can be averted by optimizing them into while-loops.

The exact way in which tail-recursion optimization happens is up to the interpreter/compiler.
Usually no new stack frame is allocated for the recursive call---instead the current stack frame is reused.
That works out because---due the call being tail-recursive---the variable values stored in the current stack frame will never be used again and can therefore be safely overridden.

\subsection{The Call Stack and Stack Frames}

Most interpreters/compilers use a stack to keep track of the nesting of function calls.
This data structure is usually called the \textbf{call stack} (or just \emph{the stack}), and its elements are called \textbf{stack frames}.

When a function call $f(t_1,\ldots,t_n)$ of a function $f(x_1,\ldots,x_n)$ is processed,
\begin{compactitem}
 \item a new stack frame is created containing at least
   \begin{compactitem}
     \item the variable definitions $x_1:=t_1, \ldots, x_n:=t_n$
     \item the current program counter (i.e., the position of the next statement to be executed)
   \end{compactitem}
 \item the frame is pushed onto the stack
 \item execution continues with the body of $f$.
\end{compactitem}

When $f$ returns, the stack frame of $f$ (which is at the top of the call stack now) is popped from the stack, and execution continues at the position where $f$ was called.

\chapter{Backtracking}\label{sec:ad:backtrack}
 \section{Overview}

Consider a problem that involves multiple successive choices.
A backtracking algorithm always chooses one of the options and proceeds.
If that leads to a situation where no further progress is possible, the algorithm reverts all steps since the most recent choice and chooses a different option.
The reversal is called \textbf{backtracking}.

The typical example is finding the exit of a maze: at any given intersection, we have to choose one of the possible paths.
If we ever reach a dead end, we backtrack all steps since the most recent intersection and choose a different path.

Backtracking algorithms make sense if
\begin{compactitem}
 \item we have little or no information to predict the best choice right away,
 \item after making a choice, we can detect quickly that it was the wrong choice.
\end{compactitem}
If the first condition is not met, we can usually do better by finding the best choice instead of trying out a random choice.
If the second condition is not met, the backtracking algorithm degenerates into DFS.

For the maze example, we cannot detect early on that a path leads to a dead end.
Therefore, we have to walk all the way to the dead end before we can backtrack.
Thus, we end up performing a DFS of the maze (seen as a graph whose nodes are the intersections).

\section{General Structure}

Consider a problem whose solution involves a series of choices among finitely many options.
We can represent this as finding a branch in a tree:
\begin{compactitem}
 \item the root is the starting point,
 \item at every node $n$, we have to choose one out of the children of $n$,
 \item at a leaf, we can test whether we have a solution or not.
\end{compactitem}

Thus, we obtain the following general problem: given a tree $T:Tree[A]$ and a property $solution:List[A]\to\Bool$, find a branch $b$ of $T$ such that $P(b)$.

Backtracking works well if we additionally have a test $abort:List[A]\to\Bool$ such that $abort(b)$ implies that there is no solution that starts with $b$.

To be efficient, we avoid ever building the entire tree $T$---otherwise, we would waste the cost-saving effect of $abort$.
Then we can give a general backtracking algorithm as
\begin{acode}
\afun[{Option[List[A]]}]{search}{state:List[A]}{
  \aifI{abort(state)}{\areturn{None}}\\
  \aifI{solution(state)}{\areturn{Some(state)}}\\
  foreach(choices(state), \alam{c}{\ablock{
    x := search(state+[c])\\
    \aifI{x \neq None}{\areturn{x}}
   }}\\)\\
   \areturn{None}
}
\end{acode}

Here the argument $state$ represents the path from the root to the current node, and $choices(state)$ yields its children.
The algorithm is essentially a DFS-algorithm in the tree $T$ that
\begin{compactitem}
 \item never builds $T$ as a whole,
 \item stops as soon as the first solution has been found,
 \item uses $abort$ to avoid traversing a subtree.
\end{compactitem}

\begin{example}
A standard example is the $8$-queens problem: place $8$ queens on a chess-board such that none threatens the other.

Clearly, there has to be exactly one queen in each row.
Thus, we can represent every solution as a list $[c_1,\ldots,c_8]$ such that $c_i$ is the column coordinate of the queen in row $i$.

The possible choices are defined by
\[choices(state) \cas{[1,2,3,4,5,6,7,8] \mifc length(state)<8 \\ [] \mifc length(state)=8}\]
i.e., in every row we have to choose one out of $8$ columns, and we reach possible solutions after making $8$ choices.

The function $abort(state)$ return $\true$ if any two queens threaten each other.
Because this can be tested efficiently already if $length(state)<8$, we can use it to backtrack early.
\end{example}

\section{Constraint Satisfaction Problems}

A constraint satisfaction problem (CSP) consists of a set $x_1:A_1,\ldots,x_m:A_m$ of variables and a set $C_1(\vec{x}),\ldots,C_n(\vec{x	})$ of boolean expressions (called the \textbf{constraints}) about the variables.

A typical example is a system of $n$ linear equations in $m$ variables.
But it is not required that the constraints are equations.

Many practical problems can be captured as CSPs.
This includes many games and puzzles:

\begin{example}[Sudoku]
The Sudoku game is a CSP with $m=81$ variables $x_{ij}\in\{1,\ldots,9\}$ for $i,j=1,\ldots,9$.
It has at least the $27$ constraints of the form $\{x_{i1},\ldots,x_{i9}\}=\{1,\ldots,9\}$ for each row $i$ and correspondingly for each column and each of the $9$ $3x3$ squares.

Additionally, each instance of Sudoku has a few constraints of the form $x_{ij}=c$, which fix the values of certain variables.
\end{example}

Some CSPs can be solved efficiently using backtracking algorithms.
Step $i$ tries to fill in a value for $x_i$.
After $i$ steps $state$ is the list $[v_1,\ldots,v_i]$ of currently tried values for $x_1,\ldots,x_i$.
$choices(state)=A_{i+1}$ returns the possible values for the next variable $x_{i+1}$.
$abort(state)$ checks if there is some $C_j$ such that $C_j(v_1,\ldots,v_i,x_{i+1},\ldots,x_n)$ is false no matter which values are used for $x_{i+1},\ldots,x_n$.
$solution(state)$ evaluates the conjunction of all constraints.

\begin{example}[SAT]
The SAT problem is the special case of CSP where $A_i=\Bool$ for all $i$ and all $C_i$ are boolean expressions (i.e., using only $\neg$, $\wedge$, and $\vee$).
Because there are only $2$ choices for each variable and boolean expressions can be evaluated very easily, it can be solved relatively quickly using a backtracking algorithm.

However, even with a good $abort$ function, the fastest known SAT algorithms are exponential.
\end{example}



\chapter{Divide and Conquer}\label{sec:ad:divcon}
 \section{Overview}

A divide-and-conquer algorithm consists of $3$ steps:
\begin{compactenum}
 \item the input is divided into parts,
 \item each part is processed recursively,
 \item the results of the parts are combined into the result of the whole.
\end{compactenum}
Divide-and-conquer works whenever a problem of size $n$ can be reduced to multiple smaller subproblems of the same kind.

The simplest special case arises when working on a list that is divided into two parts.
$mergesort$ from Sect.~\ref{sec:ad:sort:merge} is a typical example.
Here the $merge$ function combines the partial results into the result of the whole.

A more difficult example is Strassen's multiplication algorithm from Sect.~\ref{sec:ad:matrix:strassen}.
It splits the arguments matrices into $4$ parts each and recurses $7$ times.
Both the inputs of the recursions and the combination of the results are obtained by substantial computations.

\section{General Structure}

\subsection{Design}

Consider a problem with a size parameter $n\in \Z$ (e.g., the length of a list or the dimension of a square matrix).

The following is the rough general shape of a divide-and-conquer algorithm using two constants $d$ and $r$:
\begin{compactenum}
 \item argument/input: problem of size $n$
 \item if $n<d$, solve the problem directly
 \item otherwise:
 \begin{compactenum}
  \item divide: create $r$ subproblems of size $n/d$
  \item conquer: recursively solve the subproblems
  \item return/output: combine the solutions of the subproblems into the solution of the overall problem
 \end{compactenum}
\end{compactenum}

Here $d$ stands for \emph{division}: every iteration reduces the size of the problem from $n$ to $n/d$.
Technically, the size of the subproblems is $n \divop d$.
But we gloss over this problem here and assume that $n$ is a power of $d$, i.e., $n=d^k$.
All results of this section apply to the general case as well.

$r$ stands for \emph{recursion}: every iteration recurses $r$ times.
The value of $r$ varies between algorithms:
\begin{compactitem}
 \item Very often we have $r=d$, i.e., every subproblem is solved separately.
   For example, for mergesort $r=d=2$.
 \item But sometimes we do not have to recurse for every subproblem.
   A common case is $r=1$, i.e., we can identify a specific subproblem that is sufficient to solve.
   For example, for binary search, we know which sublist has the needed value and recurse into that one.
 \item In some algorithms, the subproblems are not \emph{parts} of the original problem.
   For example, in Strassen's algorithm, the smaller matrices that get multiplied recursively are not submatrices of the original matrix.
   Here we have $d=2$ (if we consider the dimension of the matrices to be the size of the problem) and $r=7$.
\end{compactitem}

\subsection{Correctness}

The correctness must be argued separately in each case.

However, this is usually easy by induction:
\begin{compactitem}
 \item check that the base cases are handled correctly
 \item assuming the recursive calls yield correct result, check that the divide-and-conquer steps are correct.
\end{compactitem}

\subsection{Complexity}

In many cases the time complexity of a divide-and-conquer algorithm can be obtained using a general method.
Let $C(n)$ be the time complexity for input of size $n$.

\paragraph{Substitution Method}
Let $div(n)$ and $combine(n)$ be the costs of dividing and combining, and let $f(n)=div(n)+combine(n)$.
Then we have
 \[C(n)=div(n) + r\cdot C(n/d) + combine(n)=r\cdot C(n/d) + f(n)\]

By recursively substituting this formula into itself, we obtain
 \[C(n)=r\cdot C(n/d) + f(n) = r(r\cdot C(n/d^2) + f(n/d)) + f(n)\]
\[= \ldots = r^k\cdot C(n/d^k) + r^{k-1}\cdot f(n/d^{k-1}) + \ldots + r\cdot f(n/d) + f(n)\]

Because the cost of the base cases does not depend on $n$, we have $C(n)\in O(1)$ for $n<d$.
Recalling that $n=d^k$, we get
\[C(n)=r^k\cdot O(1) + r^{k-1}\cdot f(n/d^{k-1}) + \ldots + r\cdot f(n/d) + f(n)\]

\paragraph{Tree Intuition}
We can visualize the execution of a divide-and-conquer algorithm as a tree:
\begin{compactitem}
 \item the root is the original call
 \item for each node, the children are the recursive calls
 \item the leafs are the base cases
\end{compactitem}
Then we get a tree of height $k$ in which every non-leaf node has $r$ children.
In total there are $r^i$ nodes at depth $i$ and in particular $r^k$ leafs.

The terms in the above complexity formula now have an intuitive interpretation:
\begin{compactitem}
\item The terms $r^i\cdot f(n/d^i)$ are the cost of dividing and combining in the $r^i$ nodes at depth $i$, which process problems of size $n/d^i$.
\item In particular, the term $f(n)=r^0\cdot f(n/d^0)$ is the cost of dividing and combining at the root.
\item The term $r^k\cdot O(1)$ is the cost of base cases at the $r^k$ leafs.
\end{compactitem}
Thus, the conquering cost at each node is given by its subtrees.
The total cost of each node is obtained by summing the values in its subtree, and the overall cost of executing the algorithm is the sum of the costs at all nodes.

\paragraph{Mathematical Preliminaries}
Before we continue we recall two general formulas that we will use:
\begin{compactitem}
 \item the geometric series for $q\neq 1$
  \[\sum_{i=0}^{k-1} q^i=\frac{1-q^k}{1-q}\]
 \item the following logarithm swap
  \[x^{\log_y z}=z^{\log_y x}\]
\end{compactitem}

\paragraph{Solving the Formula}
It remains to simplify the clunky formula
\[C(n)=r^k\cdot O(1) + r^{k-1}\cdot f(n/d^{k-1}) + \ldots + r\cdot f(n/d) + f(n)\]
into something nicer.

Clearly that depends on $f$.
Let us assume that $f$ is polynomial, i.e., $f\in\Theta(n^c)$ for some $c>0$.
(If $f$ is super-polynomial, the algorithm is probably not very useful anyway.)

Then we get
 \[C(n)\in r^k\cdot O(1) + r^{k-1}\cdot \Theta(n^c/d^{c(k-1)}) + \ldots + r\cdot \Theta(n^c/d^c) + \Theta(n^c)\]
 \[=\Theta\left(r^k + n^c\cdot \sum_{i=0}^{k-1}\left(\frac{r}{d^c}\right)^i\right)\] 
After abbreviating $q=r/d^c$, assuming $q\neq 1$ (We treat the case $q=1$ below.) and applying the geometric series, and recalling that $k=\log_d n$, that becomes
 \[=\Theta\left(r^{\log_d n} + n^c\cdot \frac{1-q^{\log_d n}}{1-q}\right)\]
After applying the logarithm swap twice and using $\log_d q=\log_d r - \log_d d^c$, that becomes
 \[=\Theta\left(n^{\log_d r} + \frac{n^c-n^{c+\log_d q}}{1-q}\right)=\Theta\left(n^{\log_d r} + \frac{n^c-n^{\log_d r}}{1-q}\right)\]
Now we want to drop the constant factor $1-q$ from inside $\Theta$.
But that is only allowed if $1-q>0$.

So we eventually have to distinguish three cases:
\begin{itemize}
\item $r=d^c$ and thus $q=1$ and $\log_d r=c$.\\
Then we cannot apply the formula for the geometric series.
Instead the $\Sigma$-sum reduces to $k-1=\log_d n -1$.
Overall we get
\[C(n)\in\Theta\left(n^{\log_d r} + n^c(\log_d n - 1)\right)=\Theta(n^c\log_d n)\]
\item $r<d^c$ and thus $0<q<1$ and $\log_d r < c$.\\
Then we can drop $1-q>0$ from inside $\Theta$, and we obtain
\[C(n)\in \Theta(n^c)\]
\item $r>d^c$ and thus $q>1$ and $\log_d r > c$.\\
 Then $1-q$ is negative and we can drop it only if we also flip the sign. That yields
 \[C(n) \in \Theta\left(n^{\log_d r}-n^c+n^{\log_d r}\right)=\Theta\left(n^{\log_d r}\right)\]
\end{itemize}

The so-called Master theorem collects these three cases into a handy cheat sheet:

\begin{theorem}[Master theorem]
For the time complexity $C(n)$ of a divide-and-conquer algorithm that
\begin{compactitem}
 \item requires $f(n)\in\Theta(n^c)$ time for dividing and combining
 \item recurses into $r$ subproblems of size $n/d$ whenever $n\geq d$
\end{compactitem}
we have
\begin{center}
\begin{tabular}{ll}
if $r<d^c$: & $C(n)\in\Theta(n^c)$ \\
if $r=d^c$: & $C(n)\in\Theta(n^c\log_d n)$ \\
if $r>d^c$: & $C(n)\in\Theta(n^{\log_d r})=\Theta(r^{\log_d n})$ \\
\end{tabular}
\end{center}
\end{theorem}

The theorem holds independent of our assumption that even if $n$ is not a power of $d$.

Not surprisingly, all cases require at least $\Theta(n^c)$, which is already the cost of dividing and combining at the root.\\
If $c>\log_d r$, that is the entire cost, and the cost of the recursions and the base cases can be neglected.\\
If $c=\log_d r$, the recursion cost yields a logarithmic factor corresponding to the depth of the recursion.\\
If $c<\log_d r$, the cost of dividing and combining can be neglected and the cost of the $r^{\log_d n}$ base cases dominates the overall cost.

\section{Examples}

\subsection{Mergesort}

For mergeseort from Sect.~\ref{sec:ad:sort:merge}, we have $r=d=2$.
Dividing and combining requires linear time, i.e., $c=1$.

The Master theorem indeed yields $\Theta(n\log_2 n)$ for the time complexity of mergesort.

\subsection{Binary Search}

Binary search checks whether a sorted list $x$ of length $n$ contains the value $a$.
The algorithm uses $d=2$ and $r=1$.

The base case for $n<d$ is easy.
The divide steps splits $x$ into two parts of length $n/2$.
The conquer step uses the property of being sorted to determine whether $x$ is the lower or the upper half and recurses only for that one.
No combination of results is needed.

The overall time complexity is $\Theta(\log_2 n)$.

\subsection{Karatsuba's Multiplication of Polynomials}

\subsubsection{Problem}

We want to multiply two polynomials $p(x)=p_mX^m+\ldots+p_1X+p_0$ and $q(x)=q_mX^m+\ldots+q_1X+q_0$ with integer coefficients $p_i,q_i\in \Z$.
The size $n$ of the problem is the number of coefficients per polynomial, i.e., $n=m+1$.

Without loss of generality, we assume that $n=2^k$, i.e., $m=2^k-1$.
(If the polynomials have a different degree, we can simply add $0$-coefficients to increase $m$.)

\subsubsection{Algorithm}

Karatsuba's divide-and-conquer algorithm uses $d=2$ and $r=3$.

\paragraph{Data Structure}
For the implementation, we need a data structure for polynomials.
The easiest choice is to use the list of coefficients.
So we assume that every polynomial is a list $[p_m,\ldots,p_0]\in List[\Z]$.

Addition/subtraction of two polynomials can be implemented easily as component-wise addition of the elements in the lists.

Note that two polynomials are equal if they only differ in initial $0$-coefficients.
We have two options:
\begin{compactitem}
\item forbid lists that start with $0$
\item make two polynomials equal if they only differ in initial $0$s
\end{compactitem}
Neither option is essential for the algorithm.
However, it is convenient to pick the second option: that way we can easily add initial $0$s to adjust the size of a polynomial.

\paragraph{Base Case}
If $n=0$, both polynomials are the $0$-polynomial, and their product is again the $0$-polynomial.
If $n=1$, both polynomials are integers, and we use plain integer multiplication: $pq=[p_0][q_0]=[p_0 q_0]$.

\paragraph{Idea}
To understand the key idea, let us first look at the special case $n=2$.
Then we have $pq=[p_1,p_0][q_1,1_0]=[p_1 q_1, p_1 q_0 + p_0 q_1, p_0 q_0]$.

The naive computation takes $4$ multiplications and $1$ addition.
But we can compute the result cleverly using only $3$ multiplications and $4$ additions:
\[pq=[a, b-a-c, c] \tb\mwhere a=p_1q_1, \;b=(p_1+p_0)(q_1+q_0),\;c=p_0 q_0\]
Because multiplication is much more complex than addition (addition is linear, multiplication is not), this is preferable even though we need additional additions.

\paragraph{Divide}
Let $n'=n/2=(m+1)/2$, i.e., $m=2n'-1$.

We split $p$ into two lists $p^u=[p_m,\ldots,p_{n'}]$ and $p^l=[p_{n'-1},\ldots,p_0]$ of length $n'$.
We split $q$ into $q^u$ and $q^l$ accordingly.\\

Now (as polynomials) $p(X)=p^u X^{n'}+p^l$ and $q(X)=q^u X^{n'}+q^l$.

\paragraph{Conquer}
We recursive multiply $3$ pairs of polynomials of size $n'$:
  \[a=p^u q^u \tb\tb b=(p^u+p^l)(q^u+q^l) \tb\tb c=p^l q^l\]

\paragraph{Combine}
We combine the results $a$, $b$, and $c$ as follows:
 \[pq=a X^{2n'} + (b-a-c)X^{n'} + c\]

\subsubsection{Correctness}

The correctness follows immediately from the construction of the algorithm.

We just have to check the mathematics of the divide and the combine step.

\subsubsection{Complexity}

This is left as an exercise.

% variant: n-bit int multiplication

\subsection{Associative Folding}\label{sec:ad:monoidfold:divide}

\subsubsection{Problem}

Consider folding over a monoid from Sect.~\ref{sec:ad:monoidfold}.

\subsubsection{Algorithm}

We give a divide-and-conquer algorithm for it using $d=r=2$.

\begin{acode}
\afun[A]{monoidFold[A]}{mon: Monoid[A], x: List[A]}{
 \aifelse{empty(x)}{mon.e}{
   n := length(x)\\
   i := n \divop 2\\
   lower := [x_0,\ldots,x_{i-1}]\\
   upper := [x_i,\ldots,x_{n-1}]\\
   lowerFold := monoidFold(mon,lower) \\
   upperFold := monoidFold(mon,upper) \\
   mon.op(lowerFold, upperFold)
 }
}
\end{acode}

\subsubsection{Correctness}

The correctness is straightforward.
The key insight is that associativity allows us to bracket the monoid operations any way we want, e.g.,
\[x_0 \;mon.op\;\ldots \;mon.op\; x_{i-1} \;mon.op\; x_i \;mon.op\; \ldots mon.op\; x_{n-1} =\]
\[(x_0 \;mon.op\; \ldots \;mon.op\; x_{i-1}) \;mon.op\; (x_i \;mon.op\; \ldots mon.op\; x_{n-1}) \]

\subsubsection{Complexity}

Let us assume that we use arrays to split the lists in constant time.
Combining only requires one monoid operation.
So the complexity of dividing and combining is in $O(1)$ and thus $c=0$.

Using $r=d=2$ and $c=0$, we have $r>d^c$ and $\log_d r=1$.
Thus, the Master theorem yields $C(n)\in\Theta(n)$.

Thus, the divide-and-conquer algorithm has the same complexity as the naive algorithm.
This is not surprising because all we do is change the bracketing.
The number of occurrences of $mon.op$ remains the same, i.e., we have to apply the monoid operation the same number of times.

So not every divide-and-conquer algorithm yields an improvement.

\chapter{Parallelization and Distribution}\label{sec:ad:parallel}
 \section{Concepts}

\paragraph{Multi-Threading}
Multi-threading executes multiple parts of a program at the same on the same machine.
These parts are called \textbf{threads}.

Ideally, this uses a separate CPU for each thread to maximize the speed-up.
But there can also be more threads running in parallel than there are CPUs.
In that case, the threads must be scheduled.

\paragraph{Speed-Up}
The speed up is the gain in time complexity if parallel execution is taken into account.

Formally, let $C^k(n)$ is the time complexity for input of size $n$ if $k$ CPUs are available for parallel execution.
In particular, $C^1(n)=C(n)$.
Then the speedup for $k$ machines is defined by $C^k(n)/C(n)$.

\paragraph{Blocking}
If multiple threads are present, a thread may block, i.e., it waits until a certain condition is fulfilled.
This gives a thread the ability to yield CPU access to other threads.

\paragraph{Fairness}
Usually there is no guarantee when or in which order threads gain access to a CPU.
In extreme cases, a thread may be finished completely before the next thread is started.

Fairness refers to a scheduling that gives each thread some guarantee on getting CPU access eventually.

Parallel algorithms are usually designed independent of fairness: they have to yield correct results no matter how the threads are scheduled.

\paragraph{Distributed Algorithm}
If the parts of a program are executed on completely different machines, we speak of a distributed algorithm.

\section{Parallelization Operators}

\subsection{Threads}

Most mainstream programming languages offer multi-threading.
The details vary a lot.

But typically a thread is a class with a method $run():\Unit$.
When $run$ is called, it returns immediately, and the program continues normally.
But from now on, the body of the method $run$ is executed in parallel with the remainder of the program.

\subsection{Parallel List Operations}

List-like data structures are great for parallelization because we often apply the same operation to all elements in a list.

We specify two new operations:

\begin{ctabular}{|l|l|l|}
\hline
function & returns & effect \\
\hline
\multicolumn{3}{|c|}{below, let $l\in A^*$ be of the form $[a_0,\ldots,a_{l-1}]$} \\
$parMap[B](l\in A^*, f:A\to B)\in B^*$ & nothing & $[f(a_0),\ldots,f(a_{l-1})]$ computed in parallel\\
$parForeach(l\in A^*, f:A\to \Unit)\in \Unit$ & $f(a_0),\ldots,f(a_{l-1}$ run in parallel & nothing\\
\hline
\end{ctabular}

\subsection{Parallel Composition}

Let $C$ and $D$ be functions that take no arguments.
Then $C|D$ is the command that runs $C()$ and $D()$ in parallel.

This can be seen as a special case of a $parForeach$ using the list $[C,D]$.

\section{Examples}

\subsection{Parallel Depth-First Search}

Often we can turn a normal algorithm into a parallel algorithm by replacing $map$ with $parMap$ or $foreach$ with $parForeach$.

For example, we obtain a parallel DFS traversal of a tree as follows:
\begin{acode}
\afun{DFS[A]}{n: Tree[A], f:Tree[A]\to \Unit}{
  f(n)\\
  parForeach(n.children, x \mapsto DFS[A](x,f))
}
\end{acode}

This executes $f(n)$ in parallel for every node $n$.
Of course, the actual order in which nodes are visited is not entirely predicable anymore.
It is still guaranteed that $f(n)$ terminates before $f(d)$ begins for any proper descendant $d$ of $n$.
But it is unpredictable in which order the children of $n$ are processed.


\subsection{Associative Folding}\label{sec:ad:monoidfold:parallel}

Consider the associative folding problem from Sect.~\ref{sec:ad:monoidfold:divide}.
The divide-and-conquer algorithm did not lower the complexity.

But we obtain a speedup if we recurse in parallel:

\begin{acode}
\afun[A]{monoidFold[A]}{mon: Monoid[A], x: List[A]}{
 \aifelse{empty(x)}{mon.e}{
   n := length(x)\\
   i := n \divop 2\\
   lower := [x_0,\ldots,x_{i-1}]\\
   upper := [x_i,\ldots,x_{n-1}]\\
   lowerFold := monoidFold(mon,lower) \tb | \tb upperFold := monoidFold(mon,upper)\\
   mon.op(lowerFold, upperFold)
 }
}
\end{acode}

If we have more CPUs than $n=length(x)$, this runs in $\Theta(\log n)$.

%More generally, we have $C^k(n)\in \Theat(\log n)+\Theta(n/k)


\chapter{Greedy Algorithms}\label{sec:ad:greedy}
 Greedy algorithms are an informal grouping of algorithms characterized by the following property: We have to make a number of choices until we are done, and at each step we choose the most attractive option.

A greedy algorithm emphasizes local optimization over global optimization: at each step it makes the optimal local choice.
That may or may not yield a good result overall.
For example, eating the cheapest available food every day saves money in the short run (locally optimal) but is not healthy in the long run (globally optimal).

But greedy algorithms are easy to implement and usually very efficient: making a locally optimal choice is usually much easier than making a globally optimal choice.
For example, finding the cheapest available food just requires browsing items in the market.
But finding out what food is healthy in the long run may require extensive research.
Humans have a strong tendency towards making local choices because they require so much fewer mental effort and yield immediate gratification.

\section{General Structure}

Consider a set $M$ with a weight function $w:M\to \N$ (or any other other set of positive numbers).
For a set $S\sq M$, we define the weight of $S$ by summing the weights of the elements.
 \[w(S)=\Sigma_{x\in S} w(x)\]

Moreover, consider a property $Acceptable$, i.e., if $S\sq M$, then $Acceptable(S)$ is a boolean.

Our goal is to find an acceptable subset of $M$ with maximal weight.

The generic greedy algorithm proceeds as follows:
\begin{acode}
\acomment{precondition: $elements$ is the list of all elements of $M$ sorted decreasingly by $w$}\\
\afun{greedy[M]}{elements:List[M], Acceptable: Set[M]\to\Bool}{
  solution := \anew{Set[M]}{}\\
  foreach(elements, \alam{x}{\aifI{Acceptable(solution \cup \{x\})}{insert(solution,x)}})\\
  solution
}
\end{acode}
Here $solution \cup \{x\}$ is an immutable operation that returns a new set, and $insert(solution,x)$ is a mutable operation that changes $solution$.

We can also use the function $greedy$ to find an acceptable subset with \emph{minimal} weight: we simply sort $elements$ by \emph{increasing} weight.

\section{Matroids}

To understand when the generic greedy algorithm yields an optimal results, we introduce matroids:

\begin{definition}[Matroid]
A \textbf{matroid} consists of
\begin{compactitem}
\item a finite set $M$
\item a property $Acceptable:Set[M]\to\B$ (subsets with this property are called \textbf{acceptable}\footnotemark)
\end{compactitem}
such that the following holds
\begin{compactitem}
 \item $M$ has at least one acceptable subset.
 \item $Acceptable$ is subset-closed, i.e., subsets of acceptable sets are also acceptable.
 \item If $A$ and $B$ are acceptable and $|A|>|B|$, then $B$ can be increased to an acceptable set $B\cup\{x\}$ by adding an element $x\in A$ to $B$. (Thus, we must have $x\in A\sm B$.)
\end{compactitem}
\end{definition}
\footnotetext{The literature calls them \emph{independent}, but \emph{acceptable} is a more intuitive name for greedy algorithms.}

The third property is the critical one: it guarantees that it does not matter which elements we add to an acceptable set, we always eventually get an acceptable set of maximal size.
Thus, local choices (which element to add) can never lead to a dead end.
More formally, we can state this as follows:
\begin{theorem}
We call an acceptable set $S$ that has no acceptable superset $S'\supset S$ a \textbf{base}.

Then, in a matroid, all bases have the same size.
\end{theorem}
\begin{proof}
Exercise.
\end{proof}

\section{Greedy Algorithms for Matroids}

\paragraph{Correctness}
Finding a base is always easy: start with the empty set and keep adding elements as long as the resulting set remains acceptable.

Now the matroid property guarantees that all bases have the same size.
Thus, it does not matter which elements we add---eventually we get an acceptable set of maximal size.

If we want to find not only an acceptable set of maximal \emph{size} but an acceptable set of maximal \emph{weight}, we simply add the elements in order of weight---that is exactly what the greedy algorithm does.
Formally, we have:

\begin{theorem}
If $M$ and $Acceptable$ form a matroid, then the greedy algorithm finds a base with greatest possible weight.
\end{theorem}

The corresponding theorem holds for finding the base with smallest possible weight.

\paragraph{Complexity}
The main structure of the greedy algorithm is linear in the number $|M|$ of elements.
But we also have to sort the elements once and check $Acceptable$ at every step.
We know sorting takes $\Theta(|M|\log |M|)$.
So if we can check $Acceptable(S)$ in $O(\log|S|)$, the overall run time (including sorting) is in $O(|M|\log|M|)$.

To be efficient, we usually implement the acceptability check in a greedy algorithm slightly smarter:
\begin{acode}
\aclass{Solution[M]}{}{}{
  \afun{insert}{m:M}{\ldots}\\
  \afun{acceptableWith}{m:M}{\ldots}
}\\
\\
\acomment{precondition: $elements$ is the list of all elements of $M$ sorted decreasingly by $w$}\\
\afun{greedy[M]}{elements:List[M]}{
  solution := \anew{Solution[M]}{}\\
  foreach(elements, \alam{x}{\aifI{solution.acceptableWith(x)}{solution.insert(x)}})\\
  solution
}
\end{acode}

Here $S.acceptableWith(x)$ is specified as follows:
\begin{compactitem}
 \item precondition: $Acceptable(S)$
 \item postcondition: if $S.acceptableWith(x)$, then $Acceptable(S\cup\{x\})$
\end{compactitem}
This works because the greedy algorithm only every needs to check $Acceptable(S\cup\{x\})$ and only if $S$ is already known to be acceptable.

$S.acceptableWith(x)$ can often be implemented much faster than $Acceptable(S\cup\{x\})$ because:
\begin{compactitem}
 \item We do not have to copy the set $S$ to build $S\cup\{x\}$.
 \item The acceptability check can use the information that $S$ is already acceptable.
\end{compactitem}

\section{Examples}

\subsection{Kruskal's Algorithm}

Kruskal's algorithm from Sect.~\ref{sec:ad:spanningtree} is a simple example of a greedy algorithm.

\paragraph{Correctness}
To show that it is correct, we only have to show that it operates on a matroid.

We use the following matroid:
\begin{compactitem}
 \item The set $M$ is the set of edges of the graph $G=(N,E)$.
 \item A set $S\sq E$ is acceptable if the graph $(N,S)$ is a set of trees.
\end{compactitem}

We have to prove the matroid properties:
\begin{compactitem}
 \item There is an acceptable set. For example, $(N,\es)$ is a graph where every node is a tree by itself.
 \item Subsets of acceptable sets are acceptable. Taking an edge away from a tree splits it into two trees. Thus, removing edges from a set of trees again yields a set of trees.
 \item For the critical third property, assume that $(N,A)$ and $(N,B)$ are sets of trees such that $A$ contains more edges than $B$.
  We have to find an edge $x\in A\sm B$ that we can add to $B$.
  We pick any $x\in A$ that connects two nodes that are not in the same tree in $(N,B)$.
  Then $(N,B\cup\{x\})$ is a set of trees again.
  We only have to check that such an $x$ exists: If there were no such $x$, the trees in $A$ and $B$ would consist of the same nodes; but then $A$ cannot have more edges then $B$ because the number of edges in a tree is already fixed by the number of nodes.
\end{compactitem}

Thus, we immediately know that Kruskal's algorithm is correct.

\paragraph{Complexity}
We have $Acceptable=isSetOfTrees$.

To improve efficiency, we implement an appropriate data structure for the class $Solution$.
This is indeed possible in $O(\log|S|)$.
Then we obtain $\Theta(|E|\log|E|)$ as the overall run time of Kruskal's algorithm.

The idea behind the implementation is that $S.acceptableWith(x)$ only has to check that $x$ connects two nodes from different trees.
By cleverly storing the set of trees, we can check that in $O(\log|S|)$.

\subsection{Dijkstra's Algorithm}

Because the term \emph{greedy algorithm} is not defined precisely, not every algorithm that has a greedy flavor is a special case of the matroid algorithm.

A counter-example is Dijkstra's algorithm from Sect.~\ref{sec:ad:shortestpath}.

\subsection{Scheduling with Deadlines and Penalties}

\paragraph{Problem}
We are given $n$ tasks.
Each task takes the same amount of time (e.g., $1$ day), and all tasks have to be done separately without overlap (e.g., it takes $n$ days in total).
We want to find the best order in which to do all tasks.

Each task has a deadline: Task $i$ must be completed by time $D(i)\in \N$ for $0<D(i)\leq n$.
Otherwise, we have to pay a penalty $w(i)$ for $w(i)>0$.

Our goal is to minimize the total penalty we have to pay (i.e., the sum of all $w(i)$ for all $i$ that are done after $D(i)$).
Equivalently, we want to maximize the total penalty that we do not have to pay (i.e., the sum of all $w(i)$ for all $i$ that are done by $D(i)$).

\paragraph{Design}
We define a matroid as follows:
\begin{compactitem}
 \item $M$ is the set $\{1,\ldots,n\}$ representing the $n$ tasks.
 \item A set $S\sq M$ is acceptable if it is possible to do all tasks in $S$ on time.
\end{compactitem}

%Note that we have the following:
%\begin{theorem}
%A set $S$ is acceptable if and only if for every $0\leq t\leq n$ there are at most $t$ tasks whose deadline is at or before $t$.
%\end{theorem}
%\begin{proof}
%Exercise.
%\end{proof}

We instantiate the generic greedy algorithm using this matroid and weight function $w$.
The algorithm returns the optimal set $S_{opt}$.

We now schedule the tasks as follows: First we schedule all tasks in $S_{opt}$ in some way that they are all done on time. (This is possible because the greedy algorithm returns an acceptable set.)
Then we schedule all other tasks (all of which will be late because the greedy algorithm returns a maximally big acceptable set) arbitrarily.

\begin{example}
Consider $5$ tasks as follows
\begin{center}
\begin{tabular}{c|ccccc}
task $i$ & 1 & 2 & 3 & 4 & 5 \\
\hline
deadline $D(i)$ & 3 & 1 & 3 & 2 & 2 \\
penalty $w(i)$ & 10 & 20 & 5 & 25 & 15
\end{tabular}
\end{center}

Ordered by decreasing penalty the tasks are $[4,2,5,1,3]$.
The greedy algorithm proceeds as follows:
\begin{center}
\begin{tabular}{|lll|}
\hline
considered task & decision & known schedule \\
\hline
4 & insert & $?4???$ \\
2 & insert & $24???$ \\
5 & skip   & $24???$ \\
1 & insert & $241??$ \\
3 & skip   & $241??$ \\
\hline
\end{tabular}
\end{center}
Here the right column tracks not only the set $solution$ but also the best schedule for its element.
That makes it easy to implement $acceptableWith$ by checking decide whether there is still a slot available before the deadline.

The algorithm return $S_{opt}=\{1,2,4\}$.
Finally, we insert the remaining tasks at the end obtaining the schedule $24135$.
We have to pay the penalty $w(3)+w(5)=20$.
\end{example}


\paragraph{Correctness}
We only have to show that the above structure is indeed a matroid.
This is left as an exercise.

%\subsection{Greatest Flow}
%
%Consider the greatest flow problem from Sect.~\ref{sec:ad:maximalflow}.
%We want to construct a greedy algorithm for it by identifying an appropriate matroid structure.
%
%For a path $p$ from $source$ to $sink$, we write $Cap(p)$ for the capacity of $p$.
%Recall that the capacity of a path is the minimum of the weights of the edges of $p$.
%
%We want to describe a flow as a set $S$ of such paths.
%Along each path of $S$, we try to flow as much as the capacity of the path.
%But we have to make sure we use each edge $e$ only up to its capacity $w(e)$.
%Because $e$ may be part of multiple paths $p\in S$, the sum of the capacities of these paths $p$ must stay below $w(e)$.
%
%That leads us to using the following matroid:
%\begin{compactitem}
%  \item $M$ is the set of paths from the source to the sink.
%  \item A set $S$ of paths is acceptable if for every edge $e\in E$
%   \[\Sigma_{p\in S\mwith e\in p} Cap(p)\leq w(e)\]
%\end{compactitem}
%Then every acceptable set yields a flow.
%
%We have to prove the matroid properties:
%\begin{compactitem}
%  \item There is an acceptable set, e.g., the empty set.
%  \item Subsets of acceptable sets are acceptable because lowering the flow cannot introduce a violation of the capacity conditions.
%  \item Consider two sets $A$ and $B$ of paths such that $|A|>|B|$.
%  
%\end{compactitem}

\chapter{Dynamic Programming}\label{sec:ad:dynprog}
 %matrix powers for min-+ or Floyd Warshall for shortest path


\chapter{Protocols}\label{sec:ad:prot}
 \input{protocol}

\chapter{Randomization}\label{sec:ad:random}
 \input{random}
 
\chapter{Quantum Algorithms}\label{sec:ad:quantum}
 \input{quantum}

%\part{Concrete Languages}

%\chapter{Data Description Languages}
% 
% \section{JSON}
%
% \section{XML}
% 
% \section{UML}
%
%\chapter{Programming Languages}
 

%\part{Conclusion}
%
%\chapter{Summary}
%  \input{summary-abs}

\part{Appendix}

\appendix

\chapter{Mathematical Preliminaries}\label{sec:math}
\input{../math/all}


\tocentryBib

\input{\currfilebase.bblp}
%\bibliographystyle{alpha}
%\bibliography{../../../../../frabe/Program_Data/Latex/bib/rabe,../../../../../frabe/Program_Data/Latex/bib/historical}
%\bibliography{../../../../../homedir/Program_Data/Latex/bib/rabe,../../../../../homedir/Program_Data/Latex/bib/historical}

\end{document}