The basic idea of asymmetric encryption is that different keys are used for encryption and decryption.
An important requirement for security is that the decryption key cannot be computed from the encryption key.

This has the practical advantage that the---not security-critical---encryption key can be made public.

\subsection{Schemes}

We only need to make a minor modification to the Def.~\ref{def:sd:symscheme}:

\begin{definition}[Asymmetric Encryption Scheme]\label{def:sd:asymscheme}
 An \textbf{asymmetric encryption scheme} is an encryption scheme $(\Sigma,K, G, E, D)$, where
  \begin{compactitem}
   \item $K_n=K^e_n\times K^d_n$
   \item if $k=(k^e,k^d)$, then $E_k$ depends only on $k^e$ and $D_k$ depends only $k^d$
  \end{compactitem}
\end{definition}
Intuitively, the keys are pairs of an encryption and a decryption key.

\subsection{Schemes based on Modular Arithmetic}

We fix the alphabet to be $\Sigma=\{0,1\}$ and identify the elements of $\Sigma^n$ and with natural numbers in binary representation.

\subsubsection{Modular Arithmetic as a Cipher}

To encrypt plaintexts from $\Sigma^n$, we proceed in two steps:
\begin{compactenum}
 \item We embed the plaintext into $\Z_N$ for certain $N>2^n$.
 \item We apply a cipher function $\Z_N\to\Z_N$.
\end{compactenum}

$N$ is part of the key.
The number of bits in $N$ is usually seen as the size of the key.
This size must be greater than $n$.

The embedding introduces randomness to make sure that encrypting the same plaintext multiple times yields different ciphertexts.
The cipher function is the main source of security.

\subsubsection{Mode of Operation and Padding}

\paragraph{No Mode of Operation}
In principle, we could use the same modes of operation as for block ciphers.
However, that is not common:
\begin{compactitem}
 \item Asymmetric encryption has the advantage that the encryption key can be made public.
 To exploit that advantage, it is practical not to change the key very often.
 In that case, using the same key for many blocks of many long message increases the chance of attacks.
 \item Asymmetric encryption tends to be more complex than symmetric encryption.
 Therefore, it is not practical to use asymmetric encryption for many long messages.
\end{compactitem}
Instead, it makes more sense to use \textbf{hybrid encryption}: use an asymmetric scheme only to transmit the key for a symmetric scheme.
In that case, it is not necessary to be able to send arbitrarily long messages with the same fixed key.
Instead, a fixed message length suffices as long as it can hold the symmetric key.

\paragraph{Padding}
For the same reason as with symmetric schemes, some randomness must be introduced into the encryption function to make sure that repeatedly mapping the same plaintext yields different ciphertexts.
That is the role of the \textbf{padding}, which embeds the plaintext from $\Z_{2^n}$ into $\Z_N$.

The basic idea is to pick $N>2^{n+i}$, append $i$ $0$s to the plaintext and then apply some cipher $\Z_{2^{n+i}}\to\Z_{2^{n+1}}$ (e.g., a substitution-permutation network with a randomly chosen parameter).

The details of padding are very difficult and subject to active research.
The most important example is Optimal Asymmetric Encryption Padding (OAEP).
Multiple padding schemes based on OAEP have been standardized in RFC 447: Public-Key Cryptography Standards (PKCS) \#1.

\subsubsection{RSA}

\paragraph{Overview}
RSA is a family of cipher functions based on modular arithmetic.

It was developed in the 1970s inspired by ideas by Diffie and Hellman.
It is named after the authors of the algorithm (Rivest, Shamir, Adleman).
A related algorithm was developed earlier by the UK secrete service but remained classified until the 1990s.

The basic ideas is to use $N=p\cdot q$ for large prime numbers $p$ and $q$.
Becuase it is (assumed to be) very difficult to compute $p$ and $q$ from $N$, $p$ and $q$ remain private even if $N$ is public.

\paragraph{Key Generation}
To compute $G(n)$, we first randomly choose two large primes $p$ and $q$ (typically of roughly equal size) such that $p\cdot q>2^{n+i}$ where $i$ is the desired number of padding bits.
We put $N=p\cdot q$.

Secondly, we put $m=(p-1)(q-1)$. (Actually, any common multiple of $p-1$ and $q-1$ is fine.)
Note that $m=\phi(N)$.
Then we pick $e\in \Z_m$ such that $\gcd(e,m)=1$ and compute the $d\in\Z_m$ with $e\cdot d\Equiv_m 1$.
Such a $d$ exists because $\gcd(e,m)=1$ and is easy to compute (see Thm.~\ref{thm:math:extendedeuclid}).

The keys are now defined as follows:
\begin{compactitem}
 \item public key (encryption key): $(N,e)$
 \item private key (decryption key): $(N,d)$
\end{compactitem}
$m$, $p$, and $q$ are not needed for encryption or decryption but must remain private: $p$ (or $q$) is enough to compute $m$ from $N$, and $m$ is enough to compute $d$ from $e$.
So we delete $m$, $p$, and $q$ after generating the key.

Note that to choose $p$ and $q$ efficiently, it is important to have access to a fast primality test.

\paragraph{Encryption and Decryption}
The cipher function and its inverse are the functions $\Z_N\to \Z_N$ defined by
\begin{compactitem}
 \item encryption: $x\mapsto x^e\modop N$
 \item decryption: $x\mapsto x^d\modop N$
\end{compactitem}

Both can be computed efficiently, e.g., using square-and-multiply.

These are indeed inverse to each other:

\begin{theorem}
For all $x\in \Z_N$, we have $(x^d)^e\Equiv_N (x^e)^d \Equiv_N x$.
\end{theorem}
\begin{proof}
In general, because $N=p\cdot q$ for prime numbers $p$ and $q$, we have that $x\Equiv_N y$ iff $x\Equiv_p y$ and $x\Equiv_q y$.

So we have to show that $x^{de}\Equiv_p x$.
(We also have to show the same result for $q$, but the proof is the same.)
We distinguish two cases:
\begin{compactitem}
\item $p|x$: Then trivially $x^{de}\Equiv_p 0\Equiv_p x$.
\item Otherwise. Then $p$ and $x$ are coprime.\\
   By construction of $e$ and $d$ and using Thm.~\ref{thm:math:extendedeuclid}, we have $k\in\Z$ such that $e\cdot d+k\cdot m=1$.
   Using $m=(p-1)(q-1)$, we obtain $x^{de}=x\cdot (x^{p-1})^{-k\cdot(q-1)}$.
   That yields $x^{de}\Equiv_p x$ by using $x^{p-1}\Equiv_p 1$ as known from Thm.~\ref{thm:math:fermatlittle}.
\end{compactitem}
\end{proof}

\paragraph{Attacks}
To break RSA, $d$ has to be computed.
There are $3$ natural ways to do that:
\begin{compactitem}
 \item Factor $N$ into $p$ and $q$. Then compute $d$ easily.
 \item Compute $n$ using $n=\phi(N)$ (which may be easier than finding $p$ and $q$). Then compute $d$ easily.
 \item Find $d$ such that $e\cdot d\Equiv_n 1$ (which may be easier than finding $n$).
\end{compactitem}
Currently these are believed to be equally hard.

It is believed that there is no algorithm for factoring $N$ that is polynomial in the number of bits of $N$.
That is not proved.
There are hypothetical machines (e.g., quantum computers) that can factor $N$ polynomially.

Note that checking if $N$ can be factored (without producing the factors) is polynomial, and practical algorithms exist (in particular, the AKS algorithm).
Incidentally, that is important to find the large prime number $p$ and $q$ efficiently.
\medskip

If there is indeed no polynomial algorithm, factoring relies on brute-force attacks that find all prime numbers $k<\sqrt{N}$ and test $k|N$.
Therefore, larger keys are harder than break to smaller ones.
Because of improving hardware, the key size that is considered secure grows over time.

Keys of size $1024$ are considered secure today, but because security is a relative term, keys of size $2048$ are often recommended. 
%It is quetionable that 1024-bit rsa is really secure, even though it has not yet been publicly broken;
%(see for instance https://en.wikipedia.org/wiki/RSA_%28cryptosystem%29#Integer_factorization_and_RSA_problem or https: //www.schneier.com/blog/archives/2007/05/307digit_number.html).
Larger keys are especially important if data is needed to remain secure far into the future, when faster hardware will be available.
